%?deps @bib-commons

\documentclass[11pt]{article}
\usepackage{adjustbox,aliascnt,amsmath,amssymb,amsthm,calc,colonequals,enumitem,mathtools,mleftright,parskip,textgreek,xcolor}
\usepackage[a4paper,left=1.3in,right=1.3in,top=1.25in,bottom=1.25in,footskip=0.6in]{geometry}
\usepackage[en-GB]{datetime2}
\usepackage[unicode]{hyperref}

\usepackage{bib-commons}

\title{$\Sigma^1_1$-computability and realisability of a global well-ordering \\[6pt] \normalsize (Abstract only)}
\author{Shuwei Wang \\[2pt] \small University of Leeds}
\date{\DTMdate{2024-07-04}}

\begin{document}

\maketitle

Using Kleene's normal form theorem for $\Sigma^1_1$-formulae, we can organise the collection of $\Sigma^1_1$-definable partial functions into a partial combinatory algebra. This can then be used as a realisability structure to interpret intuitionistic theories of arithmetic that allow comprehension over all arithmetic formulae.

In this talk, I will use $\mathrm{CM}$, a novel formal system of intuitionistic third-order arithmetic with a classical first-order fragment in Weaver \cite{weaver09-cm}, as an example to discuss the increase in power of this realisability model as we move to stronger fragments of second-order arithmetic in the meta-theory. The goal is to interpret an axiom of global well-ordering on all second-order objects, which functions as a choice-like extension of $\mathrm{CM}$ and provides an analogue of Zorn's lemma for the theory.

We will demonstrate that this approach gives an ordinal analysis of Weaver's theory. $\mathrm{CM}$ itself will have the proof-theoretic ordinal $\varphi_{\varepsilon_0}0$, while the addition of the global well-ordering will increase its strength to the Bachmann--Howard ordinal.

\bibmain

\end{document}
