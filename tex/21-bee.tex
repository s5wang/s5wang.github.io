\documentclass[12pt]{article}

\usepackage{aliascnt,amsmath,amssymb,amsthm,calc,censor,hyperref,mleftright,parskip}

\usepackage[a4paper,left=1.45in,right=1.45in,top=1.25in,bottom=1.25in,footskip=0.6in]{geometry}

\newenvironment{packeditemize}{%
	\begin{itemize}
	\setlength{\parskip}{2pt}
}{\end{itemize}}

\newcommand*{\fullref}[1]{\hyperref[{#1}]{\autoref*{#1} (\nameref*{#1})}}

\renewcommand{\thefootnote}{(\arabic{footnote})}

\makeatletter
\def\thm@space@setup{%
  \thm@preskip=\parskip \thm@postskip=0pt
}

\newtheorem*{rep@theorem}{\rep@title}
\newcommand{\newreptheorem}[2]{%
\newenvironment{rep#1}[1]{%
 \def\rep@title{#2 \ref{##1}}%
 \begin{rep@theorem}
}{\end{rep@theorem}}}

\renewcommand{\maketitle}{%
	\begin{titlepage}
		\vspace*{1.2cm}
		\begin{center}
			{\large Part B Extended Essay}\\[.3cm]
			\rule{\textwidth}{1.5pt}\\[0cm]
			{\Large \bfseries \@title \par \ }\\[-.8cm]
			\rule{\textwidth}{1.5pt}\\[1.2cm]
			{\large
				\lineskip .5em%
				\begin{tabular}[t]{c}%
					\@author
				\end{tabular}
				\par \ }\\[-.3cm]
			{\large \@date}
			\vfill
			\vfill
			{\small Final word count: 7481}\\[.1cm]
			{\footnotesize Including words outside text (captions, etc.), excluding formulae}\\[.04cm]
			{\footnotesize Word count generated by TeXcount version 3.2}
			\vfill
		\end{center}
	\end{titlepage}
}
\makeatother

\newcommand{\newaliastheorem}[2]{%
  \newaliascnt{#1}{theorem}
  \newtheorem{#1}[#1]{#2}
	\newreptheorem{#1}{#2}
  \aliascntresetthe{#1}
  \expandafter\def\csname #1autorefname\endcsname{#2}
}

\theoremstyle{plain}
\newtheorem{theorem}{Theorem}[section]
\newreptheorem{theorem}{Theorem}
\newtheorem*{theorem*}{Theorem}
\newaliastheorem{proposition}{Proposition}
\newaliastheorem{corollary}{Corollary}
\newaliastheorem{lemma}{Lemma}
\newtheorem{fact}{Fact}
\def\factautorefname{Fact}

\theoremstyle{definition}
\newaliastheorem{definition}{Definition}

\DeclareMathOperator{\Def}{Def}
\DeclareMathOperator{\Formula}{Form}
\DeclareMathOperator{\Free}{Free}
\DeclareMathOperator{\Diag}{Diag}

\DeclareMathOperator{\closure}{cl}
\DeclareMathOperator{\interior}{int}
\DeclareMathOperator{\boundary}{bd}
\DeclareMathOperator{\frontier}{fr}

\DeclareMathOperator{\Int}{Int}

\DeclareMathOperator{\Real}{Re}
\DeclareMathOperator{\Imag}{Im}

\DeclareMathOperator{\dom}{dom}

\newcommand{\RCF}{\mathrm{RCF}}

\title{The Application of Model-Theoretical O-Minimality to Complex Analysis}
\author{MMathPhil Candidate \censor*{7}}
\date{Year 2020/21}

\begin{document}

\raggedbottom

\maketitle

\tableofcontents
\thispagestyle{empty}
\newpage

\pagenumbering{arabic}
\flushbottom

\section{Introduction}

O-minimal theories are model-theoretical accounts of densely ordered algebraic structures where definable sets consist of finitely many (definably) connected components. In o-minimal structures, one can identify a natural topology as a nice candidate for Grothendieck's conception of tame topology in section 5 of \cite{sketch-of-a-programme}. This essay will discuss the topological properties of o-minimal spaces and apply them to a ``complex-like'' space\footnote{By ``complex-like'', we mean $K = R\mleft(i\mright)$ as the algebraic closure of some arbitrary real closed field $R$.} $K$:

In \autoref{sec:o-minimal-structures}, I will follow van den Dries' book \cite{tame-topology} to provide a formal definition for o-minimal structures and the topology on them. I will work towards the central results on o-minimal spaces, namely:

\begin{reptheorem}{thm:cell-decomposition}
	Given an o-minimal structure $\mathfrak{A} = \left\langle A; <, \ldots \right\rangle$ and a finite collection of definable subsets $U_1, \ldots, U_k \subseteq A^n$, there exists a cell decomposition $S$ of $A^n$ that partitions each $U_i$.
\end{reptheorem}

In \autoref{sec:definable-complex-analysis}, I will utilise Peterzil and Starchenko's approach in \cite{differentiability-theory} and explain how o-minimal properties can be used to establish and possibly strengthen classical complex analysis results on the algebraic closure of arbitrary real closed fields.

Finally, in \autoref{sec:applications}, I will discuss to what extent the theory of definable sets and functions is useful in classical analysis. Based on some excellent summary of results in this area in \cite{tame-complex-analysis-summary}, I will examine analytic functions that are made definable in various o-minimal expansions of $\mathbb{R}$.

\section{Properties of o-minimal structures}
\label{sec:o-minimal-structures}

\subsection{Preliminaries}

In the entirety of this essay I will work within the model theory of \emph{first-order predicate logic with equality}\footnote{In this essay we will work mostly with specific models, so we will not bother with distinguishing equality symbols in a model and the meta-language.}, where expressions consist of $\neg$, $\land$, $\lor$, $\rightarrow$, $\leftrightarrow$, $\forall$, $\exists$, $=$, $\top$, $\bot$ in addition to variables and the non-logical symbols; we admit predicate and functional symbols in the language, with propositional symbols realised as 0-place predicates and constants realised as 0-place functions.

Given a first-order language $L$, we use $\Formula\mleft(L\mright)$ to denote the set of all well-formed formulae in the language. We denote a formula as $\varphi\mleft(x_1, \ldots, x_n\mright) \in \Formula\mleft(L\mright)$, where $\Free\mleft(\varphi\mright) \subseteq \left\{x_1, \ldots, x_n\right\}$ is the set of free variables\footnote{Here $x_1, \ldots, x_n$ need not all appear in $\varphi$.} in $\varphi$.

A structure in the language is denote as $\mathfrak{A} = \left\langle A; \ldots \right\rangle$, where $A$ is the universe. For a formula $\varphi\mleft(x_1, \ldots, x_n\mright) \in \Formula\mleft(L\mright)$, we say that the set
\[\varphi^{\mathfrak{A}} = \left\{\left(a_1, \ldots, a_n\right) \in A^n : \mathfrak{A} \models \varphi\left(a_1, \ldots, a_n\right)\right\}\]
is \emph{defined by $\varphi$}.

\subsection{O-minimal structures}

\begin{definition}
	Given a structure $\mathfrak{A} = \left\langle A; \ldots \right\rangle$ in a first-order language $L$ and a subset $C \subseteq A$, the \emph{extended structure $\mathfrak{A}_C$ with constants from $C$} lives in the language expansion $L_C$ with additional symbols $c_a$ for each $a \in C$. For each $a \in C$, $\mathfrak{A}_C$ assigns the value $a$ to constant symbol $c_a$.

	In practice, we usually write each constant symbol $c_a$ as $a$ directly, when there is no ambiguity. For example, with the structure of reals $\mathfrak{R} = \left\langle \mathbb{R}; \ldots \right\rangle$, we view $x + 1.5 = 3$ as a valid formula in the extended structure $\mathfrak{R}_{\mathbb{R}}$, where $1.5$ and $3$ represent corresponding constant symbols.
\end{definition}

\begin{definition}[Definable sets]
	Given a structure $\mathfrak{A} = \left\langle A; \ldots \right\rangle$ in a language $L$ and some $n \in \mathbb{N}$, we denote the \emph{definable subsets of $A^n$} as
	\[\Def_n\mleft(\mathfrak{A}\mright) = \{\varphi^{\mathfrak{A}} : \varphi\mleft(x_1, \ldots, x_n\mright) \in \Formula\mleft(L\mright)\}.\]

	We call $\Def\mleft(\mathfrak{A}\mright) = \bigcup_i \Def_i\mleft(\mathfrak{A}\mright)$ the \emph{definable sets in the structure $\mathfrak{A}$}.

	For a function $f : U \rightarrow A^m$ on some $U \subseteq A^n$, we say that $f$ is \emph{definable} if its graph
	\[\Gamma\mleft(f\mright) = \left\{\left(x, f\mleft(x\mright)\right) : x \in U\right\} \subseteq A^{n + m}\]
	is a definable set.
\end{definition}

We will be mainly concerned with definability with constants in this essay. Thus, from here onwards, when we say that a set/function is definable in a structure $\mathfrak{A} = \left\langle A; \ldots \right\rangle$, we mean definability with constants, i.e.\ in $\mathfrak{A}_A$.

\begin{definition}[O-minimality]
	Consider a structure $\mathfrak{A} = \left\langle A; <, \ldots \right\rangle$, on which $<$ defines a dense linear order without endpoints. Extending $A$ to endpoints $-\infty, \infty$, we say that a subset $I \subseteq A$ is an (open) interval if there exists $a < b \in A_\infty = A \cup \left\{-\infty, \infty\right\}$ such that
	\[I = \left(a, b\right) = \left\{c \in A : a < c < b\right\}.\]

	We say that $\mathfrak{A}$ is \emph{o-minimal} if the definable subsets of $A$, namely $\Def_1\mleft(\mathfrak{A}_A\mright)$, consist only of finite unions of singletons and intervals.
\end{definition}

The following, for example, could be an immediate reason why the o-minimal condition can be useful:

\begin{proposition}[Dedekind completeness for definable sets]
	\label{prop:dedekind-completeness}
	Given an o-minimal structure $\mathfrak{A} = \left\langle A; <, \ldots \right\rangle$ and a non-empty definable set $U \subseteq A$, then $\sup\mleft(U\mright)$ and $\inf\mleft(U\mright)$ exist in $A \cup \left\{-\infty, \infty\right\}$.
\end{proposition}

\begin{proof}
	By o-minimality, we can write
	\[U = \left(\bigcup_{i = 1}^k \left(p_i, q_i\right)\right) \cup \left(\bigcup_{j = 1}^\ell \left\{r_j\right\}\right)\]
	where each $\left(p_i, q_i\right)$ is an interval and each $r_j \in A$ is an element. Obviously
	\begin{align*}
		\sup\mleft(U\mright) & = \max\mleft(\left\{\sup\left(p_i, q_i\right) : 1 \leq i \leq k\right\} \cup \left\{\sup\left\{r_j\right\} : 1 \leq j \leq \ell\right\}\mright) \\
		                     & = \max\mleft(\left\{q_i : 1 \leq i \leq k\right\} \cup \left\{r_j : 1 \leq j \leq \ell\right\}\mright),                                         \\
		\inf\mleft(U\mright) & = \min\mleft(\left\{\inf\left(p_i, q_i\right) : 1 \leq i \leq k\right\} \cup \left\{\inf\left\{r_j\right\} : 1 \leq j \leq \ell\right\}\mright) \\
		                     & = \min\mleft(\left\{p_i : 1 \leq i \leq k\right\} \cup \left\{r_j : 1 \leq j \leq \ell\right\}\mright)
	\end{align*}
	both exist in $A \cup \left\{-\infty, \infty\right\}$.
\end{proof}

\subsection{Interval topology}

\begin{definition}[Interval topology]
	Given a structure $\mathfrak{A} = \left\langle A; <, \ldots \right\rangle$, it is easy to verify that the (open) intervals on $A$ form a basis for a topology. We call this the \emph{interval topology on $A$}.

	We say that a subset $B \subseteq A^n$ is a \emph{box} if there exist intervals $I_1, \ldots, I_n \subseteq A$ such that $B = \prod_i I_i$. The boxes form a basis for the induced product topology on $A^n$. We say that this is the \emph{interval topology on $A^n$}.
\end{definition}

This topology is nice on an o-minimal structure. As Lemma 3.3(ii) in chapter 1 of van den Dries' book \cite{tame-topology}, any $1$-dimensional definable set decomposes into finitely many components:

\begin{proposition}
	\label{prop:finite-boundary}
	Consider an o-minimal structure $\mathfrak{A} = \left\langle A; <, \ldots \right\rangle$ and a definable set $U \subseteq A$. The boundary of $U$ under the interval topology,
	\[\boundary\mleft(U\mright) = \closure\mleft(U\mright) \setminus \interior\mleft(U\mright)\]
	i.e. its closure minus its interior, must be finite. Additionally, if we enumerate $\boundary\mleft(U\mright) \cup \left\{-\infty, \infty\right\}$ as
	\[-\infty = a_0 < a_1 < \cdots < a_{m + 1} = \infty,\]
	then for each $0 \leq i \leq m$, the interval $\left(a_i, a_{i + 1}\right)$ lies either entirely in $U$ or entirely in $A \setminus U$.
\end{proposition}

However, these components may not be connected in the classical sense. For example, as we will later show in \autoref{thm:rcf-o-minimal}, the set of algebraic numbers $\mathbb{R}_{\textnormal{alg}} \subseteq \mathbb{R}$ forms an o-minimal structure. However, intervals in $\mathbb{R}_{\textnormal{alg}}$ are disconnected due to the ``gaps'' of transcendental numbers. This motivates the following stipulation:

\begin{definition}
	Given a structure $\mathfrak{A} = \left\langle A; <, \ldots \right\rangle$ with the interval topology, we say that a definable set $X \subseteq A^n$ is \emph{definably disconnected} if there exists disjoint non-empty definable sets $U, V \subseteq X$ that are open in $X$, with $X = U \cup V$. Otherwise, we say that it is \emph{definably connected}.
\end{definition}

Due to \autoref{prop:dedekind-completeness}, we can now replicate exactly the proof that intervals in $\mathbb{R}$ are connected, and prove that:

\begin{proposition}
	\label{prop:interval-definably-connected}
	Consider an o-minimal structure $\mathfrak{A} = \left\langle A; <, \ldots \right\rangle$. For any $a < b \in A$, the interval $\left(a, b\right)$ is definably connected.
\end{proposition}

\subsection{Cell decomposition}

\begin{definition}
	Given a structure $\mathfrak{A} = \left\langle A; <, \ldots \right\rangle$ and any $U \subset A^n$, let $C\mleft(U\mright)$ denote the set of definable continuous functions $U \rightarrow A$; we additionally write
	\[C_\infty\mleft(U\mright) = C\mleft(U\mright) \cup \left\{-\infty, \infty\right\},\]
	where $-\infty, \infty$ refer to constant functions from $U$ to $A \cup \left\{-\infty, \infty\right\}$.

	A natural partial ordering on $C_\infty\mleft(U\mright)$ exists such that $f_1 < f_2 \in C_\infty\mleft(U\mright)$ if for any $x \in U$, $f_1\mleft(x\mright) < f_2\mleft(x\mright)$.
\end{definition}

\begin{definition}[Cells]
	For a finite sequence $i_1, \ldots, i_n \in \left\{0, 1\right\}$, we define an \emph{$\left(i_1, \ldots, i_n\right)$-cell} on a structure $\mathfrak{A} = \left\langle A; <, \ldots \right\rangle$ recursively:
	\begin{packeditemize}
		\item A $\left(0\right)$-cell is a singleton in $A$; a $\left(1\right)$-cell is a non-empty (open) interval in $A$.
		\item A subset $U \subseteq A^n$ is an $\left(i_1, \ldots, i_{n - 1}, 0\right)$-cell if there is an $\left(i_1, \ldots, i_{n - 1}\right)$-cell $U_0 \subseteq A^{n - 1}$ and a continuous definable function $f \in C\mleft(U_0\mright)$ such that
		\[U = \Gamma\mleft(f\mright) = \left\{\left(x, f\mleft(x\mright)\right) : x \in U_0\right\}\]
		is the graph of $f$; $U \subseteq A^n$ is an $\left(i_1, \ldots, i_{n - 1}, 1\right)$-cell if there is an $\left(i_1, \ldots, i_{n - 1}\right)$-cell $U_0 \subseteq A^{n - 1}$ and continuous definable functions $f < g \in C_\infty\mleft(U_0\mright)$ such that
		\[U = \left\{\left(x, y\right) \in U_0 \times A : f\mleft(x\mright) < y < g\mleft(x\mright)\right\}.\]
	\end{packeditemize}

	Finally, we say that any $U \subseteq A^n$ is a cell if it is a $\left(i_1, \ldots, i_n\right)$-cell for some $i_1, \ldots, i_n \in \left\{0, 1\right\}$.
\end{definition}

Also, since both graphs of functions and the ``spaces'' between two functions are easily formalised in first-order logic, cells are trivially definable.

They can additionally be used as a higher-dimensional analog of intervals and singletons, due to the following obvious, nice properties:

\begin{proposition}
	Given an o-minimal structure $\mathfrak{A} = \left\langle A; <, \ldots \right\rangle$, a cell $U \subseteq A^n$ is definably connected.
\end{proposition}

\begin{proposition}
	\label{prop:open-cell-condition}
	A cell is open if and only if it is a $\left(1, 1, \ldots, 1\right)$-cell; all other cells are nowhere dense in $A^n$.
\end{proposition}

\begin{corollary}
	Any $\left(i_1, \ldots, i_n\right)$-cell $U$ is homeomorphic to an open cell under some projection map $\pi_{\left(U\right)} : A^n \rightarrow A^r$, simply projecting onto the coordinates where $i_k = 1$. We have $r = n$ if and only if $U$ is open.

	In later uses we shall refer to this as the \emph{natural homeomorphism} for $U$.
\end{corollary}

\begin{corollary}
	\label{cor:open-set-decomposition}
	If a non-empty definable open subset $U \subseteq A^n$ is written as a finite union of cells, $U = \bigcup S$, then $S$ contains an open cell.
\end{corollary}

Now, we can proceed to establish the central tameness result for the interval topology on an o-minimal structure:

\begin{definition}[Cell decomposition]
	On a structure $\mathfrak{A} = \left\langle A; <, \ldots \right\rangle$, consider a partition of $A^n$ into a finite set $S$ of disjoint cells, such that $\bigcup S = A^n$. We define the condition for $S$ to be a \emph{cell decomposition} of $A^n$ recursively:
	\begin{packeditemize}
		\item Any finite partition of $A$ into disjoint cells, i.e.\ singletons and intervals, is a cell decomposition of $A$.
		\item A finite partition $S$ of $A^n$ into disjoint cells is a cell decomposition of $A^n$ if
		\[\pi\mleft(S\mright) = \left\{\pi\mleft(U\mright) : U \in S\right\}\]
		is a cell decomposition of $A^{n - 1}$, where $\pi$ is the projection onto the first $\left(n - 1\right)$ coordinates.
	\end{packeditemize}
\end{definition}

\begin{theorem}[Cell decomposition theorem]
	\label{thm:cell-decomposition}
	Given an o-minimal structure $\mathfrak{A} = \left\langle A; <, \ldots \right\rangle$ and a finite collection of definable subsets $U_1, \ldots, U_k \subseteq A^n$, there exists a cell decomposition $S$ of $A^n$ that partitions each $U_i$, i.e.\ such that each $U_i = \bigcup S_i$ for some $S_i \subseteq S$.
\end{theorem}

The proof of this theorem is lengthy and technical, so we are going to provide here just a sketch of van den Dries' inductive proof in chapter 3 of \cite{tame-topology}.

To begin, notice that the $1$-dimensional case is simply covered by \autoref{prop:finite-boundary}. In the inductive case, van den Dries made use of the following definitions:

\begin{definition}
	We say that a set $Y \subseteq A^{n + 1}$ is \emph{finite over $A^n$} if for each $x \in A^n$ the fibre $Y_x = \left\{y \in A : \left(x, y\right) \in Y\right\}$ is finite.
\end{definition}

\begin{definition}
	Given a structure $\mathfrak{A} = \left\langle A; <, \ldots \right\rangle$ and a set $Y \subseteq A^{n + 1}$ that is finite over $A^n$, we say that a box $B \subseteq A^n$ is \emph{$Y$-good} if for each $\left(x, y\right) \in Y \cap \left(B \times A\right)$, there exists an interval $I \subseteq A$ containing $y$ such that
	\[Y \cap \left(B \times I\right) = \Gamma\mleft(f\mright)\]
	is the graph of some continuous functions $f \in C\mleft(B\mright)$. We say that a point $x \in A^n$ is \emph{$Y$-good} if it is contained in some $Y$-good box.
\end{definition}

Directly using the definitions, van den Dries quickly proved the following result, numbered claim 2 in section 2.13 of the chapter:

\begin{lemma}
	\label{lem:y-good-decomposition}
	Given a structure $\mathfrak{A} = \left\langle A; <, \ldots \right\rangle$. Consider a definable subset $Y \subseteq A^{n + 1}$ that is finite over $A^n$. If a definably connected subset $U \subseteq A^n$ consists only  of $Y$-good points, then we can decompose
	\[Y \cap \left(U \times A\right) = \Gamma\mleft(f_1\mright) \cup \cdots \cup \Gamma\mleft(f_k\mright)\]
	into graphs of definable continuous functions $f_1 < \cdots < f_k \in C\mleft(U\mright)$.
\end{lemma}

More importantly, van den Dries also proved the next, more complicated lemma, numbered claim 3 in the same section:

\begin{lemma}
	\label{lem:y-good-existence}
	Given an o-minimal structure $\mathfrak{A} = \left\langle A; <, \ldots \right\rangle$, suppose that \fullref{thm:cell-decomposition} works on $A^k$ for any $k \leq n$ and a definable subset $Y \subseteq A^{n + 1}$ is finite over $A^n$, then any box in $A^n$ contains a $Y$-good point.
\end{lemma}

The $1$-dimensional case of this lemma is essentially proposition 1.8 in the same chapter, derived from the monotonicity theorem (Theorem 3.1.2 in van den Dries' book):

\begin{theorem}[Monotonicity theorem]
	\label{thm:monotonicity-theorem}
	Given an o-minimal structure $\mathfrak{A} = \left\langle A; <, \ldots \right\rangle$, let $f : \left(a, b\right) \rightarrow A$ be a definable function on the interval $\left(a, b\right) \subseteq A$. Then there are points
	\[a = a_0 < a_1 < \cdots < a_k < a_{k + 1} = b,\]
	such that for each $0 \leq i \leq k$, the restriction $\left.f\right|_{\left(a_i, a_{i + 1}\right)}$ is either constant or strictly monotone and continuous.
\end{theorem}

The higher-dimensional cases, instead, are derived from the following result on the decomposition of functions. This itself is established via another induction by looking at parts of the function that are continuous in all components (and monotone in one of them):

\begin{lemma}
	Under the same assumptions as \autoref{lem:y-good-existence}, for a definable function $f : U \rightarrow A$ on some definable set $U \subseteq A^n$, there exists a cell decomposition $S$ of $A^n$ that partitions $U$, such that for each cell $V \in S$ contained in $U$, the restriction $\left.f\right|_V$ is continuous.
\end{lemma}

Now, the concept of $Y$-goodness is expressible in first-order logic, so for any definable $Y \subseteq A^{n + 1}$ we have a definable set
\[R = \left\{x \in A^n : x \text{ is $Y$-good}\right\}.\]
Considering the open cells in some cell decomposition that partitions $R$, then \autoref{lem:y-good-decomposition} and \autoref{lem:y-good-existence} together imply the following:

\begin{corollary}
	\label{cor:finite-cell-decomposition}
	Under the same assumptions as \autoref{lem:y-good-existence}, if a definable subset $Y \subseteq A^{n + 1}$ is finite over $A^n$, then there exists a cell decomposition $S$ of $A^n$ such that for each cell $U \in S$, we can decompose
	\[Y \cap \left(U \times A\right) = \Gamma\mleft(f_1\mright) \cup \cdots \cup \Gamma\mleft(f_k\mright)\]
	into graphs of continuous functions $f_1 < \cdots < f_k \in C\mleft(U\mright)$.
\end{corollary}

Finally, for arbitrary subsets $U_1, \ldots, U_k \subseteq A^n$, we define
\[\boundary_n\mleft(U_i\mright) = \left\{\left(x, y\right) : x \in A^{n - 1}, y \in \boundary\mleft(\left(U_i\right)_x\mright)\right\}\]
where $\boundary\mleft(\left(U_i\right)_x\mright)$ is the boundary of the fibre $\left(U_i\right)_x \subseteq A$ as defined in \autoref{prop:finite-boundary}. Obviously, each $\boundary_n\mleft(U_i\mright)$ is finite over $A^{n - 1}$, and \autoref{cor:finite-cell-decomposition} applies that each of them decomposes into graphs of continuous functions. This can be used as a basis for constructing the desired cell decomposition of $A^n$, concluding the proof for the inductive case of \fullref{thm:cell-decomposition}.

We shall end this subsection by looking at two important corollaries of \autoref{thm:cell-decomposition}:

\begin{corollary}[Uniform finiteness property]
	\label{cor:uniform-finiteness}
	Given an o-minimal structure $\mathfrak{A} = \left\langle A; <, \ldots \right\rangle$ and a definable subset $Y \subseteq A^{n + 1}$ that is finite over $A^n$, $Y$ must be uniformly finite, i.e.\ there exists integer $n \in \mathbb{N}$ such that for all $x \in A^n$, $\left|Y_x\right| \leq n$.
\end{corollary}

\begin{proof}
	Since we have proven \autoref{thm:cell-decomposition}, we can now conclude that \autoref{cor:finite-cell-decomposition} holds for any dimension $n$. Let $S$ be such a cell decomposition of $A^n$, then for each cell $U \in S$ there exists $k_U \in \mathbb{N}$ such that
	\[Y \cap \left(U \times A\right) = \Gamma\mleft(f_1\mright) \cup \cdots \cup \Gamma\mleft(f_{k_U}\mright).\]
	Obviously, for any $x \in A^n$,
	\[\left|Y_x\right| \leq \max_{U \in S} k_U. \qedhere\]
\end{proof}

\begin{corollary}
	\label{cor:function-continuity-decomposition}
	Given an o-minimal structure $\mathfrak{A} = \left\langle A; <, \ldots \right\rangle$ and a definable function $f : U \rightarrow A^m$ on some definable set $U \subseteq A^n$, there exists a cell decomposition $S$ of $A^n$ that partitions $U$, such that for each cell $V \in S$ contained in $U$, the restriction $\left.f\right|_V$ is continuous.
\end{corollary}

\begin{proof}
	By \autoref{thm:cell-decomposition} we can find a cell decomposition $S$ that partitions the graph $\Gamma\mleft(f\mright) \subseteq A^{n + m}$. Let $\pi_k : A^{n + m} \rightarrow A^k$ denote the projection onto the first $k$ coordinates. Obviously,
	\[\pi_n\mleft(S\mright) = \left\{\pi_n\mleft(V\mright) : V \in S\right\}\]
	is a cell decomposition of $A^n$, and for each $V' \in \pi_n\mleft(S\mright)$, we can see via easy induction that $\Gamma\mleft(\left.f\right|_{V'}\mright)$ is an $\left(i_1, \ldots, i_n, 0, \ldots, 0\right)$-cell for some $i_1, \ldots, i_n \in \left\{0, 1\right\}$. In other words, $\left.f\right|_{V'}$ is a continuous function, and $\pi_n\mleft(S\mright)$ is the decomposition we need.
\end{proof}

\subsection{Dimension}

One of the immediate consequences of \fullref{thm:cell-decomposition} is that we can assign a dimension to every definable set:

\begin{definition}
	Given an o-minimal structure $\mathfrak{A} = \left\langle A; <, \ldots \right\rangle$, the \emph{dimension} of a definable set $U \subseteq A^n$ is given by
	\[\dim U = \max\mleft\{i_1 + \cdots + i_n : V \text{ is an $\left(i_1, \ldots, i_n\right)$-cell}, V \subseteq U\mright\}.\]

	We additionally specify that $\dim \varnothing = -\infty$.
\end{definition}

The use of cells here may seem ad-hoc, but van den Dries demonstrated in chapter 4 of \cite{tame-topology} that it has the desirable properties:

\begin{proposition}
	\label{prop:cell-dimension}
	For an $\left(i_1, \ldots, i_n\right)$-cell $U \subseteq A^n$, $\dim U = i_1 + \cdots + i_n$.
\end{proposition}

\begin{proof}
	Since $U \subseteq U$ trivially, we directly have $\dim U \geq i_1 + \cdots + i_n$.

	For the other direction, suppose that $V \subseteq U$ is a $\left(j_1, \ldots, j_n\right)$-cell. Let $\pi_k : A^n \rightarrow A^k$ denote the projection onto the first $k$ coordinates. For each $1 \leq k \leq n$, $\pi_k\mleft(V\mright) \subseteq \pi_k\mleft(U\mright)$ are $\left(j_1, \ldots, j_k\right)$- and $\left(i_1, \ldots, i_k\right)$-cells respectively. It is then obvious that $i_k = 0$ implies $j_k = 0$ because in this case, the fibres $\left(\pi_k\mleft(V\mright)\right)_x \subseteq \left(\pi_k\mleft(U\mright)\right)_x$ must be singletons or $\varnothing$ for each $x \in A^{k - 1}$.

	In other words, each $j_k \leq i_k$, i.e.\ $j_1 + \cdots + j_n \leq i_1 + \cdots + i_n$. We can thus conclude that $\dim U = i_1 + \cdots + i_n$.
\end{proof}

Particularly, the natural homeomorphism of a cell $U \subseteq A^n$ can now be written as $\pi_{\left(U\right)} : A^n \rightarrow A^{\dim U}$.

\begin{proposition}
	\label{prop:dimension-from-decomposition}
	If we write a non-empty definable set $U \subseteq A^n$ as a finite union of cells $U = \bigcup_j V_j$ where each $V_j$ is an $\left(i^{\left(j\right)}_1, \ldots, i^{\left(j\right)}_n\right)$-cell, then
	\[\dim U = \max_j \dim V_j = \max_j \left(i^{\left(j\right)}_1 + \cdots + i^{\left(j\right)}_n\right).\]
\end{proposition}

\begin{proof}
	Since each $V_j \subseteq U$ trivially, we directly have
	\[\dim U \geq \max_j \dim V_j.\]

	For the other direction, consider an $\left(\ell_1, \ldots, \ell_n\right)$-cell $W \subseteq U$. We find by \fullref{thm:cell-decomposition} a decomposition $S$ of $A^n$ that partitions $W$ and each $V_j$. We can write $W = \bigcup S'$ for some finite $S' \subseteq S$.

	Consider the natural homeomorphism $\pi_{\left(W\right)} : A^n \rightarrow A^{\dim W}$. For each $W' \in S'$, we have $W' \subseteq W$, so $\pi_{\left(W\right)}\mleft(W'\mright)$ must also be a cell in $A^{\dim W}$ with
	\[\pi_{\left(W\right)}\mleft(W\mright) = \bigcup_{W' \in S'} \pi_{\left(W\right)}\mleft(W'\mright).\]

	By \autoref{prop:open-cell-condition} and \autoref{cor:open-set-decomposition}, $S'$ must contain some $\left(\ell'_1, \ldots, \ell'_n\right)$-cell $W_0$ such that $\pi_{\left(W\right)}\mleft(W_0\mright)$ is an open cell, i.e.\ a $\left(1, \ldots, 1\right)$-cell in $A^{\dim W}$. $W_0$ must exactly be an $\left(\ell_1, \ldots, \ell_n\right)$-cell, and $W_0 \subseteq V_j$ for some $j$, i.e. $\ell_1 + \cdots + \ell_n \leq \dim V_j$. It follows that $\dim U \leq \max_j \dim V_j$.

	Combining both directions and using \autoref{prop:cell-dimension}, we have
	\[\dim U = \max_j \dim V_j = \max_j \left(i^{\left(j\right)}_1 + \cdots + i^{\left(j\right)}_n\right). \qedhere\]
\end{proof}

This is then an immediate corollary:

\begin{corollary}
	\label{cor:union-dimension}
	For definable sets $U, V \subseteq A^n$,
	\[\dim\mleft(U \cup V\mright) = \max\mleft\{\dim U, \dim V\mright\}.\]
\end{corollary}

Also, in preparation for the next subsection, we prove the following proposition:

\begin{proposition}
	\label{prop:frontier-dimension}
	Let $U \subseteq A^n$ be a non-empty definable set. Then
	\[\dim \frontier\mleft(U\mright) < \dim U,\]
	where the frontier $\frontier\mleft(U\mright) = \closure\mleft(U\mright) \setminus U$.
\end{proposition}

To prove this, we first cite some technical results\footnote{The proofs for these results are rather lengthy and hence omitted here. They can be found in chapter 4 of van den Dries' book \cite{tame-topology}: \autoref{fact:bijections-preserve-dimension} as Proposition 1.3(i), \autoref{fact:dimension-from-projection} as Corollary 4.1.6(i), and \autoref{fact:finite-closure-fiber-closure-disagreement} as Lemma 4.1.7.} from \cite{tame-topology}:

\begin{fact}
	\label{fact:bijections-preserve-dimension}
	Suppose that $U \subseteq A^n$, $V \subseteq A^m$ are definable sets with $f : U \rightarrow V$ being a definable bijection. Then $\dim U = \dim V$.
\end{fact}

\begin{fact}
	\label{fact:dimension-from-projection}
	Let $U \in A^{n + m}$ be a definable set. For each $d \in \left\{-\infty, 0, \ldots, m\right\}$, let
	\[U_{\left[d\right]} = \left\{x \in A^n : \dim U_x = d\right\}.\]
	Then each $U_{\left[d\right]} \subseteq A^n$ is definable, with
	\[\dim U = \max_d \left(\dim U_{\left[d\right]} + d\right).\]
\end{fact}

\begin{fact}
	\label{fact:finite-closure-fiber-closure-disagreement}
	Let $U \subseteq A^{n + 1}$ be a non-empty definable set. Then the set
	\[U' = \left\{x \in A : \closure\mleft(U_x\mright) \neq \left(\closure\mleft(U\mright)\right)_x\right\}\]
	is finite.
\end{fact}

\begin{proof}[Proof of \autoref{prop:frontier-dimension}]
	We prove this by induction: the base case where $n = 1$ is trivial, because for $U \subseteq A$, $\frontier\mleft(U\mright) \subseteq \boundary\mleft(U\mright)$ is finite by \autoref{prop:finite-boundary}.

	For the inductive case, first suppose that $\dim U = 0$, i.e.\ $U$ is finite. Then $\frontier\mleft(U\mright) = \varnothing$ trivially, so the proposition holds. When $\dim U > 0$, we consider definable bijections $\varphi_i : \left(x_1, \ldots, x_n\right) \mapsto \left(x_i, x_1, \ldots, x_{i - 1}, x_{i + 1}, \ldots, x_n\right)$ for any $1 \leq i \leq n$. For each $\varphi_i$, let
	\[P_i = \left\{x \in A : \closure\mleft(\left(\varphi_i\mleft(U\mright)\right)_x\mright) \neq \left(\closure\mleft(\varphi_i\mleft(U\mright)\mright)\right)_x\right\} \subseteq A,\]
	which is finite by \autoref{fact:finite-closure-fiber-closure-disagreement} and write $P_i^c = A \setminus P_i$. Then,
	\[\closure\mleft(\varphi_i\mleft(U\mright)\mright) \cap \left(P_i^c \times A^{n - 1}\right) = \bigcup_{x \in P_i^c} \left(\left\{x\right\} \times \closure\mleft(\left(\varphi_i\mleft(U\mright)\right)_x\mright)\right).\]

	Now, by inductive hypothesis, $\dim \frontier\mleft(\left(\varphi_i\mleft(U\mright)\right)_x\mright) < \dim \left(\varphi_i\mleft(U\mright)\right)_x$. For each $d_1 < d_2 \in \left\{-\infty, 0, \ldots, n - 1\right\}$, let
	\[\left(\varphi_i\mleft(U\mright)\right)_{\left[d_1, d_2\right]} = \left\{x \in P_i^c : \dim \frontier\mleft(\left(\varphi_i\mleft(U\mright)\right)_x\mright) = d_1, \dim \left(\varphi_i\mleft(U\mright)\right)_x = d_2\right\}.\]
	By \autoref{fact:dimension-from-projection} we can compute that
	\begin{align*}
		\dim \varphi_i\mleft(\frontier\mleft(U\mright) \cap \left(A^{i - 1} \times P_i^c \times A^{n - i}\right)\mright) & = \dim\mleft(\frontier\mleft(\varphi_i\mleft(U\mright)\mright) \cap \left(P_i^c \times A^{n - 1}\right)\mright)                    \\
		                                                                                                                 & = \dim \bigcup_{x \in P_i^c} \left(\left\{x\right\} \times \frontier\mleft(\left(\varphi_i\mleft(U\mright)\right)_x\mright)\right) \\
		                                                                                                                 & = \max_{d_1, d_2} \left(\dim \left(\varphi_i\mleft(U\mright)\right)_{\left[d_1, d_2\right]} + d_1\right)                           \\
		                                                                                                                 & < \max_{d_1, d_2} \left(\dim \left(\varphi_i\mleft(U\mright)\right)_{\left[d_1, d_2\right]} + d_2\right)                           \\
		                                                                                                                 & = \dim\mleft(\varphi_i\mleft(U\mright) \cap \left(P_i^c \times A^{n - 1}\right)\mright)                                            \\
		                                                                                                                 & = \dim \varphi_i\mleft(U \cap \left(A^{i - 1} \times P_i^c \times A^{n - i}\right)\mright).
	\end{align*}
	By \autoref{fact:bijections-preserve-dimension}, definable bijections preserve dimension, so indeed
	\[\dim\mleft(\frontier\mleft(U\mright) \cap \left(A^{i - 1} \times P_i^c \times A^{n - i}\right)\mright) < \dim\mleft(U \cap \left(A^{i - 1} \times P_i^c \times A^{n - i}\right)\mright).\]

	Finally, notice that
	\[A^n = \left(P_1 \times \cdots \times P_n\right) \cup \bigcup_i \left(A^{i - 1} \times P_i^c \times A^{n - i}\right)\]
	where $P_1 \times \cdots \times P_n$ is finite. By \autoref{cor:union-dimension},
	\begin{align*}
		\dim \frontier\mleft(U\mright) & = \max\mleft\{\begin{aligned}
			                                                & \dim\mleft(\frontier\mleft(U\mright) \cap \left(P_1 \times \cdots \times P_n\right)\mright);                                             \\
			                                                & \dim\mleft(\frontier\mleft(U\mright) \cap \left(A^{i - 1} \times P_i^c \times A^{n - i}\right)\mright) \text{ for each } 1 \leq i \leq n
		                                               \end{aligned}\mright\} \\
		                               & < \max\mleft\{\dim\mleft(U \cap \left(A^{i - 1} \times P_i^c \times A^{n - i}\right)\mright) : 1 \leq i \leq n\mright\}                        \\
		                               & = \dim U. \qedhere
	\end{align*}
\end{proof}

\subsection{The stratification theorem}

The analysis of dimensions allows us to establish o-minimal structures as a candidate for Grothendieck's tame topology in the sense of topological stratifications:

\begin{definition}
	Given a structure $\mathfrak{A} = \left\langle A; <, \ldots \right\rangle$ with the interval topology, we say that some subset $U \subseteq A^n$ is a (topological) \emph{$k$-manifold}\footnote{There is no need to enforce additionally that $U$ is Hausdorff, because $A^n$ is trivially Hausdorff under the interval topology, given that the ordering $<$ is dense.} if, in the sub-topology on $U$, every element $x \in U$ lies in a neighbourhood $V \subseteq U$ that is homeomorphic to a box $B \subseteq A^k$.
\end{definition}

The condition of bing a $k$-manifold is preserved under homeomorphisms, or in particular, the natural homeomorphism $\pi_{\left(U\right)}$ for a cell $U$. We thus have:

\begin{proposition}
	\label{prop:cell-manifold}
	Each cell $U \subseteq A^n$ is a $\left(\dim U\right)$-manifold.
\end{proposition}

\begin{definition}[Stratification]
	We say that a closed set in a topological space, $U \subseteq T$, has a \emph{stratification} if it has a filtration, i.e.\ an increasing chain of subsets
	\[\varnothing = X_{-1} \subseteq X_0 \subseteq X_1 \subseteq \cdots \subseteq X_k = U,\]
	such that for each $0 \leq i \leq k$, $X_i \setminus X_{i - 1}$ is a $i$-manifold with frontier $\frontier\mleft(X_i \setminus X_{i - 1}\mright) \subseteq X_{i - 1}$.
\end{definition}

\begin{theorem}[Stratification theorem]
	\label{thm:stratification}
	Given an o-minimal structure $\mathfrak{A} = \left\langle A; <, \ldots \right\rangle$, any definable closed set $U \subseteq A^n$ has a stratification.
\end{theorem}

\begin{proof}
	We prove by induction on the dimension of $U$:

	For the base case where $U = \varnothing$ or $\dim U = 0$, i.e.\ $U$ is finite, the statement is trivial.

	For the inductive case, by \fullref{thm:cell-decomposition} we have a cell decomposition $S$ of $A^n$ that partitions $U$ and by \autoref{prop:dimension-from-decomposition} there exists a cell $V \subseteq U$ in $S$ such that $\dim V = \dim U$.

	For each such cell $V$, we write $\interior_U\mleft(V\mright)$ for its interior in the sub-topology on $U$. Now let $S'$ be a refinement of $S$, i.e.\ a cell decomposition of $A^n$ that partitions any cell in $S$ as well as $\interior_U\mleft(V\mright)$ for any cell $V \in S$, $V \subseteq U$ such that $\dim V = \dim U$. For any cell $V' \in S'$, $V' \subseteq U$ such that $\dim V' = \dim U$, we know that $V' \subseteq V \subseteq U$ for some cell $V \in U$ and
	\[\dim V' = \dim V = \dim U.\]

	Additionally, $V \setminus \interior_U\mleft(V\mright) \subseteq \frontier\mleft(\interior_U\mleft(V\mright)\mright)$, so by \autoref{prop:frontier-dimension}
	\[\dim\mleft(V \setminus \interior_U\mleft(V\mright)\mright) \leq \dim \frontier\mleft(\interior_U\mleft(V\mright)\mright) < \dim \interior_U\mleft(V\mright) \leq \dim V = \dim V',\]
	i.e.\ $V' \not\subseteq V \setminus \interior_U\mleft(V\mright)$. We must have $V' \subseteq \interior_U\mleft(V\mright)$.

	Finally, consider the natural homeomorphism $\pi_{\left(V\right)}$. Since $\dim V' = \dim V$, $\pi_{\left(V\right)}\mleft(V'\mright) \subseteq \pi_{\left(V\right)}\mleft(V\mright)$ are obviously both open cells in $A^{\dim V}$, so $V'$ is open in $V$ and thus open in $U$.

	Now, we can set $X_{\dim U} = U$ and
	\[X_{\dim U - 1} = \bigcup \left\{V' \in S' : V' \subseteq U, \dim V' < \dim U\right\}.\]
	Then $X_{\dim U} \setminus X_{\dim U - 1}$ is a finite union of cells $V' \in S$, $V' \subseteq U$ such that $V'$ is open in $U$ and $\dim V' = \dim U$. By \autoref{prop:cell-manifold}, each such $V'$ is a $\left(\dim U\right)$-manifold. Since the interval topology is naturally Hausdorff, the finite union of $\left(\dim U\right)$-manifolds open in $U$ is again a $\left(\dim U\right)$-manifold. This implies $X_{\dim U - 1}$ is closed, hence applying the inductive hypothesis yields the remainder of the filtration:
	\[\varnothing = X_{-1} \subseteq X_0 \subseteq \cdots \subseteq X_{\dim U - 1}. \qedhere\]
\end{proof}

\section{Definable complex analysis}
\label{sec:definable-complex-analysis}

\subsection{Real closed fields}

One of the most important example of an o-minimal structure is the ordered field of reals $\left\langle \mathbb{R}; <, +, -, \times, 0, 1 \right\rangle$. In the second half of this essay, we shall apply the properties of o-minimality to complex analysis by viewing $\mathbb{C} \cong \mathbb{R}^2$ as a real vector space.

Before we start, do note that, due to \autoref{cor:uniform-finiteness}:

\begin{proposition}
	O-minimality is preserved under elementary equivalence.
\end{proposition}

This is proven as Theorem 0.2 in \cite{o-minimality-elementary-equivalence}. As a result, we can work with a complete first-order axiomatisation of $\mathbb{R}$ instead, and our results will also apply to non-standard real and complex fields. Namely, we consider an arbitrary real closed field:

\begin{definition}[Real closed fields]
	\label{def:real-closed-fields}
	We say that a field $F$ is \emph{formally real} if $-1$ is not a sum of squares in $F$. We say that a formally real field is \emph{real closed} if it has no proper algebraic extensions that are also formally real.

	We additionally say that a field extension $L \mid F$ is a \emph{real closure} if $L$ is algebraic over $F$ and real closed. As proven in chapter XI of Serge Lang's \textit{Algebra} \cite{algebra-lang}, the real closure of an ordered formally real field is unique up to order-preserving isomorphisms.
\end{definition}

Of course, the widely-used conditions in \autoref{def:real-closed-fields} are not readily axiomatisable in first-order logic. We consider an equivalent definition:

\begin{definition}[Ordered fields]
	We say that $\left\langle F; <, +, -, \times, 0, 1 \right\rangle$ is an \emph{ordered field} if $\left\langle F; +, -, \times, 0, 1 \right\rangle$ is a field, and $<$ is a total ordering on $F$ satisfying
	\begin{packeditemize}
		\item for any $a, b, c \in F$ such that $a < b$, we have $a + c < b + c$;
		\item for any $a, b \in F$ such that $0 < a$, $0 < b$, we have $0 < ab$.
	\end{packeditemize}
\end{definition}

\begin{definition}
	The first-order theory of (ordered) real closed fields, denoted $\RCF$, is axiomatised such that $\left\langle F; <, +, -, \times, 0, 1 \right\rangle \vDash \RCF$ if and only if
	\begin{packeditemize}
		\item $\left\langle F; <, +, -, \times, 0, 1 \right\rangle$ is an ordered field;
		\item for any $a \in F$ such that $a > 0$, there exists $b \in F$ such that $b^2 = a$;
		\item any polynomial of odd degree in $F\mleft[X\mright]$ has a root in $F$.
	\end{packeditemize}
\end{definition}

For some $\left\langle F; <, +, -, \times, 0, 1 \right\rangle \vDash \RCF$, it is easy to observe that every element in the field extension $F\mleft(\sqrt{-1}\mright)$ has a square root: namely, for any element $a + b\sqrt{-1} \in F\mleft(\sqrt{-1}\mright)$, we have
\[\left(\sqrt{\frac{a + \sqrt{a^2 + b^2}}{2}} + \sqrt{\frac{-a + \sqrt{a^2 + b^2}}{2}} \cdot \sqrt{-1}\right)^2 = a + b\sqrt{-1}.\]
With fundamentally the same proof via Galois theory of $\mathbb{C}$ being algebraically closed, this means that $F\mleft(\sqrt{-1}\mright)$ is algebraically closed. It follows that our first-order axiomatisation $\RCF$ is equivalent to the conditions for real closed fields:

\begin{proposition}
	$\left\langle F; <, +, -, \times, 0, 1 \right\rangle \vDash \RCF$\footnote{Here the ordering $<$ is naturally given by the property that $a < b$ if and only if $b - a = c^2$ for some $c \in F \setminus \left\{0\right\}$.} if and only if $F$ is a real closed field under $+$ and $\times$.
\end{proposition}

\begin{proof}
	The backward direction is trivial. For the forward direction, the axioms of an ordered field immediately implies that squares are non-negative, so $-1 < 0$ cannot be a sum of squares, i.e.\ $F$ is formally real.

	Now, $F\mleft(\sqrt{-1}\mright)$ is algebraically closed, so any algebraic extension $L \geq F$ must satisfy
	\[F \leq L \leq F\mleft(\sqrt{-1}\mright).\]
	Here $F\mleft(\sqrt{-1}\mright)$ is an extension of degree $2$, which is not formally real. Thus the only algebraic extension of $F$ that is formally real is $F$ itself. $F$ is real closed.
\end{proof}

We now work towards proving that $\left\langle F; <, +, -, \times, 0, 1 \right\rangle$ is an o-minimal structure:

\begin{lemma}[Intermediate value theorem]
	\label{lem:intermediate-value-theorem}
	Given some real closed field $F$ and a polynomial $f\mleft(X\mright) \in F\mleft[X\mright]$ such that $f\mleft(a\mright) > 0$, $f\mleft(b\mright) < 0$ for $a, b \in F$. Then there exists $c$ between $a$ and $b$ such that $f\mleft(c\mright) = 0$.
\end{lemma}

\begin{proof}
	$F\mleft(\sqrt{-1}\mright)$ is algebraically closed, so $f$ must be factorisable into linear and quadratic factors. Consider an irreducible quadratic factor
	\[X^2 + pX + q = \left(X + \frac{p}{2}\right)^2 + q - \frac{p^2}{4}.\]
	It cannot have a root in $F$, so we must have $\left(X + p / 2\right)^2 \geq 0$, $q - p^2 / 4 > 0$, i.e.\ the sign of a quadratic factor cannot change. Therefore, there must exist a linear factor $\left(X - c\right)$ of $f$ such that $\left(a - c\right)$, $\left(b - c\right)$ are of different signs. Then $c$ is a root of $f$ that lies between $a$ and $b$.
\end{proof}

\begin{lemma}
	\label{lem:quantifer-free-intervals-and-singletons}
	Consider a real closed field $\mathfrak{F} = \left\langle F; <, +, -, \times, 0, 1 \right\rangle$ and a real closed subfield $K \subseteq F$. Let $L = \left\{<, +, -, \times, 0, 1\right\}$ denote the first-order language for real closed fields and $L_K$ its expansion with constants from $K$. For any quantifier-free sentence $\varphi\mleft(x\mright) \in \Formula\mleft(L_K\mright)$ with only one variable, we can write
	\[\varphi^{\mathfrak{F}} = \left(\bigcup_{i = 1}^k \left(p_i, q_i\right)\right) \cup \left(\bigcup_{j = 1}^\ell \left\{r_j\right\}\right) \subseteq \Def_1\mleft(\mathfrak{F}_K\mright)\]
	as a finite union of intervals and singletons, where $p_i, q_i, r_j \in K$.
\end{lemma}

\begin{proof}
	The propositional connectives $\neg$, $\land$, $\lor$, $\rightarrow$, $\leftrightarrow$ trivially corresponds to set operations like complements, intersections and unions. Therefore, it suffices that we prove the lemma for atomic formulae, i.e.\ $f\mleft(X\mright) = g\mleft(X\mright)$ or $f\mleft(X\mright) < g\mleft(X\mright)$ where $f, g \in K\mleft[X\mright]$ are polynomials.

	By the fundamental theorem of algebra, $f - g$ has finitely many roots in $F$. We can enumerate them as
	\[a_1 < a_2 < \ldots < a_k.\]
	Since $K$ is real closed, we must have $a_1, \ldots, a_k \in K$. Let $a_0 = -\infty$, $a_{k + 1} = \infty$, then for any $0 \leq i \leq k$, the sign of $f - g$ does not change on the interval $\left(a_i, a_{i + 1}\right)$, by \fullref{lem:intermediate-value-theorem}. In other words, $f = g$ and $f < g$ both correspond to unions of finitely many sets among the following:
	\[\left(-\infty, a_1\right), \left\{a_1\right\}, \left(a_1, a_2\right), \ldots, \left\{a_{k - 1}\right\}, \left(a_{k - 1}, a_k\right), \left\{a_k\right\}, \left(a_k, \infty\right). \qedhere\]
\end{proof}

\begin{proposition}
	\label{prop:rcf-quantifier-elimination}
	$\RCF$ admits quantifier elimination.
\end{proposition}

A regular approach to this proposition, for example taken in van den Dries' book \cite{tame-topology}, first establishes cell decomposition specifically for semialgebraic sets and deduces the Tarski-Seidenberg theorem:

\begin{theorem*}[Tarski-Seidenberg]
	Given a real closed field $F$, we say that a set $U \subseteq F^n$ is semialgebraic if it is a finite union of sets of the form
	\[\left\{x \in F^n : p_1\mleft(x\mright) = \cdots = p_k\mleft(x\mright) = 0, q_1\mleft(x\mright) > 0, \ldots, q_\ell\mleft(x\mright) > 0\right\}.\]

	Let $\pi : F^{n + 1} \rightarrow F^n$ be the projection onto the first $n$ coordinates, then for any semialgebraic set $U \subseteq F^{n + 1}$, $\pi\mleft(U\mright)$ is semialgebraic.
\end{theorem*}

However, we shall mention here a simpler model-theoretical proof by David Marker, presented in chapter 1 of \cite{field-model-theory}. Marker first established the following test for quantifier elimination as Theorem 1.4 in his book:

\begin{fact}
	\label{fact:quantifier-elimination-from-substructure}
	Let $T$ be a theory in a first-order language $L$. Given some formula $\varphi\mleft(x_1, \ldots, x_n\mright) \in \Formula\mleft(L\mright)$, if for any models $\mathfrak{A}, \mathfrak{B} \vDash T$ such that $\mathfrak{C} \leq \mathfrak{A}$, $\mathfrak{C} \leq \mathfrak{B}$ is a shared substructure and any $a_1, \ldots, a_n \in \mathfrak{C}$,
	\[\mathfrak{A} \vDash \varphi\mleft(a_1, \ldots, a_n\mright) \quad \text{if and only if} \quad \mathfrak{B} \vDash \varphi\mleft(a_1, \ldots, a_n\mright),\]
	then $T \vDash \forall x_1 \cdots \forall x_n \left(\varphi \leftrightarrow \psi\right)$ for some quantifier-free $\psi\mleft(x_1, \ldots, x_n\mright)$.
\end{fact}

\begin{proof}[Proof of \autoref{prop:rcf-quantifier-elimination}]
	By induction, it obviously suffices to prove that for any quantifier-free formula $\varphi\mleft(x, v_1, \ldots, v_n\mright)$, there exists quantifier-free $\psi\mleft(v_1, \ldots, v_n\mright)$ such that
	\[\RCF \vDash \forall v_1 \cdots \forall v_n \left(\exists x \ \varphi \leftrightarrow \psi\right).\]

	For any such $\varphi$, we invoke \autoref{fact:quantifier-elimination-from-substructure}: consider $\mathfrak{A}, \mathfrak{B} \vDash \RCF$ and some shared substructure $\mathfrak{C}$. Given our language $L = \left\{<, +, -, \times, 0, 1\right\}$, $\mathfrak{C}$ must be an ordered integral domain. Let $\mathfrak{F}$ be the real closure of the fraction field of $\mathfrak{C}$, then $\mathfrak{F} \leq \mathfrak{A}$ and $\mathfrak{F} \leq \mathfrak{B}$ due to its uniqueness.

	Now, suppose that $\mathfrak{A} \vDash \exists x \ \varphi\mleft(x, a_1, \ldots, a_n\mright)$ for $a_1, \ldots, a_n \in \mathfrak{C}$. Since $\varphi$ is quantifier-free, by \autoref{lem:quantifer-free-intervals-and-singletons} $\varphi\mleft(x, a_1, \ldots, a_n\mright)^{\mathfrak{A}}$ is a finite union of intervals and singletons, where the singletons all lie in $\mathfrak{F}$ and the intervals have endpoints in $\mathfrak{F}$. Then, there must exist
	\[y \in \mathfrak{F} \cap \varphi\mleft(x, a_1, \ldots, a_n\mright)^{\mathfrak{A}} \subseteq \mathfrak{B},\]
	so $\mathfrak{B} \vDash \exists x \ \varphi\mleft(x, a_1, \ldots, a_n\mright)$ as well. By symmetry, the proof for the assumption in \autoref{fact:quantifier-elimination-from-substructure} finishes.
\end{proof}

An immediate consequence is that the theory $\RCF$ is model-complete. Additionally, for a model $\mathfrak{F} = \left\langle F; <, +, -, \times, 0, 1 \right\rangle \vDash \RCF$, the ordering ensures that $F$ is a field of characteristic $0$, i.e.\ the minimal substructure of any such $\mathfrak{F}$ is naturally $\mathbb{Z}$. By quantifier elimination, any first-order sentence is equivalent to a quantifier-free proposition about $\mathbb{Z}$, so $\RCF$ is trivially also a complete theory as we desire.

We can finally prove that:

\begin{theorem}
	\label{thm:rcf-o-minimal}
	Suppose that $\mathfrak{F} = \left\langle F; <, +, -, \times, 0, 1 \right\rangle \vDash \RCF$, then $\mathfrak{F}$ is an o-minimal structure.
\end{theorem}

\begin{proof}
	Given the language $L = \left\{<, +, -, \times, 0, 1\right\}$, consider any $\varphi\mleft(x\mright) \in \Formula\mleft(L_F\mright)$ with one free variable. We can view it as $\varphi\mleft(x; a_1, \ldots, a_n\mright)$ with constant parameters $a_1, \ldots, a_n \in F$, where
	\[\varphi\mleft(x; v_1, \ldots, v_n\mright) \in \Formula\mleft(L\mright).\]

	By \autoref{prop:rcf-quantifier-elimination}, we have quantifier-free $\psi\mleft(x; v_1, \ldots, v_n\mright) \in \Formula\mleft(L\mright)$ such that
	\[\mathfrak{F} \vDash \forall x \forall v_1 \cdots \forall v_n \left(\varphi\mleft(x; v_1, \ldots, v_n\mright) \leftrightarrow \psi\mleft(x; v_1, \ldots, v_n\mright) \right),\]
	i.e.\ $\varphi\mleft(x; a_1, \ldots, a_n\mright)^{\mathfrak{F}} = \psi\mleft(x; a_1, \ldots, a_n\mright)^{\mathfrak{F}}$. However, here $\psi\mleft(x; a_1, \ldots, a_n\mright)$ is a quantifier-free formula in $\Formula\mleft(L_F\mright)$, and we have shown in \autoref{lem:quantifer-free-intervals-and-singletons} that $\psi\mleft(x; a_1, \ldots, a_n\mright)^{\mathfrak{F}}$ is a finite union of intervals and singletons. This proves that $\mathfrak{F}$ is o-minimal.
\end{proof}

\subsection{Paths and winding numbers}

From this section onwards, we shall use $\mathfrak{R} = \left\langle R; <, +, -, \times, 0, 1, \ldots \right\rangle$ to denote an arbitrary o-minimal structure\footnote{Here we allow $\mathfrak{R}$ to denote not just the real closed field $\left\langle R; <, +, -, \times, 0, 1 \right\rangle$, but also any o-minimal extensions of it (in an expanded language).} where $R$ is a real closed field, i.e.\ such that $\mathfrak{R} \vDash \RCF$. We use $K$ to denote the ``complex'' extension of $\mathfrak{R}$ that contains a square root of $-1$. We write $i = \sqrt{-1}$, then $K = R\mleft(i\mright)$ is algebraically closed.

We write every element in $K$ as $z = x + iy$ for $x, y \in R$. Just as in classical analysis, we say that $x = \Real\mleft(z\mright)$ is the real part and $y = \Imag\mleft(z\mright)$ is the imaginary part. We also define as usual the complex conjugate $\overline{z} = x - iy$ and the modulus of $z$:
\[\left|z\right| = z \overline{z} = \sqrt{x^2 + y^2} \in R.\]

However, we do not have the concept of argument $\arg\mleft(z\mright)$ in $K$ because we cannot access the exponential function in an arbitrary real closed field. In order to recover winding numbers on $K$, we consider the unit circle $S^1 = \left\{z \in K : \left|z\right| = 1\right\}$, which is a multiplicative subgroup of $K$. It is easy to verify that, viewed in $R^2$, we can decompose
\[S^1 = \left\{-1, 1\right\} \cup \Gamma\mleft(s_0\mright) \cup \Gamma\mleft(-s_0\mright),\]
where $s_0 : (-1, 1) \rightarrow R$ is a definable continuous function, for example given by $s_0\mleft(x\mright) = \sqrt{1 - x^2}$. We then have a definable continuous bijection $\sigma_0 : \left[0, 1\right) \rightarrow S^1$ given by
\[\sigma_0\mleft(t\mright) = \left\{\begin{aligned}
		 & 1                                  &  & \text{if } t = 0,         \\
		 & 1 - 4t + is_0\mleft(1 - 4t\mright) &  & \text{if } 0 < t < 1 / 2, \\
		 & {-1}                               &  & \text{if } t = 1 / 2,     \\
		 & 4t - 3 - is_0\mleft(4t - 3\mright) &  & \text{if } 1 / 2 < t < 1.
	\end{aligned}\right.\]

Now, we consider a covering of $S^1$, given by $\mathcal{H} = \mathbb{Z} \times S^1$ with a group operation
\[\left(m, x\right) + \left(n, y\right) = \left\{\begin{aligned}
		 & \left(m + n, xy\right)     &  & \text{if } \begin{aligned}[t]
			                                               & \sigma_0^{-1}\mleft(xy\mright) > \sigma_0^{-1}\mleft(x\mright), \\
			                                               & \sigma_0^{-1}\mleft(xy\mright) > \sigma_0^{-1}\mleft(y\mright),
		                                              \end{aligned} \\
		 & \left(m + n + 1, xy\right) &  & \text{otherwise.}
	\end{aligned}\right.\]
Notice that the lexicographic ordering
\[\left(m, x\right) < \left(n, y\right) \quad \text{if} \quad m < n \text{ or } \left(m = n \text{ and } \sigma_0^{-1}\mleft(x\mright) < \sigma_0^{-1}\mleft(y\mright)\right)\]
makes $\mathcal{H}$ a linearly ordered group, and the induced interval topology makes $\mathcal{H}$ a covering space of $S^1$ under the covering map $\pi : \left(n, x\right) \mapsto x$. Viewing $S^1$ as a multiplicative group, then $\pi$ is at the same time a group homomorphism.

\begin{definition}[Definable paths]
	We call a definable continuous function $\gamma: \left[0, 1\right] \rightarrow K$ a \emph{definable path}. We denote its image as
	\[\gamma^* = \left\{\gamma\mleft(t\mright) : t \in \left[0, 1\right]\right\}.\]

	We say that a definable path $\gamma$ is \emph{circular} if $\gamma\mleft(0\mright) = \gamma\mleft(1\mright)$.
\end{definition}

\begin{theorem}[Path lifting theorem]
	\label{thm:path-lifting}
	Any definable path $\gamma$ such that $\gamma^* \subseteq S^1$ lifts to a continuous function $\tilde{\gamma} : \left[0, 1\right] \rightarrow \mathcal{H}$, such that $\gamma = \pi \circ \tilde{\gamma}$.

	Additionally, $\tilde{\gamma}$ is definable if we view $\mathcal{H} = \mathbb{Z} \times S^1 \subseteq R^3$.
\end{theorem}

\begin{proof}
	We consider the definable set
	\[E = \left\{t \in \left(0, 1\right) : \gamma\mleft(t\mright) = 1\right\}.\]
	Since $\mathfrak{R}$ is o-minimal, by \autoref{prop:finite-boundary} we can enumerate $\boundary\mleft(E\mright) \cup \left\{0, 1\right\}$ as
	\[0 = a_0 < a_1 < \ldots < a_k < a_{k + 1} = 1.\]

	We first assign an integer to each interval and singleton in the decomposition above recursively with
	\[N : \left\{\left\{a_0\right\}, \left(a_0, a_1\right), \left\{a_1\right\}, \ldots, \left(a_k, a_{k + 1}\right), \left\{a_{k + 1}\right\}\right\} \rightarrow \mathbb{Z}\]
	defined as the following:

	Firstly, if $\gamma\mleft(0\mright) = 1$, then we set $N\mleft(\left\{a_0\right\}\mright) = 0$; otherwise we set $N\mleft(\left\{a_0\right\}\mright) = N\mleft(a_0, a_1\mright) = 0$.

	Now, for each next interval $\left(a_i, a_{i + 1}\right)$ where $\gamma\mleft(a_i\mright) = 1$, if $\gamma = 1$ constantly on $\left(a_i, a_{i + 1}\right)$, then we set $N\mleft(a_i, a_{i + 1}\mright) = N\mleft(\left\{a_i\right\}\mright)$. Otherwise, $\gamma \neq 1$ on $\left(a_i, a_{i + 1}\right)$. Since $\gamma\mleft(a_i\mright) = 1$, by continuity there exists a sub-interval $\left(a_i, \delta\right) \subseteq \left(a_i, a_{i + 1}\right)$ on which
	\[\left|\gamma\mleft(t\mright) - 1\right| < 2,\]
	i.e.\ $\gamma\mleft(t\mright) \neq -1$, and thus $\Real\mleft(\gamma\mleft(t\mright)\mright) \neq 0$. By \fullref{lem:intermediate-value-theorem}, we must then have either $\Real \circ \gamma > 0$ or $\Real \circ \gamma < 0$ on $\left(a_i, \delta\right)$. In the former case set $N\mleft(a_i, a_{i + 1}\mright) = N\mleft(\left\{a_i\right\}\mright)$, while in the latter case set $N\mleft(a_i, a_{i + 1}\mright) = N\mleft(\left\{a_i\right\}\mright) - 1$.

	For each next singleton $a_i$ such that $\gamma\mleft(a_i\mright) = 1$, if $\gamma = 1$ constantly on $\left(a_{i - 1}, a_i\right)$, then we set $N\mleft(\left\{a_i\right\}\mright) = N\mleft(a_{i - 1}, a_i\mright)$. Otherwise, $\gamma \neq 1$ on $\left(a_{i - 1}, a_i\right)$, and similarly we must have either $\Real \circ \gamma > 0$ or $\Real \circ \gamma < 0$ on some $\left(\delta, a_i\right) \subseteq \left(a_{i - 1}, a_i\right)$. In the former case set $N\mleft(\left\{a_i\right\}\mright) = N\mleft(a_{i - 1}, a_i\mright)$, while in the latter case set $N\mleft(\left\{a_i\right\}\mright) = N\mleft(a_{i - 1}, a_i\mright) + 1$.

	Finally, if $\gamma\mleft(1\mright) \neq 1$, then we simply set $N\mleft(\left\{a_{k + 1}\right\}\mright) = N\mleft(a_k, a_{k + 1}\mright)$.

	We then have definable function $\tilde{\gamma} : \left[0, 1\right] \rightarrow \mathcal{H}$ given by
	\[\tilde{\gamma}\mleft(t\mright) = \left\{\begin{aligned}
			 & \left(N\mleft(\left\{a_i\right\}\mright), \gamma\mleft(t\mright)\right) &  & \text{if } t = a_i,                           \\
			 & \left(N\mleft(a_i, a_{i + 1}\mright), \gamma\mleft(t\mright)\right)     &  & \text{if } t \in \left(a_i, a_{i + 1}\right). \\
		\end{aligned}\right.\]
	It is easy to verify that $\tilde{\gamma}$ is continuous -- it suffices to check at each $a_i$ -- hence it is the desired lift of $\gamma$.
\end{proof}

It follows immediately that:

\begin{corollary}
	\label{cor:path-winding-number-map}
	For any definable path $\gamma$ and some $w \in K \setminus \gamma^*$, there exists a definable continuous function $\tilde{\gamma}_w : \left[0, 1\right] \rightarrow \mathcal{H}$ such that, for any $t \in \left[0, 1\right]$,
	\[\gamma\mleft(t\mright) = w + \left|\gamma\mleft(t\mright) - w\right| \cdot \pi\mleft(\tilde{\gamma}_w\mleft(t\mright)\mright).\]

	Additionally, this function is unique up to an integer constant, i.e.\ if $\alpha, \beta : \left[0, 1\right] \rightarrow \mathcal{H}$ both satisfy the condition above, then $\alpha = \beta + \left(n, 1\right)$ for some $n \in \mathbb{Z}$.
\end{corollary}

\begin{definition}[Winding numbers]
	For any definable path $\gamma$ and some $w \in K \setminus \gamma^*$, we define $\gamma$'s \emph{winding number around $w$} to be
	\[W\mleft(\gamma, w\mright) = \tilde{\gamma}_w\mleft(1\mright) - \tilde{\gamma}_w\mleft(0\mright) \in \mathcal{H}.\]
	\autoref{cor:path-winding-number-map} ensures that this is well-defined and independent of the choice of $\tilde{\gamma}_w$.

	Especially, observe that
	\[\pi\mleft(W\mleft(\gamma, w\mright)\mright) = \frac{\pi\mleft(\tilde{\gamma}_w\mleft(1\mright)\mright)}{\pi\mleft(\tilde{\gamma}_w\mleft(0\mright)\mright)} = \frac{\left(\gamma\mleft(1\mright) - w\right) \left|\gamma\mleft(0\mright) - w\right|}{\left(\gamma\mleft(0\mright) - w\right) \left|\gamma\mleft(1\mright) - w\right|}.\]
	Therefore, if $\gamma$ is circular, then $W\mleft(\gamma, w\mright) = \left(n, 1\right)$ for some $n \in \mathbb{Z}$. We will abuse notation and simply say $W\mleft(\gamma, w\mright) = n$ instead if we are only concerned with circular paths.
\end{definition}

As shown in \cite{differentiability-theory}, this reconstruction of the winding numbers shares many immediate properties of its classical counterpart:

\begin{proposition}
	\label{prop:winding-number-for-path-operations}
	For definable paths $\gamma_1$, $\gamma_2$ and $w \in K \setminus \left(\gamma_1^* \cup \gamma_2^*\right)$,
	\begin{packeditemize}
		\item we denote the \emph{opposite path} of $\gamma_1$ as $\gamma_1^-$, such that $\gamma_1^-\mleft(t\mright) = \gamma_1\mleft(1 - t\mright)$, then $W\mleft(\gamma_1^-, w\mright) = -W\mleft(\gamma_1, w\mright)$;
		\item for any definable increasing bijection $s : \left[0, 1\right] \rightarrow \left[0, 1\right]$, $\gamma_1 \circ s$ is also a definable path with $W\mleft(\gamma_1 \circ s, w\mright) = W\mleft(\gamma_1, w\mright)$;
		\item if $\gamma_1\mleft(1\mright) = \gamma_2\mleft(0\mright)$, then we denote the \emph{concatenation} of $\gamma_1$ and $\gamma_2$ as $\gamma_1 \star \gamma_2$, such that
		\[{\left(\gamma_1 \star \gamma_2\right)}\mleft(t\mright) = \left\{\begin{aligned}
				 & \gamma_1\mleft(2t\mright)     &  & \text{if } 0 \leq t \leq 1 / 2, \\
				 & \gamma_2\mleft(2t - 1\mright) &  & \text{if } 1 / 2 < t \leq 1,
			\end{aligned}\right.\]
		and $W\mleft(\gamma_1 \star \gamma_2, w\mright) = W\mleft(\gamma_1, w\mright) + W\mleft(\gamma_2, w\mright)$.
	\end{packeditemize}
\end{proposition}

\begin{proposition}
	\label{prop:winding-number-additivity}
	For definable paths $\gamma_1$, $\gamma_2$ such that $0 \not\in \gamma_1^* \cup \gamma_2^*$, we have
	\[W\mleft(\gamma_1 \cdot \gamma_2, 0\mright) = W\mleft(\gamma_1, 0\mright) + W\mleft(\gamma_2, 0\mright),\]
	where $\gamma_1 \cdot \gamma_2$ is given by $t \mapsto \gamma_1\mleft(t\mright) \cdot \gamma_2\mleft(t\mright)$.
\end{proposition}

\begin{proposition}
	\label{prop:non-surjective-winding-number-0}
	If $\gamma$ is a definable circular path such that $\gamma^* \subsetneq S^1$, then $W\mleft(\gamma, 0\mright) = 0$.
\end{proposition}

\begin{proposition}
	\label{prop:constant-winding-number-in-connected-component}
	Let $\gamma$ be a definable circular path and $U$ a definably connected component of $C \setminus \gamma^*$. For any $w_1, w_2 \in U$,
	\[W\mleft(\gamma, w_1\mright) = W\mleft(\gamma, w_2\mright).\]
\end{proposition}

Here \autoref{prop:constant-winding-number-in-connected-component} is a direct corollary of the following useful lemma:

\begin{lemma}
	\label{lem:connected-parameter-constant-winding-number}
	Let $U \subseteq K$ be a definable, definably connected set and $h : \left[0, 1\right] \times U \rightarrow S^1$ be a definable continuous function such that, for any $z \in U$, $h_z : t \mapsto h\mleft(t, z\mright)$ is a circular path. Then $W\mleft(h_z, 0\mright)$ is constant over $U$.
\end{lemma}

To prove this lemma, we shall cite without proof the following fact about general o-minimal structures, numbered Fact 2.4 in \cite{differentiability-theory}:

\begin{fact}
	\label{fact:compact-parameter-uniform-convergence}
	Let $X, Y, Z$ be definable sets in an o-minimal structure with definable choice\footnote{Definable choice refers to finding a definable function $f : \pi\mleft(S\mright) \rightarrow R^n$, whose graph is contained in the given definable set $S \subseteq R^{m + n}$, where $\pi : R^{m + n} \rightarrow R^m$ is the projection onto the first $m$ coordinates. As proven in chapter 6 of van den Dries' book \cite{tame-topology}, this is possible for any o-minimal structure with an ordered abelian group operation.}, where $X$ is definably compact\footnote{Due to length constraints we do not discuss the concept of definable compactness in detail in this essay. We do use here Theorem 2.1 in \cite{definable-compactness}, an o-minimal analog of the
		Bolzano-Weierstrass theorem, that a definable set in an o-minimal structure is definably compact if and only if it is closed and bounded.}, an interval for example, and $Z$ carries some definable norm onto a real closed field. If $f : X \times Y \rightarrow Z$ is a definable continuous map and $y_0 \in Y$, then for every $\varepsilon > 0$ there is a neighbourhood $U$ of $y_0$, open in $Y$, such that
	\[\left|f\mleft(x, y\mright) - f\mleft(x, y_0\mright)\right| < \varepsilon\]
	for any $x \in X$, $y \in U$.
\end{fact}

\begin{proof}[Proof of \autoref{lem:connected-parameter-constant-winding-number}]
	\fullref{thm:cell-decomposition} yields a finite decomposition of $K$ that partitions the set
	\[\left\{\left(t, z\right) : t \in \left[0, 1\right], z \in U, h\mleft(t, z\mright) = 1\right\},\]
	on which a similar recursion as in the proof of \autoref{thm:path-lifting} can be performed. Therefore, $h$ has a definable lift $\tilde{h} : \left[0, 1\right] \times U \rightarrow \mathcal{H}$. Let
	\[W\mleft(h_z, 0\mright) = \tilde{h}\mleft(1, z\mright) - \tilde{h}\mleft(0, z\mright),\]
	then we have definable sets $\left\{z \in U : W\mleft(h_z, 0\mright) = n\right\}$ for each $n \in \mathbb{Z}$.

	Now, we assume that $W\mleft(h_{z_0}, 0\mright) = n$ for some $z_0 \in \mathbb{Z}$. We consider $h' : \left[0, 1\right] \times U \rightarrow S^1$ given by
	\[h'\mleft(t, z\mright) = \frac{h\mleft(t, z\mright)}{h\mleft(t, z_0\mright)},\]
	then $h'$ is continuous with $h'\mleft(t, z_0\mright) = 1$ for any $t \in \left[0, 1\right]$.

	Using \autoref{fact:compact-parameter-uniform-convergence}, we can find a neighbourhood $V$ of $z_0$, open in $U$, such that for any $z \in V$,
	\[\left|h'\mleft(t, z\mright) - h'\mleft(t, z_0\mright)\right| < 2,\]
	i.e. $h'\mleft(t, z\mright) \neq -1$. If we consider $h'_z : t \mapsto h'\mleft(t, z\mright)$, then $h'_z$ is not surjective, and by \autoref{prop:non-surjective-winding-number-0}, $W\mleft(h'_z, 0\mright) = 0$ for any $z \in V$, and by \autoref{prop:winding-number-additivity} we have
	\[W\mleft(h_z, 0\mright) = W\mleft(h'_z, 0\mright) + W\mleft(h_{z_0}, 0\mright) = W\mleft(h_{z_0}, 0\mright).\]

	Therefore, the sets $\left\{z \in U : W\mleft(h_z, 0\mright) = n\right\}$ are open in $U$, and since $U$ is definably connected we must have $W\mleft(h_z, 0\mright) = n$ on the entirety of $U$ for some fixed $n \in \mathbb{Z}$.
\end{proof}

\subsection{Simple closed curves}

\begin{definition}[Simple closed curves]
	We say that a definable set $C \subseteq K$ is a \emph{simple closed curve} if $C = \gamma^*$ for some definable circular path $\gamma$, such that its restriction $\left.\gamma\right|_{\left[0, 1\right)}$ is a bijection.
\end{definition}

The o-minimal analog of Jordan curve theorem is proven in Woerheide's PhD thesis \cite{woerheide-curve-theorem}:

\begin{theorem}
	\label{thm:jordan-curve-theorem}
	If $C \subseteq K$ is a definable simple closed curve, then $K \setminus C$ is a union of two disjoint definably connected open sets, one of which is bounded and the other is unbounded.
\end{theorem}

We call the bounded component the \emph{interior} of $C$, $\Int\mleft(C\mright)$. Due to \autoref{prop:winding-number-for-path-operations} and \autoref{prop:non-surjective-winding-number-0} for our definition of winding numbers, we can now replicate exactly the classical proof\footnote{For example, refer to Theorem V.1.4 and V.2.2 in \cite{topological-analysis}.} for the following result:

\begin{lemma}
	Let $\gamma$ be a definable circular path with bijective restriction $\left.\gamma\right|_{\left[0, 1\right)}$, such that $C = \gamma^*$ is a simple closed curve. Then
	\begin{packeditemize}
		\item for $w$ in the unbounded component of $K \setminus C$, $W\mleft(\gamma, w\mright) = 0$;
		\item for $w \in \Int\mleft(C\mright)$, $W\mleft(\gamma, w\mright) \in \left\{1, -1\right\}$.
	\end{packeditemize}

	We specifically say that $\gamma$ has \emph{positive orientation} if $W\mleft(\gamma, w\mright) = 1$ for $w \in \Int\mleft(C\mright)$, and we say that $\gamma$ has \emph{negative orientation} otherwise.
\end{lemma}

Notably, if a simple closed curve $C = \gamma^*$ is given by a definable circular path $\gamma$ with negative orientation, then $W\mleft(\gamma^-, w\mright) = 1$ by \autoref{prop:winding-number-for-path-operations}. So we can always define a simple closed curve using a path with positive orientation. Thus, from here onwards, when we say that a definable path $\gamma$ is a \emph{parametrisation} for simple closed curve $C = \gamma^*$, we always imply that $\gamma$ has positive orientation.

\begin{definition}
	Let $C$ be a simple closed curve with parametrisation $\gamma$ and $f : C \rightarrow K$ be a definable continuous function. For any $w \in K \setminus f\mleft(C\mright)$, we define $f$'s \emph{winding number along curve $C$ around $w$} to be
	\[W_C\mleft(f, w\mright) = W\mleft(f \circ \gamma, w\mright).\]
\end{definition}

\begin{definition}[Star-shaped curves]
	A simple closed curve $C$ is \emph{star-shaped} if there exists some $p \in \Int\mleft(C\mright)$ such that
	\[\left(1 - t\right)p + tz \in \Int\mleft(C\mright)\]
	for any $z \in C$ and $t \in \left[0, 1\right)$. Infomally, $C$ is \emph{star-shaped} if for any $z \in C$, the line segment between $z$ and $p$ lies in its interior.
\end{definition}

When $C$ is star-shaped, we can explicitly identify the two definably connected subsets of $K \setminus C$, namely the inner subset
\[\Int\mleft(C\mright) = \left\{\left(1 - t\right)p + tz : t \in \left[0, 1\right), z \in C\right\}\]
and the outer one $\left\{\left(1 - t\right)p + tz : t > 1, z \in C\right\}$. This allows us to prove the following proposition:

\begin{proposition}
	\label{prop:exterior-of-image-zero-winding-number}
	Let $C$ be a definable, star-shaped simple closed curve. Let $f : C \cup \Int\mleft(C\mright) \rightarrow K$ be a definable continuous function. For any $w \in K \setminus f\mleft(C \cup \Int\mleft(C\mright)\mright)$, $W_C\mleft(f, w\mright) = 0$.
\end{proposition}

\begin{proof}
	Suppose that $C$ has parametrisation $\gamma$ and is made star-shaped by the point $p \in \Int\mleft(C\mright)$. We consider the map $z : \left[0, 1\right] \times \left[0, 1\right] \rightarrow f\mleft(C \cup \Int\mleft(C\mright)\mright)$ given by
	\[z\mleft(t, s\mright) = f\mleft(\left(1 - s\right)p + s\gamma\mleft(t\mright)\mright).\]

	Define $h : \left[0, 1\right] \times \left[0, 1\right] \rightarrow S^1$ by
	\[h\mleft(t, s\mright) = \frac{z\mleft(t, z\mright) - w}{\left|z\mleft(t, z\mright) - w\right|},\]
	which is a well-defined continuous function because $w \not\in f\mleft(C \cup \Int\mleft(C\mright)\mright)$. By \autoref{lem:connected-parameter-constant-winding-number} $W\mleft(h_s, 0\mright)$ is constant over $\left(0, 1\right]$, so
	\[W_C\mleft(f, w\mright) = W\mleft(h_1, 0\mright) = W\mleft(h_s, 0\mright)\]
	for any $s \in \left(0, 1\right)$.

	Additionally, $h_0 : t \mapsto \left(f\mleft(p\mright) - w\right) / \left|z\mleft(t, z\mright) - w\right|$ is constant. Invoking \autoref{fact:compact-parameter-uniform-convergence}, we must be able to find $\varepsilon > 0$ such that, for any $s \in \left(0, \varepsilon\right)$, $t \in \left[0, 1\right]$,
	\[\left|h\mleft(t, s\mright) - h\mleft(t, 0\mright)\right| < 2.\]

	This means that $h_s$ must not be a surjective function onto $S^1$ and hence $W\mleft(h_s, 0\mright) = 0$ by \autoref{prop:non-surjective-winding-number-0}. We can thus conclude that $W_C\mleft(f, w\mright) = 0$ as well.
\end{proof}

\subsection{Differentiability}

\begin{definition}
	Given a definable open set $U \subseteq R^m$ and definable $f : U \rightarrow R^n$, we say that $f$ is \emph{$R$-differentiable} at $x \in U$ if there exists a linear map $T : R^m \rightarrow R^n$ such that
	\[\lim_{h \rightarrow 0} \frac{\left|f\mleft(x + h\mright) - f\mleft(x\mright) - T\mleft(h\mright)\right|}{\left|h\right|} = 0.\]
	We would then denote $T$ as $df\mleft(x\mright)$, the $R$-differential at $x$.

	Given definable open $U \subseteq K$ and definable $f : U \rightarrow K$, we say that $f$ is \emph{$K$-differentiable} at $x \in U$ if $f$ is $R$-differentiable as a function on $R^2$ and its differential can be written as $df\mleft(x\mright) : h \mapsto \lambda h$ for some $\lambda \in K$. We then say that the $K$-derivative of $f$ at $x$, $f'\mleft(x\mright) = \lambda$.
\end{definition}

It is easy to see that for $f : x + iy \mapsto u\mleft(x, y\mright) + iv\mleft(x, y\mright)$, $f$ is $K$-differentiable at $z$ if and only if it is $R$-differentiable at $z$ and the Cauchy-Riemann equations hold:
\[\frac{\partial u}{\partial x}\mleft(z\mright) = \frac{\partial v}{\partial y}\mleft(z\mright), \quad \frac{\partial u}{\partial y}\mleft(z\mright) = -\frac{\partial v}{\partial x}\mleft(z\mright).\]

The concept of winding numbers is useful here due to the following simple lemma, proven in subsection 2.4 of Peterzil and Starchenko's paper \cite{differentiability-theory}:

\begin{fact}
	\label{fact:non-zero-derivative-winding-number-1}
	Given a definable open set $U \subseteq K$ and definable $f : U \rightarrow K$, if $f$ is $K$-differentiable at $z \in U$ with $f'\mleft(z\mright) \neq 0$, then there exists an $\varepsilon > 0$ such that, for every $r < \varepsilon$, if $C_r$ is the circle around $z$ with radius $r$, i.e.\
	\[C_r = \left\{x \in K : \left|x - z\right| = r\right\},\]
	then $W_{C_r}\mleft(f, f\mleft(z\mright)\mright) = 1$.
\end{fact}

This leads immediately to the following result, part of Lemma 2.30 in \cite{ differentiability-theory}:

\begin{lemma}
	\label{lem:image-of-interior-in-interior-of-image}
	Let $C$ be a definable simple closed curve and $f : C \cup \Int\mleft(C\mright) \rightarrow K$ be a definable continuous function that is $K$-differentiable on $\Int\mleft(C\mright) \setminus L$ for some definable subset $L$ with $\dim L \leq 1$. Let $U$ be a definably connected component of $K \setminus f\mleft(C\mright)$ and $U \cup f\mleft(\Int\mleft(C\mright)\mright) \neq \varnothing$, then for any $w \in U$, $W_C\mleft(f, w\mright) > 0$.
\end{lemma}

As a sketch for the proof, via cell decomposition one can find a generic point $w \in U$ here, such that $f^{-1}\mleft(w\mright) = \left\{z_1, \ldots, z_k\right\}$ is finite, and $f$ is differentiable at each $z_i$ with $f'\mleft(z_i\mright) \neq 0$. Using \autoref{fact:non-zero-derivative-winding-number-1} and path operations in \autoref{prop:winding-number-for-path-operations}, we can compute
\[W_C\mleft(f, w\mright) = \sum_{i = 1}^k W_{C_i}\mleft(f, w\mright) = k > 0,\]
where each $C_i$ is a small enough circle around $z_i$.

Now we can prove an important lemma in definable complex analysis:

\begin{lemma}[Maximum principle for star-shaped curves]
	\label{lem:maximum-principle-star-shaped}
	Let $C$ be a definable, star-shaped simple closed curve. Let $f : C \cup \Int\mleft(C\mright) \rightarrow K$ be a definable continuous function that is $K$-differentiable on $\Int\mleft(C\mright) \setminus L$ for some definable subset $L$ with $\dim L \leq 1$. For any $w \in \Int\mleft(C\mright)$,
	\[f\mleft(w\mright) \in f\mleft(C\mright) \cup \interior\mleft(f\mleft(\Int\mleft(C\mright)\mright)\mright).\]
	In particular, we have
	\[\left|f\mleft(w\mright)\right| \leq \max_{z \in C \cup \Int\mleft(C\mright)} \left|f\mleft(z\mright)\right| = \max_{z \in C} \left|f\mleft(z\mright)\right|.\]
\end{lemma}

\begin{proof}
	Assume that $f\mleft(w\mright) \not\in f\mleft(C\mright)$. We can find the definably connected component $W$ of $K \setminus f\mleft(C\mright)$ containing $f\mleft(w\mright)$, then
	\[f\mleft(w\mright) \in W \cup f\mleft(\Int\mleft(C\mright)\mright).\]

	Therefore, by \autoref{lem:image-of-interior-in-interior-of-image}, for any $u \in W$, $W_C\mleft(f, u\mright) \neq 0$. By \autoref{prop:exterior-of-image-zero-winding-number}, this means that $W \subseteq f\mleft(C \cup \Int\mleft(C\mright)\mright)$. Now, by definition, $W \subseteq K \setminus f\mleft(C\mright)$, so $W \subseteq f\mleft(\Int\mleft(C\mright)\mright)$ and hence
	\[f\mleft(w\mright) \in \interior\mleft(f\mleft(\Int\mleft(C\mright)\mright)\mright)\]
	because $W$ is additionally open.
\end{proof}

From the maximum principle, useful results can be established:

\begin{theorem}[Identity theorem]
	Let $U \subseteq K$ be a definably connected open set and $f : U \rightarrow K$ be a definable $K$-differentiable function. We define
	\[\hat{U} = \left\{w \in \closure\mleft(U\mright) : f\mleft(z\mright) \text{ converges as } z \rightarrow w\right\},\]
	and consider $\hat{f} : \hat{U} \rightarrow K$ given by $\hat{f}\mleft(w\mright) = \lim_{z \rightarrow w} f\mleft(z\mright)$. If $\hat{f}^{-1}\mleft(u\mright)$ is infinite for some $u \in K$, then $\hat{f}\mleft(z\mright) = u$ for all $z \in \hat{U}$.
\end{theorem}

The proof for this result, found as Theorem 2.33(i) in \cite{differentiability-theory}, is again rather technical. We provide here a sketch for it:

Let $X = \hat{f}^{-1}\mleft(u\mright)$ be infinite. Suppose for contradiction that $\hat{f}$ is not constant, then $X$ is closed in $\hat{U}$ and in an appropriate cell decomposition we can find an open cell $C$ in $U \setminus X$ with a $1$-dimensional intersection $\frontier\mleft(C\mright) \cap X$. Now, by choosing some $w \in C$ close enough to that intersection and rotate $\hat{f}$ around $w$:
\[g\mleft(z\mright) = \prod_{a = 0}^3 \left(\hat{f}\mleft(i^a \left(z - w\right) + w\mright) - u\right),\]
$g$ will be zero on some star-shaped curve around $w$ while non-zero in the interior, contradicting \autoref{lem:maximum-principle-star-shaped}. Therefore, $\hat{f}$ must be constant in this case.

Now, an analog of the standard identity theorem in classical complex analysis is an immediate corollary:

\begin{corollary}[Identity theorem]
	\label{cor:identity-theorem}
	Let $U \subseteq K$ be a definably connected open set and $f : U \rightarrow K$ be a definable $K$-differentiable function. If $f^{-1}\mleft(u\mright)$ is infinite for some $u \in K$, then $f\mleft(z\mright) = u$ for all $z \in U$.
\end{corollary}

Furthermore, we shall end by recovering the definable version of Liouville's theorem:

\begin{theorem}[Liouville's theorem]
	\label{thm:liouvilles-theorem}
	If $f : K \rightarrow K$ is a definable $K$-differentiable function such that $\left|f\right|$ is bounded on $K$, then $f$ is a constant function.
\end{theorem}

\begin{proof}
	Consider $h : K \rightarrow K$ given by
	\[h\mleft(z\mright) = \left\{\begin{aligned}
			 & \frac{f\mleft(z\mright) - f\mleft(0\mright)}{z} &  & \text{if } z \neq 0, \\
			 & f'\mleft(0\mright)                              &  & \text{otherwise}.
		\end{aligned}\right.\]
	We can see that $h$ is continuous on $K$ and $K$-differentiable on $K \setminus \left\{0\right\}$. Therefore, given some $w \in K \setminus \left\{0\right\}$, for any circle
	\[C_r = \left\{z \in K : \left|z\right| = r\right\}\]
	with radius $r > \left|w\right|$, we can apply \autoref{lem:maximum-principle-star-shaped} and claim that
	\[\left|h\mleft(w\mright)\right| = \frac{\left|f\mleft(w\mright) - f\mleft(0\mright)\right|}{\left|z\right|} \leq \max_{z \in C_r} \frac{\left|f\mleft(z\mright) - f\mleft(w\mright)\right|}{r} \leq \frac{2M}{r},\]
	where $M$ is an upper bound for $\left|f\right|$.

	Since $2M / r \rightarrow 0$ as $r \rightarrow \infty$, we can conclude that $f\mleft(w\mright) - f\mleft(0\mright) = 0$ for any $w \in K \setminus \left\{0\right\}$, i.e.\ $f$ is constant on $K$.
\end{proof}

\subsection{Singularities}

We now proceed to the analysis of singularities of a $K$-differentiable function, where definability allows us to disregard many badly behaving functions. We start with quite a strong theorem, that $1$-dimensional singularities of a continuous function are removable:

\begin{theorem}
	\label{thm:dimension-1-singularity-removable}
	Let $U \subseteq K$ be a definable open set and $f : U \rightarrow K$ be a definable continuous function that is $K$-differentiable on $U \setminus L$ for some definable subset $L$ with $\dim L \leq 1$. Then $f$ is $K$-differentiable on $U$.
\end{theorem}

In fact, for any $w \in U \cap L$, one can pick an arbitrary path\footnote{Here, in any decomposition that partitions $U$ and $\closure\mleft(L\mright)$, all open cells in $U$ lies in $U \setminus \closure\mleft(L\mright)$, so $w$ must lie on the frontier of one such open cell. In a real closed field structure, these open cells are homeomorphic to unit boxes, and we can easily find a path from the inside of the cell to $w$ on the frontier.} $\gamma$ such that $\gamma\mleft(\left[0, 1\right)\mright) \subseteq U \setminus \closure\mleft(L\mright)$ and $\gamma\mleft(1\mright) = w$, then the derivative can be expicitly computed as
\[f'\mleft(w\mright) = \lim_{t \rightarrow 1^-} \frac{f\mleft(\gamma\mleft(t\mright)\mright) - f\mleft(w\mright)}{\gamma\mleft(t\mright) - w}.\]
Due to length constraints, we shall omit the technical verification in subsection 2.7 of \cite{differentiability-theory} that the limit is independent of the choice of $\gamma$: it is basically multiple applications of \autoref{lem:maximum-principle-star-shaped}.

The more well-known result, that isolated singularities can be removed so long as the function is locally bounded, is a corollary of this:

\begin{corollary}
	\label{cor:bounded-singularity-removable}
	Let $U \subseteq K$ be a definable open set and $f : U \setminus \left\{w\right\} \rightarrow K$ be a definable $K$-differentiable function for some $w \in U$. If $f$ is bounded in a neighbourhood of $w$, then $f$ extends to $w$ as a $K$-differentiable function.
\end{corollary}

\begin{proof}
	We define $g : U \rightarrow K$ by
	\[g\mleft(z\mright) = \left\{\begin{aligned}
			 & \left(z - w\right) f\mleft(z\mright) &  & \text{if } z \neq w, \\
			 & 0                                    &  & \text{otherwise.}
		\end{aligned}\right.\]
	Since $f$ is bounded near $z$, $g$ is continuous on $U$ and $K$-differentiable on $U \setminus \left\{w\right\}$. By \autoref{thm:dimension-1-singularity-removable}, $g$ is $K$-differentiable on $U$.

	By definition, $g'\mleft(w\mright) = \lim_{z \rightarrow w} f\mleft(z\mright)$. Therefore, setting $f\mleft(w\mright) = g'\mleft(w\mright)$ makes $f$ continuous on $U$. By \autoref{thm:dimension-1-singularity-removable} again we know that this extension is $K$-differentiable on $U$.
\end{proof}

While these results are analogous to their classical counterparts, definable complex analysis is actually more ``tame''. As proven in subsection 2.9 of \cite{differentiability-theory}, isolated non-removable singularities are all poles:

\begin{proposition}
	\label{prop:function-or-inverse-extend-to-singularity}
	Let $U \subseteq K$ be a definable open set and $f : U \setminus \left\{w\right\} \rightarrow K$ be a definable $K$-differentiable function for some $w \in U$. Then, either $f$ or $1 / f$ extends to a $K$-differentiable function in an open neighbourhood of $w$.
\end{proposition}

\begin{proof}
	Consider the graph $\Gamma\mleft(f\mright) \subseteq K^2 \cong R^4$. By \autoref{fact:bijections-preserve-dimension} we have $\dim \Gamma\mleft(f\mright) = \dim\mleft(U \setminus \left\{w\right\}\mright) = 2$. Thus, by \autoref{prop:frontier-dimension}, $\dim \frontier\mleft(\Gamma\mleft(f\mright)\mright) < 2$ and hence $\left\{w\right\} \times K \not\subseteq \frontier\mleft(\Gamma\mleft(f\mright)\mright)$. In other words, there exists some $z \in K$ such that
	\[\left(w, z\right) \in K^2 \setminus \closure\mleft(\Gamma\mleft(f\mright)\mright).\]

	$K^2 \setminus \closure\mleft(\Gamma\mleft(f\mright)\mright)$ is open. Viewing $K^2$ as a product topology, we know that there exist $\varepsilon, \delta > 0$ such that, if $D_w$ and $D_z$ are open discs of radius $\varepsilon$ and $\delta$ around $w$ and $z$ respectively, then
	\[\left(D_w \times D_z\right) \cap \Gamma\mleft(f\mright) = \varnothing.\]
	This means that $\left|f\mleft(x\mright) - z\right| \geq \delta$ for any $x \in D_w \setminus \left\{w\right\}$, so $1 /  \left(f\mleft(x\mright) - z\right)$ is bounded by $1 / \delta$. By \autoref{cor:bounded-singularity-removable}, $x \mapsto 1 /  \left(f\mleft(x\mright) - z\right)$ extends to a $K$-differentiable function on $D_w$.

	This means that $\lim_{x \rightarrow w} 1 / \left(f\mleft(x\mright) - z\right)$ is well-defined. Depending on whether $\lim_{x \rightarrow w} 1 / \left(f\mleft(x\mright) - z\right) = 0$, either $1 / f$ or $f$ is bounded near $w$, so \autoref{cor:bounded-singularity-removable} ensures that one of them extends to a $K$-differentiable function near $w$.
\end{proof}

The non-existence of essential singularities on definable functions further implies that:

\begin{theorem}
	\label{thm:rational-function}
	If $f : K \setminus A \rightarrow K$ is a definably $K$-differentiable function for finite $A \subseteq K$, then $f$ is a rational function.
\end{theorem}

To prove this, we first need the following counterpart of classical results, established as Lemma 2.42(ii) and Theorem 2.45 in \cite{differentiability-theory}. It follows easily from \autoref{prop:function-or-inverse-extend-to-singularity} above:

\begin{lemma}[Poles and principle parts]
	\label{lem:pole-order}
	Let $U \subseteq K$ be a definable open set and $f : U \setminus \left\{w\right\} \rightarrow K$ be a definable $K$-differentiable function for some $w \in U$. If $1 / f$ can be extended to a $K$-differentiable function near $w$ such that $1 / f\mleft(w\mright) = 0$, i.e.\ $\lim_{z \rightarrow w} f\mleft(z\mright) = \infty$, then there exists a unique integer $n > 0$ such that
	\[\lim_{z \rightarrow w} f\mleft(z\mright) \left(z - w\right)^n = a\]
	is well-defined for some $a \in K \setminus \left\{0\right\}$.

	We say that $w$ is a \emph{pole of order $-n$} for $f$.

	Additionally, there exists $a_{-n}, \ldots, a_{-1} \in K$ such that the function
	\[f\mleft(z\mright) - \sum_{i = 1}^n \frac{a_{-i}}{\left(z - w\right)^i}\]
	extends to a $K$-differentiable function near $w$.
\end{lemma}

\begin{proof}[Proof of \autoref{thm:rational-function}]
	If $w \in A$ is a pole for $f$, then by \autoref{lem:pole-order} we can find $a_{-n}, \ldots, a_{-1}$ such that
	\[f - \sum_{i = 1}^n \frac{a_{-i}}{\left(z - w\right)^i}\]
	can be extended to $w$ as a $K$-differentiable function. The subtrahend is additionally $K$-differentiable on $K \setminus \left\{w\right\}$. Therefore, let $w_1, \ldots, w_k$ enumerate all the poles for $f$, then we can define rational function
	\[g\mleft(z\mright) = \sum_{j = 1}^k \sum_{i = 1}^{n_j} \frac{a^{(j)}_{-i}}{\left(z - w_j\right)^i}\]
	such that $f - g$ extends to a $K$-differentiable function on the entirety of $K$.

	Now, define $h : z \mapsto f\mleft(1 / z\mright) - g\mleft(1 / z\mright)$, then $h$ is $K$-differentiable on $K \setminus \left\{0\right\}$. Again by \autoref{lem:pole-order}, we can find $b_{-r}, \ldots, b_{-1}$ (where possibly $r = 0$, if the singularity $0$ is removable for $h$) such that
	\[h - \sum_{i = 1}^r \frac{b_{-i}}{z^i}\]
	is $K$-differentiable on the entirety of $K$. It is definable and continuous, so it will be bounded on any definably compact set. Additionally,
	\[\lim_{z \rightarrow \infty} \left(h\mleft(z\mright) - \sum_{i = 1}^r \frac{b_{-i}}{z^i}\right) = f\mleft(0\mright) - g\mleft(0\mright),\]
	so the function is bounded near infinity as well. By \autoref{thm:liouvilles-theorem},
	\[h - \sum_{i = 1}^r \frac{b_{-i}}{z^i} = c\]
	for some constant $c \in K$. Hence
	\[f = g + c + \sum_{i = 1}^r b_{-1} z^i\]
	is a rational function.
\end{proof}

\section{O-minimal structures for classical analysis}
\label{sec:applications}

\subsection{Restricted analytic functions and subanalytic sets}

We now return to the standard fields $\mathbb{R}$ and $\mathbb{C}$. Let $\bar{\mathbb{R}} = \left\langle \mathbb{R}; <, +, -, \times, 0, 1 \right\rangle$ be the model for reals in the language of $\RCF$, then \autoref{prop:rcf-quantifier-elimination} implies that the theory for $\bar{\mathbb{R}}$ admits elimination of quantifiers, and immediately:

\begin{proposition}
	The definable sets in $\bar{\mathbb{R}}$ are exactly the semialgebraic sets.
\end{proposition}

We seek to obtain an expansion of $\bar{\mathbb{R}}$ that admits a larger portion of real analytic functions\footnote{For a holomorphic function $f$ in $\mathbb{C}$, both $\Real\mleft(f\mright)$ and $\Imag\mleft(f\mright)$ are real analytic. Thus, simply looking for an expanded model of reals that defines more analytic functions enables us to apply definably complex analysis to more holomorphic functions.}. Notably, we can make the following explicit stipulation:

\begin{definition}
	We say that a function $f : \mathbb{R}^n \rightarrow \mathbb{R}$ is a \emph{restricted analytic function}\footnote{We use the definition in \cite{real-subanalytic-sets}. It is obvious that any analytic function with domain restricted to a bounded box can be put into this definition with affine transformations.} if $f\mleft(\left[-1, 1\right]^n\mright) \subseteq \left[-1, 1\right]$, $f = 0$ on $\mathbb{R}^n \setminus \left[-1, 1\right]^n$, and $f$ is real analytic on $\left[-1, 1\right]^n$, i.e.\ $f$ can be written as a power series that converges on $\left[-1, 1\right]^n$.

	Let $\mathbb{R}_{\textnormal{an}}$ be an expansion of $\bar{\mathbb{R}}$ with a new function symbol $f$ for each restricted analytic function; let $\mathbb{R}_{\textnormal{an}}^D$ be an expansion of $\mathbb{R}_{\textnormal{an}}$ with a new binary function symbol $D$ such that
	\[D\mleft(x, y\mright) = \left\{\begin{aligned}
			 & x / y &  & \text{if } y \neq 0 \text{ and } \left|x\right| \leq \left|y\right|, \\
			 & 0     &  & \text{otherwise}.
		\end{aligned}\right.\]
\end{definition}

$D$ is easily definable in $\mathbb{R}_{\textnormal{an}}$, thus the definable sets in $\mathbb{R}_{\textnormal{an}}$ and $\mathbb{R}_{\textnormal{an}}^D$ coincide. However, with the addition of $D$, we have the following result by Denef and van den Dries in \cite{real-subanalytic-sets}:

\begin{proposition}
	$\left[-1, 1\right]$ has elimination of quantifiers in the theory of $\mathbb{R}_{\textnormal{an}}^D$.
\end{proposition}

In other words, all definable subsets of $\left[-1, 1\right]$ are obtainable from boolean algebra and projection in addition to real analytic equalities and inequalities. In fact, given the follow notions in real analysis:

\begin{definition}[Semianalytic and subanalytic sets]
	A set $S \subseteq \mathbb{R}^n$ is called \emph{semianalytic} if, at each point $x \in \mathbb{R}^n$, there exists an open neighbourhood $U$ of $x$ such that $S \cap U$ is a finite union of the sets of the form
	\[\left\{y \in U : f\mleft(y\mright) = 0, g_1\mleft(y\mright) > 0, \ldots, g_k\mleft(y\mright) > 0\right\},\]
	where $f, g_1, \ldots, g_k$ are analytic functions on $U$.

	A set $S \subseteq \mathbb{R}^n$ is called \emph{subanalytic} if at each point $x \in \mathbb{R}^n$, there exists an open neighbourhood $U$ of $x$ and a relatively compact semianalytic $S' \subseteq \mathbb{R}^{n + m}$ such that $S \cap U = \pi\mleft(S'\mright) \cap U$, where $\pi : \mathbb{R}^{n + m} \rightarrow \mathbb{R}^n$ is the projection onto the first $n$ coordinates.
\end{definition}

Then it is proven in \cite{real-subanalytic-sets} that

\begin{theorem}
	\label{thm:compact-definable-sets-of-ran}
	The definable subsets of $\left[-1, 1\right]^n$ in $\mathbb{R}_{\textnormal{an}}^D$ are exactly the subanalytic subsets.
\end{theorem}

In order to deal with definable subsets of the whole of $\mathbb{R}^n$, we identify $\mathbb{R}^n$ as $\left(-1, 1\right)^n$ through a compactification:

\begin{definition}[Globally subanalytic sets]
	Consider $\varphi_n : \mathbb{R}^n \rightarrow \left[-1, 1\right]^n$ given by
	\[\varphi_n\mleft(x_1, \ldots, x_n\mright) = \left(\frac{x_1}{\sqrt{x_1^2 + 1}}, \ldots, \frac{x_n}{\sqrt{x_n^2 + 1}}\right).\]

	We say that $S \subseteq \mathbb{R}^n$ is \emph{globally subanalytic} if $\varphi_n\mleft(S\mright)$ is subanalytic.
\end{definition}

\begin{corollary}
	The definable sets in $\mathbb{R}_{\textnormal{an}}$ or $\mathbb{R}_{\textnormal{an}}^D$ are exactly the globally subanalytic sets.
\end{corollary}

\begin{proof}
	We simply note that $\varphi_n$ is a definable bijection between $\mathbb{R}^n$ and $\left(-1, 1\right)^n$. Thus, by \autoref{thm:compact-definable-sets-of-ran}, $S \subseteq \mathbb{R}^n$ is definable in $\mathbb{R}_{\textnormal{an}}^D$ if and only if $\varphi_n\mleft(S\mright)$ is definable in $\mathbb{R}_{\textnormal{an}}^D$, if and only if $\varphi_n\mleft(S\mright)$ is subanalytic, if and only if $S$ is globally subanalytic.
\end{proof}

It is proven by \L{}ojasiewicz in section 16 of \cite{subanalytic-sets-lojasiewicz} that a semianalytic set is a locally finite union of its connected components. Via projection, subanalytic subsets of compact $\left[-1, 1\right]$ must be finite unions of intervals and singletons. This means that $\mathbb{R}_{\textnormal{an}}$ is o-minimal.

We know from classical complex analysis that holomorphic functions are analytic. It follows immediately that:

\begin{proposition}
	\label{prop:holomorphic-compact-restrictions-definable}
	If $U \subseteq \mathbb{C}$ is open with $f : U \rightarrow \mathbb{C}$ being a holomorphic function, then for any definable compact $V \subseteq U$, $\left.f\right|_V$ is an analytic function on compact domain, hence is definable in $\mathbb{R}_{\textnormal{an}}$.
\end{proposition}

This allows the easy transferral of local results in definable complex analysis to arbitrary holomorphic functions. For example, we have the following proof of the classical Liouville's theorem using o-minimality, by Kovacsics in \cite{liouvilles-theorem}. This is in almost the same manner as \autoref{thm:liouvilles-theorem}:

\begin{theorem}[Liouville's theorem]
	If $f : \mathbb{C} \rightarrow \mathbb{C}$ is a bounded entire function, then $f$ is a constant function.
\end{theorem}

\begin{proof}
	Let $M$ denote an upper bound for $\left|f\right|$. Given some $w \in \mathbb{C} \setminus \left\{0\right\}$, for any circle
	\[C_r = \left\{z \in \mathbb{C} : \left|z\right| = r\right\}\]
	with radius $r > \left|w\right|$, $\left.f\right|_{C_r \cup \Int\mleft(C_r\mright)}$ is definable in $\mathbb{R}_{\textnormal{an}}$ by \autoref{prop:holomorphic-compact-restrictions-definable}. We can consider $h : C_r \cup \Int\mleft(C_r\mright) \rightarrow \mathbb{C}$ given by
	\[h\mleft(z\mright) = \left\{\begin{aligned}
			 & \frac{f\mleft(z\mright) - f\mleft(0\mright)}{z} &  & \text{if } z \neq 0, \\
			 & f'\mleft(0\mright)                              &  & \text{otherwise}.
		\end{aligned}\right.\]
	$h$ is then definable and continuous on $C_r \cup \Int\mleft(C_r\mright)$ and holomorphic on $\Int\mleft(C_r\mright) \setminus \left\{0\right\}$. We can apply \autoref{lem:maximum-principle-star-shaped} and claim that
	\[\left|h\mleft(w\mright)\right| = \frac{\left|f\mleft(w\mright) - f\mleft(0\mright)\right|}{\left|z\right|} \leq \max_{z \in C_r} \frac{\left|f\mleft(z\mright) - f\mleft(w\mright)\right|}{r} \leq \frac{2M}{r}.\]
	Since $2M / r \rightarrow 0$ as $r \rightarrow \infty$, we can conclude that $f\mleft(z\mright) - f\mleft(w\mright) = 0$ for any $w \in \mathbb{C} \setminus \left\{0\right\}$, i.e.\ $f$ is constant on $\mathbb{C}$.
\end{proof}

\subsection{O-minimal models for the exponential function}

$\mathbb{R}_{\textnormal{an}}$ is still a somewhat restricted model, due to the following fact identified in \cite{polynomial-growth}:

\begin{fact}[Polynomial growth]
	Consider any globally subanalytic function $f : \left(0, \infty\right) \rightarrow \mathbb{R}$. There is some $d \in \mathbb{N}$, $a > 0$ such that $\left|f\mleft(t\mright)\right| < t^d$ for any $t > a$.
\end{fact}

This directly means that the real exponential function $x \mapsto e^x$ cannot be definable in $\mathbb{R}_{\textnormal{an}}$.

It is due to results by Wilkie, Khovanskii, van den Dries and Miller that larger o-minimal extensions of $\bar{\mathbb{R}}$ can contain the real exponential function. Wilkie proved in \cite{r-exp-model-complete} that

\begin{proposition}
	The theory of the structure $\mathbb{R}_{\textnormal{exp}} = \left\langle \bar{\mathbb{R}}; \exp \right\rangle$, where $\exp$ is the usual exponential function $x \mapsto e^x$ on $\mathbb{R}$, is model complete.
\end{proposition}

In other worlds, any first-order formula is equivalent to an existential formula in the theory, and all definable sets will correspondingly be projections of finite unions of sets given by equalities and inequalities involving arithmetic operations and the exponential function only.

Noticeably, if we have an equality (or respectively an inequality) of the form $P\mleft(\exp\mleft(Q\mleft(x\mright)\mright)\mright) = 0$, then we can introduce an additional variable $y$, so that the roots of the equality are exactly the projection of
\[\left\{\left(x, y\right) : P\mleft(\exp\mleft(y\mright)\mright) = y - Q\mleft(x\mright) = 0\right\}\]
onto the first coordinate. Therefore, we can recursively eliminate all expressions where a complex formula occurs inside the exponential, and all definable sets in $\mathbb{R}_{\textnormal{exp}}$ can thus be given by equalities and inequalities only involving the form
\[P\mleft(x_1, \ldots, x_n, \exp\mleft(x_1\mright), \ldots, \exp\mleft(x_n\mright)\mright),\]
where $P \in \mathbb{R}\mleft[X_1, \ldots, X_n, Y_1, \ldots, Y_n\mright]$ is a polynomial. These form Khovanskii's Pfaffian systems for the Pfaffian chain\footnote{Strictly speaking, the chain needed is $\left\langle \exp\mleft(x_1\mright), \ldots, \exp\mleft(x_n\mright) \right\rangle$ here.} $\left\langle \exp \right\rangle$:

\begin{definition}[Pfaffian chains]
	We say that a finite sequence of real analytic functions $\left\langle f_1, \ldots, f_k \right\rangle$ on $n$ variables form a \emph{Pfaffian chain} if for all $1 \leq i \leq n$, $1 \leq j \leq k$ there exists a polynomial $P_{ij} \in \mathbb{R}\mleft[X_1, \ldots, X_n, Y_1, \ldots, Y_j\mright]$ such that
	\[\frac{\partial f_j}{\partial x_i} = P_{ij}\mleft(x_1, \ldots, x_n, f_1, \ldots, f_j\mright).\]

	A \emph{Pfaffian system} for the chain is then a system of equations
	\[Q_1\mleft(x_1, \ldots, x_n, f_1, \ldots, f_k\mright) = \cdots = Q_m\mleft(x_1, \ldots, x_n, f_1, \ldots, f_k\mright) = 0\]
	where $Q_1, \ldots, Q_m \in \mathbb{R}\mleft[X_1, \ldots, X_n, Y_1, \ldots, Y_k\mright]$ are polynomials.
\end{definition}

Khovanskii proved as Theorem 4 in \cite{pfaffian-systems} that:

\begin{theorem}
	Let a set $X \subseteq \mathbb{R}^n$ be defined by a Pfaffian system of equations. Then the number of connected components of $X$ is finite.
\end{theorem}

This immediately implies that the definable subsets of $\mathbb{R}$ in $\mathbb{R}_{\textnormal{exp}}$ are finite unions of intervals and singletons and thus $\mathbb{R}_{\textnormal{exp}}$ is o-minimal.

Extending Wilkie's results, van den Dries and Miller also proved in \cite{r-an-exp} that:

\begin{theorem}
	$\mathbb{R}_{\textnormal{an,exp}} = \left\langle \mathbb{R}_{\textnormal{an}}; \exp \right\rangle$ is model complete and o-minimal.
\end{theorem}

In this expansion, the restriction of the complex exponential function $z \mapsto e^z$ is definable in any horizontal strip $\left\{z \in \mathbb{C} : a < \Imag\mleft(z\mright) < b\right\}$: we can simply write
\[e^z = e^{\Real\mleft(z\mright)} e^{i\Imag\mleft(z\mright)} = e^{\Real\mleft(z\mright)} \left(\cos\mleft(\Imag\mleft(z\mright)\mright) + i \sin\mleft(\Imag\mleft(z\mright)\mright)\right),\]
where $\sin$ and $\cos$ are definable in $\mathbb{R}_{\textnormal{an}}$ on the bounded interval $\left(a, b\right)$.

This is in fact the best one can do with the complex exponential function in any o-minimal structure, due to the following proof in \cite{complex-like-analysis}:

\begin{proposition}
	If $z \mapsto e^z$ is definable in an o-minimal expansion $\mathfrak{A}$ of $\bar{\mathbb{R}}$ on a definable set $U \subseteq \mathbb{R}$, then $\left\{\Imag\mleft(z\mright) : z \in U\right\}$ is bounded on $\mathbb{R}$.
\end{proposition}

\begin{proof}
	Due to results in \cite{pfaffian-closure}, we can always move into to a Pfaffian closure of $\mathfrak{A}$, where the real exponential function is definable.

	Let $S = \left\{\Imag\mleft(z\mright) : z \in U\right\} \subseteq \mathbb{R}$, then we can choose some definable\footnote{An o-minimal expansion of $\bar{\mathbb{R}}$ has definable choice, as proven in chapter 6 of van den Dries' book \cite{tame-topology}.} $z : S \rightarrow U$ such that $\Imag\mleft(z\mleft(r\mright)\mright) = r$ for any $r \in S$. Now, let $f : \mathbb{R} + iS \rightarrow \mathbb{C}$ be given by
	\[f\mleft(x + ir\mright) = e^x e^{z\mleft(r\mright)} e^{-\Real\mleft(z\mleft(r\mright)\mright)} = e^{x + ir}.\]
	$f$ will be holomorphic on $\mathbb{R} + iS$.

	If $S$ is unbounded, since it is definable, we can find some $a \in \mathbb{R}$ such that either $\left(a, \infty\right) \subseteq S$ or $\left(-\infty, a\right) \subseteq S$. Either way,
	\[\left\{2k \pi i \in S : k \in \mathbb{Z}\right\} \subseteq f^{-1}\mleft(1\mright)\]
	will be infinite, contradicting \fullref{cor:identity-theorem}.
\end{proof}

Despite this, we are still able to perform a large portion of classical complex analysis in o-minimal structures. For example, we can mention some results by Kaiser on Riemann mappings:

Let $D = \left\{z \in \mathbb{C} : \left|z\right| < 1\right\}$ denote the unit disc in $\mathbb{C}$ and let $P$ denote a polygon with vertices $\omega_1, \ldots, \omega_n$ in counter-clockwise order. If the internal angle of $\boundary\mleft(P\mright)$ at each $\omega_i$ is given by $\pi \alpha_i$ where $\alpha_i \in \left(0, 2\right)$, then for $b_1, \ldots, b_n \in \boundary\mleft(D\mright)$, also in counter-clockwise order, the unique biholomorphism $f : D \rightarrow P$, whose continuous extension to $\boundary\mleft(D\mright)$ maps each $b_i$ to $\omega_i$, is given by the Schwarz-Christoffel map
\[f\mleft(z\mright) = c_0 \int_0^z \prod_{j = 1}^n \left(\xi - b_j\right)^{\alpha_j - 1} \mathrm{d}\xi + c_1.\]

To begin, if we bound away from any $b_j$, then the integrand will be holomorphic, thus the integral will be analytic and hence definable in $\mathbb{R}_{\textnormal{an}}$ on the bounded domain $D$. Kaiser showed additionally in \cite{kaiser-schwarz-christoffel-map} that:

\begin{proposition}
	For each $1 \leq j \leq n$,
	\begin{packeditemize}
		\item if $\alpha_j \in \mathbb{Q}$, then $f$ above is definable in $\mathbb{R}_{\textnormal{an}}$ near $b_j$,
		\item if $\alpha_j \not\in \mathbb{Q}$, then $f$ above is definable in $\mathbb{R}^{\mathbb{R}}_{\textnormal{an}}$ near $b_j$, which\footnote{This is a reduct of $\mathbb{R}_{\textnormal{an,exp}}$ because with $\exp$ we also have $\log$, and we can write $x^\alpha = \exp\mleft(\alpha \log\mleft(x\mright)\mright)$.} includes a function $x \mapsto x^\alpha$ for each $\alpha \in \mathbb{R}$.
	\end{packeditemize}
\end{proposition}

\begin{proof}
	The function $\prod_{i \neq j} \left(\xi - b_i\right)^{\alpha_i - 1}$ is holomorphic near $b_j$, so we can write it as a convergent power series
	\[\prod_{i \neq j} \left(\xi - b_i\right)^{\alpha_i - 1} = \sum_{m = 0}^\infty a_m \left(\xi - b_j\right)^m.\]
	Then
	\begin{align*}
		\int_0^z \prod_{i = 1}^n \left(\xi - b_i\right)^{\alpha_i - 1} \mathrm{d}\xi & = \int_0^z \sum_{m = 0}^\infty a_m \left(\xi - b_j\right)^{m + \alpha_j - 1} \mathrm{d}\xi                 \\
		                                                                             & = \left(\xi - b_j\right)^{\alpha_j} \sum_{m = 0}^\infty \frac{a_m}{m + \alpha_j} \left(\xi - b_j\right)^m,
	\end{align*}
	where the summation is analytic near $b_j$. The function is thus definable in $\mathbb{R}_{\textnormal{an}}$ if $\alpha_j \in \mathbb{Q}$, and definable in $\mathbb{R}^{\mathbb{R}}_{\textnormal{an}}$ otherwise.
\end{proof}

Thus, a Riemann mapping $f : D \rightarrow P$ must be definable in $\mathbb{R}^{\mathbb{R}}_{\textnormal{an}}$ at least. As Theorem 3.3 in \cite{kaiser-riemann-mappings}, Kaiser is also able to construct some o-minimal structure $\mathbb{R}_{\mathcal{Q}}$ and prove in general that:

\begin{theorem}
	Let $\Omega \subseteq \mathbb{C}$ be a bounded, semianalytic and simply connected domain, such that at each singular boundary point $x \in \boundary\mleft(\Omega\mright)$, the angle made by $\boundary\mleft(\Omega\mright)$ at $x$ is an irrational multiple of $\pi$, then a Riemann mapping $f : \Omega \rightarrow D$ is definable in $\mathbb{R}_{\mathcal{Q}}$.
\end{theorem}

\section{Conclusion}

In classical real and complex analysis, one very often looks at polynomials, rational functions and other naturally useful entities with tame topological properties. This essay examined o-minimality from model theory as a possible explanation. Following van den Dries, Peterzil and Starchenko, I explained cell decomposition on o-minimal structures with its exciting implications, and discussed the large portion of complex analysis one can recover and possibly simplify when only definable sets and functions are concerned.

We are able to extend such technique to many interesting complex functions by considering the o-minimal definability of the exponential and other analytic functions on bounded domains. This demonstrates the dense occurrences of o-minimal (hence topologically tame) classes of entities in classical analysis and the potential significance of o-minimal theories themselves, even for someone without much model-theoretical interest.

\newpage
\begin{thebibliography}{99}
	\addcontentsline{toc}{section}{References}

	\bibitem{real-subanalytic-sets} Denef, J., \& van den Dries, L.\ (1988). P-adic and Real Subanalytic Sets. \textit{Annals of Mathematics}, 128(1), 79-138.

	\bibitem{polynomial-growth} van den Dries, L.\ (1986). A generalization of the Tarski-Seidenberg theorem, and some nondefinability results. \textit{Bulletin of the American Mathematical Society}, 15(2), 189-193.

	\bibitem{tame-topology} van den Dries, L.\ (1998). \textit{Tame topology and o-minimal structures} (London Mathematical Society lecture note series ; 248). Cambridge.

	\bibitem{r-an-exp} van den Dries, L., \& Miller, C.\ (1994). On the real exponential field with restricted analytic functions. \textit{Israel Journal of Mathematics}, 85(1-3), 19–56.

	\bibitem{sketch-of-a-programme} Grothendieck, A.\ (1997). Esquisse d'un Programme. Translation in Schneps, L., \& Lochak, P.\ (Eds.), \textit{Geometric Galois actions.\ 1, Around Grothendieck's Esquisse d'un programme} (London Mathematical Society lecture note series ; 242), 243-284. Cambridge.

	\bibitem{kaiser-schwarz-christoffel-map} Kaiser, T.\ (2006). Definability results for the Poisson equation. \textit{Advances in Geometry}, 6(4).

	\bibitem{kaiser-riemann-mappings} Kaiser, T.\ (2009). The Riemann mapping theorem for semianalytic domains and o‐minimality. \textit{Proceedings of the London Mathematical Society}, 98(2), 427-444.

	\bibitem{pfaffian-systems} Khovanskii, A.\ G.\ (1980). On a class of systems of transcendental equations. \textit{Dokl.\ Akad.\ Nauk SSSR}, 255:4, 804-807. Translation in \textit{Soviet Math.\ Dokl.} Vol. 22 (1980), No. 3, 762-765.

	\bibitem{o-minimality-elementary-equivalence} Knight, J.\ F., Pillay, A., \& Steinhorn, C.\ (1986). Definable sets in ordered structures. II. \textit{Transactions of the American Mathematical Society}, 295(2), 593-605.

	\bibitem{liouvilles-theorem} Kovacsics, P.\ (2017). A proof of Liouville's theorem via o-minimality. \texttt{arXiv:1712.07593 [math.LO]}.

	\bibitem{algebra-lang} Lang, S.\ (2002). \textit{Algebra} (Rev.\ 3rd ed., Graduate texts in mathematics ; 211). New York ; London: Springer.

	\bibitem{subanalytic-sets-lojasiewicz} \L{}ojasiewicz, S.\ (1964). \textit{Ensembles semi-analytiques}.  Institut des Hautes \'Etudes Scientifiques, Bures-sur-Yvette.

	\bibitem{field-model-theory} Marker, D., Messmer, M., \& Pillay, A.\ (1996). \textit{Model theory of fields} (Lecture notes in logic ; 5). Berlin ; New York.

	\bibitem{differentiability-theory} Peterzil, Y., \& Starchenko, S.\ (2001). Expansions of algebraically closed fields in o-minimal structures. \textit{Selecta Mathematica}, 7(3), 409-445.

	\bibitem{complex-like-analysis} Peterzil, Y., \& Starchenko, S.\ (2003). ``Complex-like'' analysis in o-minimal structures. \textit{Proceedings of the RAAG Summer school Lisbon 2003, o-minimal structures}, Network RAAG 2005, 77–103.

	\bibitem{tame-complex-analysis-summary} Peterzil, Y., \& Starchenko, S.\ (2010). Tame Complex Analysis and o-minimality. \textit{Proceedings of the International Congress of Mathematicians 2010}, ICM 2010.

	\bibitem{definable-compactness} Peterzil, Y., \& Steinhorn, C.\ (1999). Definable Compactness and Definable Subgroups of o‐Minimal Groups. \textit{Journal of the London Mathematical Society}, 59(3), 769-786.

	\bibitem{pfaffian-closure} Speissegger, P.\ (1999). The Pfaffian closure of an o-minimal structure. \textit{Journal F\"ur Die Reine Und Angewandte Mathematik (Crelles Journal)}, 1999(508), 189-211.

	\bibitem{topological-analysis} Whyburn, G.\ (1958). \textit{Topological analysis} (Princeton mathematical series ; 23). Princeton: Princeton University Press.

	\bibitem{r-exp-model-complete} Wilkie, A.\ J.\ (1996). Model completeness results for expansions of the ordered field of real numbers by restricted Pfaffian functions and the exponential function. \textit{Journal of the American Mathematical Society}, 9(4), 1051-1094.

	\bibitem{woerheide-curve-theorem} Woerheide, A., \& van den Dries, L.\ (1996). \textit{O-minimal Homology}, ProQuest Dissertations and Theses.

\end{thebibliography}

\end{document}