%?deps asl.cls

% workaround for asl.cls compile error
\let\negmedspace\undefined
\let\negthickspace\undefined

\documentclass[bsl,meeting]{asl}

\AbstractsOn

\pagestyle{plain}

\def\urladdr#1{\endgraf\noindent{\it URL Address}: {\tt #1}.}
\newcommand{\NP}{}

\begin{document}
\thispagestyle{empty}

\NP
\absauth{Shuwei Wang}
\meettitle{Realisability semantics and choice principles for Weaver's third-order conceptual mathematics}
\affil{School of Mathematics, University of Leeds, Leeds LS2 9JT, UK}
\meetemail{mmsw@leeds.ac.uk}

In \cite{weaver09-cm}, Nik Weaver introduced a formal semi-intuitionistic third-order arithmetic as an alternative foundation for mainstream mathematics. This system is claimed to follow his philosophical vision of \emph{mathematical conceptualism}, which is a variant of predicativitism and admits all and only countable procedurals built up from below as concrete mathematical objects. Supplemented by a layer of class-like third-order objects, Weaver maintains that his arithmetic can be utilised for much of modern mathematics that mainly concerns uncountable structure with a tame, countable description, such as separable metric spaces and second-countable topological spaces.

In section 2.3 of his paper, Weaver proposed a direction to further extend this theory to accommodate for uncountable choice principles that are commonly used in mainstream mathematics, by assuming a global well-ordering structure on the second-order objects. However, it remains suspectable how such an extension may stay loyal to his predicative philosophical ideals. In this talk, we shall present a realisability interpretation for Weaver's mathematics through hyperarithmetic functions in a classical second-order meta-theory. By doing this, we compare the proof-theoretic strength of Weaver's base theory and possible extensions to well-known fragments of second-order arithmetic --- ranging in strength from $\Sigma^1_1$-choice to Bar induction --- and use this as a benchmark to discuss how well they fit into the predicativity picture.

Detailed proofs of many results in this talk are available in the speaker's arXiv preprint \cite{wang-sw25-cm}, with some more recent improvements.

\begin{thebibliography}{10}

  \bibitem{weaver09-cm}
  {\scshape Nik Weaver},
  {\itshape Axiomatizing mathematical conceptualism in third order arithmetic}, 2009.
  Preprint available at \texttt{arXiv:0905.1675 [math.HO]}.

  \bibitem{wang-sw25-cm}
  {\scshape Shuwei Wang},
  {\itshape An ordinal analysis of CM and its extensions}, 2025.
  Preprint available at \texttt{arXiv:2501.12631 [math.LO]}.

\end{thebibliography}

\vspace*{-0.5\baselineskip}

\end{document}
