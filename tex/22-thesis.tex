\documentclass[12pt]{article}

\usepackage{aliascnt,amsmath,amssymb,amsthm,calc,censor,mathtools,mleftright,parskip,tensor}
\usepackage[hidelinks]{hyperref}

\usepackage[a4paper,left=1.45in,right=1.45in,top=1.25in,bottom=1.25in,footskip=0.6in]{geometry}

\newenvironment{packeditemize}{%
	\begin{itemize}
	\setlength{\parskip}{2pt}
}{\end{itemize}}

\renewcommand{\thefootnote}{\arabic{footnote}}

\makeatletter
\newtheorem*{rep@theorem}{\rep@title}
\newcommand{\newreptheorem}[2]{%
\newenvironment{rep#1}[1]{%
 \def\rep@title{#2 \ref{##1}}%
 \begin{rep@theorem}
}{\end{rep@theorem}}}

\renewcommand{\maketitle}{%
	\begin{titlepage}
		\vspace*{1.0cm}
		\begin{center}
			{\large \censor*{6} Philosophy Thesis}\\[.1cm]
			{submitted for the degree of Master in Mathematics and Philosophy}\\[.04cm]
			{University of Oxford}\\[.3cm]
			\rule{\textwidth}{1.5pt}\\[0cm]
			{\Large \bfseries \@title \par \ }\\[-.8cm]
			\rule{\textwidth}{1.5pt}\\[1.2cm]
			{\large
				\lineskip .5em%
				\begin{tabular}[t]{c}%
					\@author
				\end{tabular}
				\par \ }\\[-.3cm]
			{\large \@date}
			\vfill
			\vfill
			\vfill
			{\small Final word count: 17563}\\[.1cm]
			{\footnotesize Including words outside text (captions, etc.), excluding mathematical formulae}\\[.04cm]
			{\footnotesize Word count generated by TeXcount version 3.2}
			\vfill
		\end{center}
	\end{titlepage}
}
\makeatother

\newcommand{\newaliastheorem}[2]{%
  \newaliascnt{#1}{theorem}
  \newtheorem{#1}[#1]{#2}
	\newreptheorem{#1}{#2}
  \aliascntresetthe{#1}
  \expandafter\def\csname #1autorefname\endcsname{#2}
}

\theoremstyle{plain}
\newtheorem{theorem}{Theorem}[section]
\newreptheorem{theorem}{Theorem}
\newaliastheorem{proposition}{Proposition}
\newaliastheorem{corollary}{Corollary}
\newaliastheorem{lemma}{Lemma}
\newtheorem*{fact*}{Fact}
\newtheorem*{hypothesis*}{Hypothesis}

\theoremstyle{definition}
\newaliastheorem{definition}{Definition}

\newcommand{\PA}{\mathrm{PA}}

\newcommand{\ZFC}{\mathrm{ZFC}}
\newcommand{\ZF}{\mathrm{ZF}}
\newcommand{\NBG}{\mathrm{NBG}}

\newcommand{\On}{\mathrm{On}}
\newcommand{\Card}{\mathrm{Card}}

\newcommand{\CH}{\mathrm{CH}}
\newcommand{\GCH}{\mathrm{GCH}}

\newcommand{\IMH}{\mathrm{IMH}}
\newcommand{\CIMH}{\mathrm{CIMH}}

\newcommand{\PCWC}{\mathrm{PCWC}}
\newcommand{\PD}{\mathrm{PD}}
\newcommand{\AD}{\mathrm{AD}}

\DeclareMathOperator{\Def}{Def}
\DeclareMathOperator{\Formula}{Form}
\DeclareMathOperator{\Con}{Con}
\DeclareMathOperator{\Ind}{Ind}

\DeclareMathOperator{\rank}{rank}
\DeclareMathOperator{\cf}{cf}
\DeclareMathOperator{\crit}{crit}

\title{A Carnapian resolution for pluralism \\ about the large cardinal hierarchy}
\author{MMathPhil Candidate \censor*{7}}
\date{Year 2021/22}

\begin{document}

\pagestyle{empty}
\raggedbottom

\maketitle

\begin{abstract}
	\noindent Pluralism for the mathematical ontology has been a popular option since G\"odel's discovery of incompleteness in formal systems, especially in the philosophy of set theory with many disputable axioms. This thesis aims to examine the large cardinal axioms that generate a linear interpretability hierarchy, and critically assess related arguments for and against pluralism, primarily referring to Koellner's discussions of reflection principles and $\Omega$-completeness and Woodin's (sequence of) work on the axioms of determinacy. It shall be argued that the axioms have implications in various areas of mathematics and philosophical justification for the axioms is received differently across the branches. The thesis shall propose that this debate between pluralists and anti-pluralists can be dismissed by taking a Carnapian perspective.
\end{abstract}

\vspace*{1.5cm}
\renewcommand{\abstractname}{Declaration of Authorship}
\begin{abstract}
	\noindent I hereby declare that this thesis is the candidate's own work except where otherwise indicated, and has not been previously submitted, either wholly or partially, for this Honour School or qualification, or for another Honour School or qualification of this University, or for a qualification at any other institution. This thesis is completed in accordance with the University's examination regulations for the Honour School of Mathematics and Philosophy, on a topic approved by the Director of Undergraduate Studies of the Faculty of Philosophy. The presence in this thesis of any material I have quoted or paraphrased from other sources has been clearly cited in text and included in the selected list of references at the end of the document.
\end{abstract}

\newpage
\tableofcontents
\newpage

\pagestyle{plain}
\pagenumbering{arabic}
\flushbottom

\section{Introduction}

After Zermelo--Fraenkel set theory was first proposed around the beginning of the twentieth century, there have been multiple attempts to modify or strengthen the system through alternative extensions, the most well-known examples being the axiom of choice and the continuum hypothesis. Differences between such choices of axiomatisation ultimately rose to the centre of the study of set theory due to the independence results by Kurt G\"odel, Paul Cohen and subsequent mathematicians. Ever since, ontological pluralism has been a position that one must address in the philosophical discussions of set theory and even the whole of mathematics.

This thesis will, specifically, look at the collection of large cardinal axioms asserting the existence of inaccessible cardinals, Mahlo cardinals, measurable cardinals, \emph{et cetera}, on the foundation of the standard Zermelo-Fraenkel axioms with Choice ($\ZFC$). I will assess why there is strong evidence for a pluralist ontology with respect to such entities in a traditional realist perspective and how such a position endangers even the most intuitive concepts in standard mathematics, such as the real numbers. I will argue that a Carnapian viewpoint, as advocated in his famous paper \emph{Empiricism, Semantics and Ontology} in 1950 (\cite{carnap-eso}, abbreviated as \emph{ESO} henceforth), provides a reasonable middle-ground for the pluralists and anti-pluralists and allows a philosophical analysis of contemporary mathematical practices.

In this thesis, I will first elaborate on Carnapian positivism in section 2 as a philosophical background and argue that this is a viable philosophy of mathematics, despite the major attacks on the theory in the past century. In section 3, I will introduce the large cardinal axioms and the disputes surrounding their mathematical implications. Some set-theoretic attempts at solving the problems in favour of anti-pluralism will be examined in section 4, where I will comment on how these pragmatic arguments are best understood under pluralist assumptions. Section 5 will explain why the set-theoretic disputes have consequences in the more intuitive areas of real analysis and combinatorics and establish more support for a resolution between the evidence for pluralism in set theory and that for anti-pluralism in the rest of mathematics. Finally, I will contrast and argue for the Carnapian position in section 6, in comparison to perspectives of contemporary philosophers of mathematics such as Penelope Maddy.

\section{A Carnapian philosophy of mathematics}

\subsection{Linguistic frameworks}

Here, by a Carnapian philosophy, I refer to Carnap's later writings after his \emph{Aufbau}, most importantly his book \emph{The Logical Syntax of Language} (\cite{logical-syntax}, abbreviated as \emph{Logical Syntax} henceforth) and the essay \textit{ESO} (\cite{carnap-eso}). During this stage of his career, Carnap founded his meta-ontology --- in fact, the rejection thereof --- upon the concept of linguistic frameworks. As Carnap explained in section 2 of \emph{ESO}:

\begin{quote}
	if someone wishes to speak in his language about a new kind of entities, he has to introduce a system of new ways of speaking, subject to new rules; we shall call this procedure the construction of a \emph{framework} for the new entities.
\end{quote}

When assessing formal languages in \emph{Logical Syntax}, Carnap gave a clearer definition of the ``logical syntax of a language'' as ``the systematic statement of the formal rules which govern it together with the development of the consequences which follow from these rules''. In the terminology of modern logic, Carnap's formal frameworks specify the grammar for the formation of formulae and both semantic and syntactic rules governing the inferential relations between them. However, it is worth noting that, as an empiricist, Carnap intended for his concept to work with not only logic and mathematics, but also empirical scenarios such as a framework of the ``spatio-temporally ordered system of observable things and events'' in the grammar of an everyday language.

Carnap adopted the concept of linguistic frameworks mainly to justify the position of an empiricist who wishes to talk about abstract entities such as in mathematics. Carnap attempted to explain that such a person is neither guilty of committing to a Platonist ontology nor required to reduce his abstract references to mere calculus in a nominalist way. Carnap's argument based on his introduction of the internal/external distinction:

As he suggested, claims like ``there are numbers'' --- and essentially every other metaphysical claim --- are inherently \emph{relativised} on the perspective of linguistic frameworks. There is an internal understanding of this statement within each specific framework, which is trivially true so long as the rules of the framework defines both the concept ``number'' and the constants that it applies to, such as ``$5$''. However, to evaluate such a statement objectively, external to any linguistic framework, is impossible, as there lacks a clear cognitive interpretation of the external claim within any scientific language without trivialising it. As Carnap argued, one can at best raise the practical question of whether to accept a linguistic framework equipped with the aforementioned account of numbers and this is no longer a metaphysical problem where one debates for an absolute, correct position.

Carnap thus simply dismissed the debate between a Platonist and a nominalist ontology and granted that an empiricist can adopt a language to speak of abstract entities like numbers for its effectiveness and efficiency, for example, in scientific endeavours, without providing metaphysical justifications for his ontological commitments. One who does so is free to assert that numbers and other abstract entities exist so long as the statement is only understood internally to his chosen language.

\subsection{Quine's attacks on empiricism}

Admittedly, Carnap's positivist project is often seen as defeated by various criticisms that ensued, most famously Quine's essay \emph{Two Dogmas of Empiricism} (\cite{two-dogmas}). In the article, Quine accused empiricists of making the two mistakes of both drawing a non-existent line between analytic and synthetic truths and sweepingly reducing every meaningful sentence to immediate experience.

Here, I shall focus on the former problem, namely the analytic/synthetic distinction present throughout the Carnapian constructions. The reason is that, as Quine also acknowledged, later work of Carnap has diverted from the problematic radical reductionist position since his unsuccessful attempt in the \emph{Aufbau}, and Quine's dissatisfaction towards the dogma of reductionism that survived ``in a subtler and more tenuous form'' is inherently the same as the first criticism that Quine raised --- that the truths of a theory ought to depend holistically on both linguistic and factual features, as he believed, where the identification of single statements as true either by specific empirical data or entirely by syntactic analysis is unintelligible.

Quine's view is founded upon his observation of the indeterminacy of meaning. Specifically, a sentence like ``no bachelor is married'' is traditionally identified as analytic, or true by virtue of meaning, because it is possible to replace ``bachelor'' by its synonym ``unmarried man'', such that the resultant is true under any reinterpretation of its non-logical terms. However, Quine pointed out that the statement ```bachelor' is synonymous to `unmarried man''' is simply a lexicographer's report of some observed, pre-existent usage in the English language and itself requires justification. Therefore, if an empiricist would like to support his substitution of ``bachelor'' by ``unmarried man'', or that the pair is ``interchangeable \emph{salva veritate}'', without appealing to empirical observations, he must present some form of necessity that the term coincides with its definiens, which, as Quine commented, is a circular argument invoking a notion of analyticity in the language before defining it.

The problem can seemingly be solved by considering only artificial languages, where statements like ``all bachelors are unmarried men'' are explicitly true by semantic stipulations of the language, or legislative definitions as Quine called them in his 1960 commentary \cite{quine-to-carnap}. Legislative definitions are less problematic in motivating analytic truths, as compared to the example above in a natural language. Indeed, artificial languages have been Carnap's primary interest all along, both in his \emph{Aufbau} and later in his \emph{Logical Syntax}. However, Carnap never succeeded in providing a general notion of analyticity for his artificial languages and instead only proposed individual criteria --- for which I shall call ``analytic-for-$L$'', following Quine --- to identify analytic sentences in a specific language $L$ based on semantic rules. Since Carnap had different semantic rules in mind for, for example, his Language I and Language II in \emph{Logical Syntax}, his account does not constitute a philosophical analysis of the concept ``sentence $S$ is analytic for language $L$'' for artificial languages in general.

This problem is more clearly raised by Koellner in \cite{koellner-pluralism}, in the section ``The Argument from Free Parameters'': specifically in \emph{Logical Syntax}, Carnap defined analyticity through two different approaches. The first approach is his treatment of analyticity in Language II in \S 34d where he operated on the cases of logical and descriptive sentences separately; the second approach is his discussion in \S 52 of a general language equipped with a division between logical and physical rules of syntactic transformation. Either way, Carnap's analyticity is formulated through the restriction of terminology and rules to a pre-defined subset, avoiding the use of descriptive or physical concepts that, for Carnap, are subject to empirical evaluation. As Koellner argued, such constructions depend on some arbitrary distinctions in the linguistic features that remain unjustified. In an extreme case, if Carnap had taken names for physical entities to also be logical symbols or assigned natural laws as logical transformation rules, he would come to the absurd conclusion that all truths in the language are analytic.

Of course, it must be mentioned that Carnap's stipulations in \emph{Logical Syntax} are entirely formal and he himself has never claimed to take this as a philosophical analysis of the intuitive concept of analyticity in general. Therefore, Carnap shall not be guilty of the criticism that his definitions are simply artificial. However, it can still be problematic, as Quine suggested in section X of his 1960 article \cite{quine-to-carnap}, for this notion to constitute a part of Carnap's own project in the way that

\begin{quote}
	[it] is only by assuming the cleavage between analytic and synthetic truths that he is able e.g.\ to declare the problem of universals to be a matter not of theory but of linguistic decision.
\end{quote}

Namely, as Carnap suggested in \S 51 of \emph{Logical Syntax}, ``if P-rules [i.e.\ physical rules] are stated, we may frequently be placed in the position of having to alter the language'', so he acknowledged that the choice of physical rules is a matter of not only practical but also empirical --- that is, theoretical --- evidence. However, if the distinction between logical and physical rules is indeed arbitrary as Koellner suggested, he will be in no position to maintain that the acceptance of logical rules is a purely linguistic decision and not in danger of revision upon new empirical evidence. This is precisely the worry expressed in Quine's holistic view of knowledge and one of the problems at root why Carnap's and Quine's theories cannot cohere.

\subsection{As a philosophy of mathematics}

While Carnap's primary interest, as he described in Chapter 1 of his 1935 book \emph{Philosophy and Logical Syntax} (\cite{carnap-pls}), was to adopt and potentially expand one definite linguistic framework to unify all theoretical knowledge under logic and the empirical sciences, I believe that his approach has some overlooked potential in explaining a philosophy of abstract entities with an implicitly plural ontology, fitting into the contemporary scene of mathematical research. Namely, a Carnapian perspective allows one to acknowledge the existence of multiple mathematical structures --- in a model-theoretic sense, for example, all demonstrating the features of classical real numbers --- in the form of distinctive frameworks of evaluation, while still being able to speak of the \emph{unique} collection of ``real numbers'' as an internal notion in everyday research. The theory further allows an artificial choice between the plurality of theories, based on pragmatic considerations alone. This is a much needed resolution for the philosophy of mathematics given the independence of the axioms of determinacy that I will elaborate further in \autoref{subsec:determinacy}.

As for Quine, I think that his worries are quite reasonable for Carnap's theory as a foundation of empirical sciences. However, Quine, as a naturalist who aimed to establish the contemporary scientific practices as the sole justification for itself, failed to respect and account for the difference between mathematical and scientific practices. Essentially, I wish to point out that the analytic/synthetic distinction does exist in a modern scientific view of knowledge as the boundary between mathematical and empirical facts. This is the boundary where, when a mathematical model is applied to certain measurements and produces incorrect predictions, a scientist would only reasonably question the empirical claim that the chosen model applies to the scenario in concern, while it is unimaginable to state that the mathematical model itself is simply ``false''. For example, we can say that the discovery of general relativity forces one to adopt a non-Euclidean view of the physical space, but it still does not constitute evidence to say that Euclidean geometry is false. In Carnap's words, the rejection of logical assumptions and rules external to any framework is non-cognitive.

Quine's central assumption about mathematics, as clearly seen from his 1960 essay \cite{quine-to-carnap}, is that much of it is --- potentially --- concerning an interpreted theory like that of the natural sciences, where one has certain entities of discourse in mind first and proceeds to discover properties of them. However, the modern mathematical scene is one where pluralism has been a pervasive danger across multiple areas, in the sense that many natural questions remain uncertain from a standard description of the mathematical entities involved, such as sets. Given this issue of unsolvability in the formal languages, such an interpretation is ambiguous even in the most intuitive areas of today's mathematics if one considers implications of set theory.

On the other hand, I believe that a Carnapian position is preferrable in an entirely uninterpreted language of mathematics, where all linguistic rules as specified priorly in the form of mathematical axiomatisation. A mathematical fact is considered analytic in some linguistic framework in Carnap's sense by being the consequence of the null set or a set of axioms that the framework comes equipped with. By refusing to also interpret any empirical entities, a linguistic framework residing entirely in the domain of mathematics escapes the Quinean criticism that its corpus of artificially specified ``analytic truths'' may be subjected to any form of revision upon empirical evidence, and thus is true, as Carnap desired, in virtue of the linguistic and pragmatic decisions only.

\subsubsection*{G\"odel's Criticisms}

It is worth mentioning that, when understood as positing a plurality of mathematical realities through entirely formal means, other concerns for Carnap's philosophy may be raised. As G\"odel claimed\footnote{Further analysis of G\"odel's criticisms can be seen in Goldfarb's exposition \cite{goldfarb-positivism}.} in his unpublished paper \cite{godel-positivism}, an interpretation of mathematics as syntax may be accepted as support for conventionalism only if it admits finitary syntax and rules of inference as well as a consistency proof. A majority of the constructions in Carnap's \emph{Logical Syntax} fail to meet such requirements: namely, his Language II is neither finitary, due to his explicit use of the semantic evaluation of unbounded quantifiers as part of the syntax, nor proved consistent, as he admitted that his analysis ``gives no absolute certainty that contradictions in the object language II cannot arise'' around the end of \S 34.

In fact, G\"odel's seemingly reasonable requirements can be a huge obstacle for any similar interpretations because firstly, by G\"odel's first incompleteness theorem, any finitary, formal account of mathematics cannot be complete --- there will exist a sentence that is ``intuitively true'' but not derivable from the system. Secondly, by G\"odel's second incompleteness theorem, the consistency proof of a formal system cannot exist within the system itself, thus requires a meta-theory formulating sufficient mathematics prior to presenting the account of mathematics in concern, leading towards a infinite regression.

I agree with Goldfarb here that a Carnapian philosophy of mathematics is capable of dismissing G\"odel's attacks as a mere ``dogmatic insistence''. Finitary methods was never seen as a condition to abide by in Carnap's work and this can probably find some justification nowadays in the spreading interest in non-finitary methods in alternative logics such as the language $L_{\omega_1\omega}$ or the $\omega$-rule. Indeed, even if one instead decided to follow G\"odel's criteria, reject Carnap's indefinite c-rules and specify a syntactic system in the modern, proof-theoretic way only --- this may be slightly inconvenient for an empiricist like Carnap, who wished to adopt one language to speak of all of mathematics and the empirical sciences --- incompleteness is never a problem for a pluralist perspective of mathematics because, as one will see in the subsequent sections, being able to switch to and discuss about incompatible expansions generated by an independent sentence in the base theory is a feature instead of a problem for such a philosophy.

Regarding the argument on consistency proofs, Carnap was not too worried, in his comment at the end of \S 34 in \emph{Logical Syntax}, about the fact that, due to G\"odel's researches, a consistency proof as certain as Hilbert desired can potentially be impossible. Ironically, this is a potential problem for his empiricist project because, should he discover that any of his languages are inconsistent and derives contradictory propositions about empirical observations from the null set, the rejection of such a language and its analytic truths would be for theoretical reasons instead of pure linguistic decisions.

However, while it is uninteresting to work with a mathematical theory shown to be contradictory, modern practices of mathematical logic make perfect sense of proofs from a set of axioms not known to be consistent. In set theory, it has been conventional indeed that all it requires is simply a proof of relative consistency from a standard theory like $\ZFC$ --- which is not an absolute justification of consistency at all in Hilbert and G\"odel's sense --- before a newly proposed axiom system is accepted as a well-behaved context to research in. Thus, it is reasonable, as I believe, for a pluralist to maintain that forcing a linguistic framework of mathematics to conform to G\"odel's criteria is mere pragmatic preference and not an absolute requirement at all. In other words, as Carnap's principle of tolerance goes in \emph{ESO}:

\begin{quote}
	Let us be cautious in making assertions and critical in examining them, but tolerant in permitting linguistic forms.
\end{quote}

Therefore, I shall conclude from this section that Carnapian positivism is a reasonable candidate for a philosophy of mathematics despite its inadequacy for justifying empirical sciences. I will elaborate in subsequent sections, by looking at the specific mathematical results related to the large cardinal axioms, that this is indeed a fitting position to take.

\section{New axioms from large cardinals}

\subsection{The large cardinals}

In this essay, all mathematical discussions will be based on the following standard axiomatisation of the Zermelo-Fraenkel set theory with Choice ($\ZFC$) unless otherwise indicated. I follow Jech's textbook \emph{Set Theory} \cite{jech-set-theory} and denote
\begin{align*}
	\ZFC & \coloneqq \text{Extensionality} + \text{Pairing} + \text{Separation} + \text{Union} + \text{PowerSet} \\
	     & + \text{Infinity} + \text{Replacement} + \text{Regularity} + \text{Choice},
\end{align*}
while $\ZF \coloneqq \ZFC - \text{Choice}$. Standard modern model and proof theory will be invoked for the semantic interpretation and syntactic manipulation of first-order --- and higher-order, when necessary --- formulae, although proofs included in this essay will be presented informally. Elementary results, as covered in the first two chapters of Jech's book, will be assumed. On this foundation, I shall first mention a few examples of the large cardinal axioms that will be involved in subsequent discussions, based on expositions in Kanamori's book \emph{The Higher Infinite} (\cite
{higher-infinite}):

\subsubsection*{Inaccessible and Mahlo cardinals}

\begin{definition}[Limit cardinals]
	A cardinal $\kappa \in \Card$ is a \emph{weak limit cardinal} if it is neither a successor cardinal or zero; $\kappa$ is a \emph{strong limit cardinal} if it is a weak limit cardinal and additionally, for any cardinal $\lambda < \kappa$, it follows that $2^\lambda < \kappa$.
\end{definition}

\begin{definition}[Regular cardinals]
	A cardinal $\kappa \in \Card$ is \emph{regular} if it equals to its own cofinality, that is, any unbounded subset $C \subseteq \kappa$ has cardinality $\kappa$.
\end{definition}

\begin{definition}[Inaccessible cardinals]
	A cardinal $\kappa \in \Card$ is \emph{weakly inaccessible} if it is an uncountable, regular, weak limit cardinal; $\kappa$ is \emph{(strongly) inaccessible} if it is an uncountable, regular, strong limit cardinal.
\end{definition}

Obviously, inaccessible cardinals are also weakly inaccessible. Following convention, weakly inaccessible cardinals are the first large cardinals we look at on the hierarchy of large cardinals. This hierarchy then extends, through a few more technical constructions that I will not cover in this essay, such as $\alpha$-inaccessibility, to the Mahlo cardinals:

\begin{definition}[Stationary sets]
	Let $\kappa \in \Card$ be a cardinal. A set $S \subseteq \kappa$ is a \emph{club\footnote{The word ``club'' is an abbreviation for ``closed unbounded''.} in $\kappa$} if $S$ is unbounded and, for any bounded subset $A \subsetneq S$, $\sup A \in S$.

	$S \subseteq \kappa$ is \emph{stationary in $\kappa$} if it intersects every club in $\kappa$.
\end{definition}

\begin{definition}[Mahlo cardinals]
	A cardinal $\kappa \in \Card$ is \emph{Mahlo} if the set
	\[\left\{\lambda \in \kappa : \text{$\lambda$ is an inaccessible cardinal}\right\}\]
	is stationary in $\kappa$.
\end{definition}

\begin{proposition}
	Let $\kappa \in \Card$ be a Mahlo cardinal, then $\kappa$ is inaccessible.
\end{proposition}

\begin{proof}
	Let $R = \left\{\lambda \in \kappa : \text{$\lambda$ is an inaccessible cardinal}\right\}$. Firstly, $\kappa$ is uncountable because $R$ is non-empty.

	To check that $\kappa$ is regular, suppose otherwise that $S \subseteq \kappa$ is unbounded, yet $\left|S\right| < \kappa$. $\left|S\right|$ must be uncountable because otherwise one will be able to write $\kappa$ as a limit of a non-decreasing sequence $s_0 < s_1 < \cdots$ of elements in $S$ and thus
	\[\kappa = \sup \left\{s_n + 1 : n \in \omega\right\}\]
	where each $s_n + 1 \not\in R$. Now, consider the limit points in $\overline{S} \setminus \left|S\right|$, that is,
	\[S' = \left\{\lambda \in \overline{S} : \text{$\lambda > \left|S\right|$, $\lambda = \sup A$ for some $A \subseteq S \setminus \left\{\lambda\right\}$}\right\}\]
	where $\overline{S}$ is the closure of $S$ in the order topology and $\left|\overline{S}\right| = \left|S\right|$. $S'$ is a club in $\kappa$ but contains no regular cardinals, so $S' \cap R = \varnothing$, a contradiction.

	Finally, to check that $\kappa$ is a strong limit, it suffices to check that for any cardinal $\lambda < \kappa$, it follows that $2^\lambda < \kappa$. Here $\kappa$ is a limit ordinal and $R$ is unbounded in $\kappa$, thus there exists an inaccessible cardinal $\delta < \kappa$ such that $\lambda < \delta$. It follows that $2^\lambda < \delta < \kappa$.
\end{proof}

The weakly inaccessible, inaccessible and Mahlo cardinals, along with a few other types of cardinals not covered here, are sometimes classified as the ``small'' large cardinals:

\begin{definition}
	The $\varphi$-cardinals, where $\varphi$ denotes ``weakly inaccessible'', ``inaccessible'', ``Mahlo'' or other conditions, is said to be \emph{small} if, suppose that the formulae ``$\kappa$ is a $\varphi$-cardinal'' is consistent, ``$L \vDash \text{$\kappa$ is a $\varphi$-cardinal}$'' is also consistent, where $L$ is the G\"odel constructible universe.

	Otherwise, the $\varphi$-cardinals are said to be \emph{large}.
\end{definition}

\begin{proposition}
	\label{prop:small-cardinals-absolute-for-inner-model}
	Let $\varphi$ denote a condition ``weakly inaccessible'', ``inaccessible'' or ``Mahlo'' and suppose that $\kappa$ is a $\varphi$-cardinal. For an inner model $M$, that is, a transitive model of $\ZFC$ containing $\On$, it follows that $M \vDash \text{$\kappa$ is a $\varphi$-cardinal}$.

	Specifically, it follows that $L \vDash \text{$\kappa$ is a $\varphi$-cardinal}$.
\end{proposition}

\begin{proof}
	Both ``$\kappa$ is a cardinal'' and ``$\kappa$ is a regular cardinal'' are $\Pi^0_1$-formulae. Now, suppose that $\kappa$ is also a weak limit, yet $M \vDash \exists \lambda \in \Card \ \kappa = \lambda^+$. Then $\left|\lambda\right| < \kappa$ and thus $\left|\lambda\right|^+ < \kappa$. Denote $\delta = \left|\lambda\right|^+$ and it follows that
	\[M \vDash \lambda < \delta < \kappa,\]
	contradicting $M \vDash \lambda^+ = \kappa$. Thus, if $\kappa$ is a weakly inaccessible cardinal, then
	\[M \vDash \text{$\kappa$ is a weakly inaccessible cardinal}.\]

	When $\kappa$ is inaccessible, it suffices to show that also $M \vDash \text{$\kappa$ is a strong limit}$. For any $\lambda \in \kappa$, obviously $\mathcal{P}^M\mleft(\lambda\mright) \subseteq \mathcal{P}\mleft(\lambda\mright)$, i.e.\ $\left(2^{\left|\lambda\right|}\right)^M \leq 2^{\left|\lambda\right|} < \kappa$. Thus indeed $\kappa$ is a strong limit in $M$.

	Finally, for a Mahlo cardinal, simply notice that
	\begin{align*}
		 & \left\{\lambda \in \kappa : \text{$\lambda$ is an inaccessible cardinal}\right\}                                 \\
		 & \qquad \subseteq \left\{\lambda \in \kappa : \left(\text{$\lambda$ is an inaccessible cardinal}\right)^M\right\}
	\end{align*}
	by above and ``$S$ is stationary in $\kappa$'' is a $\Pi^0_1$-formulae.
\end{proof}

It follows as an immediate corollary that the weakly inaccessible, inaccessible and Mahlo cardinals are all small.

\subsubsection*{``Large'' large cardinals}

I will also give the example of the simplest ``large'' large cardinal in the introductory Chapter 1 in Kanamori's book \cite{higher-infinite}:

\begin{definition}[Ultrafilters]
	Let $x$ be a set. A \emph{filter} $F$ on $x$ is a subset of $\mathcal{P}\mleft(x\mright)$ satisfying the following properties:
	\begin{itemize}
		\item	$\varnothing \not\in F$;
		\item for $u, v \in F$, also $u \cap v \in F$;
		\item for $u \in F$ and some $v \subseteq x$ such that $u \subseteq v$, also $v \in F$.
	\end{itemize}

	An \emph{ultrafilter} $U$ on $x$ is a filter on $x$ such that, for any $y \subseteq x$, either $y \in U$ or $x \setminus y \in U$.
\end{definition}

\begin{definition}
	An ultrafilter $U$ on $x$ is \emph{principal} if there exists some $a \in x$ such that, for any subset $y \subseteq x$, $y \in U$ if and only if $a \in y$.

	Otherwise, $U$ is \emph{non-principal}.
\end{definition}

\begin{definition}
	Let $U$ be an ultrafilter on $x$. For a cardinal $\lambda \in \Card$, $U$ is \emph{$\lambda$-complete} if, for any subset $S \subseteq U$ such that $\left|S\right| < \lambda$, $\bigcap S \in U$.
\end{definition}

\begin{definition}[Measurable cardinals]
	A cardinal $\kappa \in \Card$ is \emph{measurable} if $\kappa$ is uncountable and there is a $\kappa$-complete non-principal ultrafilter on $\kappa$.
\end{definition}

The measurable cardinals first arose from the study of measures:

\begin{definition}[Measures]
	Let $x$ be a set. A \emph{(non-trivial) measure on $x$} is a function $\mu : \mathcal{P}\mleft(x\mright) \rightarrow \left[0, 1\right]$ such that
	\begin{itemize}
		\item $\mu\mleft(x\mright) = 1$;
		\item $\mu\mleft(\left\{a\right\}\mright) = 0$ for each $a \in x$;
		\item for pairwise disjoint sets $\left\{y_n : n \in \omega\right\} \subseteq \mathcal{P}\mleft(x\mright)$,
		      \[\mu\mleft(\bigcup_{n \in \omega} y_n\mright) = \sum_{n \in \omega} \mu\mleft(y_n\mright).\]
	\end{itemize}

	For a cardinal $\lambda$, $\mu$ is additionally \emph{$\lambda$-additive} if, for any $\gamma \in \lambda$ and pairwise disjoint sets $\left\{y_\alpha : \alpha \in \gamma\right\} \subseteq \mathcal{P}\mleft(x\mright)$,
	\[\mu\mleft(\bigcup_{\alpha \in \gamma} y_\alpha\mright) = \sum_{\alpha \in \gamma} \mu\mleft(y_\alpha\mright).\]
\end{definition}

Here, $\kappa$-complete non-principal ultrafilters correspond to $\left\{0, 1\right\}$-valued\footnote{$\left\{0, 1\right\}$-valued $\kappa$-additive measures further represent real-valued measures with atoms, that is, where there exists some $y \subseteq x$ such that the measure $\mu\mleft(y\mright) > 0$ and, for any subset $z \subseteq y$, either $\mu\mleft(z\mright) = 0$ or $\mu\mleft(z\mright) = \mu\mleft(y\mright)$. These measures are the more interesting ones compared to their atomless counterparts in the study of large cardinals. Further discussions can be found in detail in section 27 of Jech's book \cite{jech-set-theory}.} $\kappa$-additive measures, where, for some such ultrafilter $U$, the associated measure is given by
\[\mu_U\mleft(y\mright) = \left\{\begin{aligned}
		 & 1 &  & \text{if $y \in U$},    \\
		 & 0 &  & \text{if $y \not\in U$}
	\end{aligned}\right.\]
for each $y \subseteq x$. Hence this large cardinal condition is named ``measurable''.

In order to show that measurable cardinals are ``large'', I will mention an equivalent way of characterising such cardinals through elementary embeddings:

\begin{definition}[Elementary embeddings]
	Let $M$ be a transitive class of sets and $j : V \rightarrow M$ be a class function. $j$ is an \emph{elementary embedding}, denoted $j : V \prec M$, if, for any (first-order) formula $\varphi\mleft(v_1, \ldots, v_n\mright)$ and any sets $x_1, \ldots, x_n$,
	\[V \vDash \varphi\mleft[x_1, \ldots, x_n\mright] \quad \text{if and only if} \quad M \vDash \varphi\mleft[j\mleft(x_1\mright), \ldots, j\mleft(x_n\mright)\mright].\]
\end{definition}

\begin{lemma}
	\label{lem:elementary-embedding-preserve-ordinals}
	Let $j : V \prec M$ be an elementary embedding. For any ordinal $\kappa \in \On$, $j\mleft(\kappa\mright)$ is also an ordinal and $j\mleft(\kappa\mright) \geq \kappa$.
\end{lemma}

\begin{proof}
	$\kappa \in \On$ is a first-order assertion, thus it follows from the elementarity of $j$ that $j\mleft(\kappa\mright) \in \On$. It also follows from the elementarity that $j\mleft(\varnothing\mright) = \varnothing$ and, for ordinals $\lambda < \kappa$, $j\mleft(\lambda\mright) < j\mleft(\kappa\mright)$. Therefore, $j\mleft(\kappa\mright) \geq \kappa$ for any ordinal $\kappa \in \On$ by transfinite induction.
\end{proof}

\begin{definition}
	Let $j : V \prec M$ be an elementary embedding. Suppose that $j$ is not identity on $\On$, then the \emph{critical point} of $j$ is
	\[\crit\mleft(j\mright) = \min\mleft\{\kappa \in \On : j\mleft(\kappa\mright) \neq \kappa\mright\}.\]
\end{definition}

The key characterisation I wish to prove is that:

\begin{theorem}
	\label{thm:measurable-cardinals-are-critical-points}
	Some ordinal $\alpha \in \On$ is a measurable cardinal if and only if it is the critical point of some elementary embedding $j : V \prec M$.
\end{theorem}

\begin{proof}
	The backward direction is straightforward. First notice that $j$ does not move the finite ordinals and $\omega$, so $\crit\mleft(j\mright) > \omega$. Let $U$ be a subset of $\mathcal{P}\mleft(\crit\mleft(j\mright)\mright)$ given by
	\[x \in U \quad \text{if and only if} \quad \text{$x \subseteq \crit\mleft(j\mright)$ and $\crit\mleft(j\mright) \in j\mleft(x\mright)$}.\]
	I shall show that $U$ is a $\crit\mleft(j\mright)$-complete\footnote{I have not shown that $\crit\mleft(j\mright)$ is a cardinal. However, notice that the definition of $\lambda$-completeness extends naturally to ordinals in the degenerate way that non-principal ultrafilters cannot be $\lambda$-complete for some $\lambda \in \On \setminus \Card$ because then $\left|\lambda\right| < \lambda$. I will in fact use this observation later to justify that $\crit\mleft(j\mright)$ is indeed a cardinal.} non-principal ultrafilter:

	The fact that $U$ is a filter follows from the corresponding facts
	\begin{itemize}
		\item $j\mleft(\varnothing\mright) = \varnothing$;
		\item $j\mleft(u \cap v\mright) = j\mleft(u\mright) \cap j\mleft(v\mright)$;
		\item $u \subseteq v$ implies $j\mleft(u\mright) \subseteq j\mleft(v\mright)$,
	\end{itemize}
	which all follow from the elementarity of $j$. $U$ is an ultrafilter because similarly
	\[\crit\mleft(j\mright) \in j\mleft(\crit\mleft(j\mright)\mright) = j\mleft(x\mright) \cup j\mleft(\crit\mleft(j\mright) \setminus x\mright)\]
	for any $x \subseteq \crit\mleft(j\mright)$, thus either $x \in U$ or $\crit\mleft(j\mright) \setminus x \in U$.

	$U$ is non-principal because, if $x = \left\{\alpha\right\} \subseteq \crit\mleft(j\mright)$ is a singleton, then
	\[j\mleft(x\mright) = \left\{j\mleft(\alpha\mright)\right\} = \left\{\alpha\right\} = x.\]
	Thus $\crit\mleft(j\mright) \not\in j\mleft(x\mright)$. Finally $U$ is $\kappa$-complete because, for some $S \subseteq U$ such that $\left|S\right| < \crit\mleft(j\mright)$, let $f : \left|S\right| \rightarrow S$ be a bijection. By the elementarity of $j$,
	\[j\mleft(\bigcap_{\alpha \in \left|S\right|} f\mleft(\alpha\mright)\mright) = \bigcap_{\alpha \in \left|S\right|} j\mleft(f\mright)\mleft(\alpha\mright) = \bigcap_{\alpha \in \left|S\right|} j\mleft(f\mleft(\alpha\mright)\mright).\]
	Thus $\crit\mleft(j\mright) \in j\mleft(\bigcap_{\alpha \in \left|S\right|} f\mleft(\alpha\mright)\mright)$.

	Finally, observe that $\crit\mleft(j\mright)$ must be a cardinal because, otherwise, no ultrafilter can be $\crit\mleft(j\mright)$-complete and non-principal.

	The forward direction is more complicated and I will give a sketch of the proof, found as proposition 5.4 in Kanamori's book \cite{higher-infinite}:

	Let $\kappa \in \Card$ be a measurable cardinal and $U \subseteq \mathcal{P}\mleft(\kappa\mright)$ be a $\kappa$-complete non-principal ultrafilter on $\kappa$. We wish to realise the model-theoretic structure of an \emph{ultraproduct} $\Pi_{\lambda \in \kappa} V / U$ as a class of sets. This is the quotient of the class of functions $f : \kappa \rightarrow V$ by the equivalence relation
	\[f \sim g \quad \text{if and only if} \quad \left\{\lambda \in \kappa : f\mleft(\lambda\mright) = g\mleft(\lambda\mright)\right\} \in U.\]
	For each function $f : \kappa \rightarrow V$, using Scott's trick in \cite{scotts-trick} and let its equivalence class be represented by the set
	\[\left(f\right)^0_U = \left\{g \in \tensor[^\kappa]{V}{} : g \sim f, \ \rank\mleft(g\mright) = m_f\right\},\]
	where $\rank\mleft(x\mright) = \min\mleft\{\alpha \in \On : x \in V_\alpha\mright\}$ is the rank of a set in the standard cumulative hierarchy and $m_f = \min\mleft\{\rank\mleft(g\mright) : g \in \tensor[^\kappa]{V}{}, \ g \sim f\mright\}$ is the minimum rank of any element in the equivalence class of $f$.

	Thus, let $\Pi_{\lambda \in \kappa} V / U = \left\{\left(f\right)^0_U : f \in \tensor[^\kappa]{V}{}\right\}$ and define the $\in$-relation on it as
	\[\left(f\right)^0_U E_U \left(g\right)^0_U \quad \text{if and only if} \quad \left\{\lambda \in \kappa : f\mleft(\lambda\mright) \in g\mleft(\lambda\mright)\right\} \in U.\]
	It follows from model theory that, for any (first-order) formulae $\varphi\mleft(v_1,
		\ldots, v_n\mright)$ and $f_1, \ldots, f_n \in \tensor[^\kappa]{V}{}$,
	\begin{align*}
		 & \left\langle \Pi_{\lambda \in \kappa} V / U, E_U \right\rangle \vDash \varphi\mleft[\left(f_1\right)^0_U \ldots, \left(f_n\right)^0_U\mright]                    \\
		 & \qquad \text{if and only if} \quad \left\{\lambda \in \kappa : \varphi\mleft[f_1\mleft(\lambda\mright), \ldots, f_n\mleft(\lambda\mright)\mright]\right\} \in U.
	\end{align*}
	Thus, it can be easily verified that the relation $E_U$ is set-like and well-founded. By Mostowski collapse, there exists a class $M \subseteq V$ such that $\left\langle \Pi_{\lambda \in \kappa} V / U, E_U \right\rangle \cong \left\langle M, \in \right\rangle$.

	Finally, let class function $j : V \rightarrow M$ represent mapping each set $x$ to the constant function $y \mapsto x$ in $\tensor[^\kappa]{V}{}$. Then $j$ is clearly an elementary embedding. Notice that, suppose $\alpha \in \kappa$ is the least ordinal such that $j\mleft(\alpha\mright) > \alpha$, let $f : \kappa \rightarrow V$ be such that $\left(f\right)^0_U$ represent $\alpha$ in the ultraproduct, then
	\[\kappa \setminus \bigcap_{\beta \in \alpha} \left\{\lambda \in \kappa : f\mleft(\lambda\mright) \neq \beta\right\} = \left\{\lambda \in \kappa : f\mleft(\lambda\mright) \in \alpha\right\} \in U.\]
	Thus $\left\{\lambda \in \kappa : f\mleft(\lambda\mright) = \beta\right\} \in U$ for some $\beta \in \alpha$, i.e.\ $\left(f\right)^0_U$ represents $j\mleft(\beta\mright)$. In other words, $j\mleft(\beta\mright) = \alpha > \beta$, contradicting the minimality of $\alpha$. Therefore, for any $\alpha < \kappa$, it must follow that $j\mleft(\alpha\mright) = \alpha$.

	It suffices now to verify that $j\mleft(\kappa\mright) > \kappa$. To prove this, we consider the identity function $\mathrm{id} : \kappa \rightarrow \kappa$ and the ordinal $\delta$ that corresponds to $\left(\mathrm{id}\right)^0_U$. Obviously $\delta < j\mleft(\kappa\mright)$. On the other hand, for each $\alpha < \kappa$, $\alpha = j\mleft(\alpha\mright) < \delta$. Thus $\delta \geq \kappa$. Therefore, it follows that $\kappa < j\mleft(\kappa\mright)$. Indeed $\kappa = \crit\mleft(j\mright)$.
\end{proof}

\begin{corollary}
	If there exists a measurable cardinal, then $V \neq L$.
\end{corollary}

\begin{proof}
	Let $\kappa$ be the least measurable cardinal and consider the corresponding elementary embedding $j : V \prec M$ given in \autoref{thm:measurable-cardinals-are-critical-points}. Here, $M$ is a transitive model of $\ZFC$ containing $\On$ by \autoref{lem:elementary-embedding-preserve-ordinals}, thus it follows\footnote{It is an elementary result that G\"odel's constructive universe $L$ is contained in any inner model of $\ZFC$. See Theorem 32 in section 12 of \cite{jech-set-theory} for the proof.} that $L \subseteq M$.

	Since $\kappa = \crit\mleft(j\mright)$ is the least measurable cardinal, by elementarity of $j$,
	\[M \vDash \text{$j\mleft(\kappa\mright)$ is the least measurable cardinal}.\]
	As $j\mleft(\kappa\mright) \neq \kappa$, it must follow that $L \subseteq M \subsetneq V$, i.e.\ $V \neq L$.
\end{proof}

It follows that $L \not\vDash \text{there exists a measurable cardinal}$ and thus measurable cardinals are ``large''. To demonstrate that measurable cardinals indeed lie above inaccessible and Mahlo cardinals on a linear hierarchy, I will also mention the following results:

\begin{proposition}
	\label{prop:measurable-is-inaccessible}
	Let $\kappa \in \Card$ be a measurable cardinal, then $\kappa$ is inaccessible.
\end{proposition}

\begin{proof}
	Let $U$ be a $\kappa$-complete non-principal ultrafilter on $\kappa$. Suppose that $C \subseteq \kappa$ is unbounded, yet $\left|C\right| < \kappa$. For each $\alpha \in C$, $\left|\alpha\right| < \kappa$. Thus, if $\alpha \in U$, then
	\[\varnothing = \bigcap \left(\left\{\alpha\right\} \cup \left\{\kappa \setminus \left\{\delta\right\} : \delta \in \alpha\right\}\right) \in U.\]
	Therefore, $\kappa \setminus \alpha \in U$ instead. However, then
	\[\varnothing = \bigcap_{\alpha \in C} \kappa \setminus \alpha \in U.\]
	By contradiction, one must have $\left|C\right| = \kappa$, i.e.\ $\kappa$ is regular.

	To show that $\kappa$ is additionally a strong limit, suppose that some cardinal $\lambda < \kappa$, yet $2^\lambda \geq \kappa$. Consider some arbitrary injection $f : \kappa \rightarrow \mathcal{P}\mleft(\lambda\mright)$ and sets
	\begin{align*}
		x_\alpha & = \left\{\delta \in \kappa : \alpha \in f\mleft(\delta\mright)\right\},    \\
		y_\alpha & = \left\{\delta \in \kappa : \alpha \not\in f\mleft(\delta\mright)\right\}
	\end{align*}
	for $\alpha \in \lambda$. Notice that, for each $\alpha \in \lambda$, exactly one of $x_\alpha$ and $y_\alpha$ lies in $U$. Thus, define
	\[z_\alpha = \left\{\begin{aligned}
			 & x_\alpha &  & \text{if $x_\alpha \in U$}, \\
			 & y_\alpha &  & \text{if $y_\alpha \in U$},
		\end{aligned}\right.\]
	then $z = \bigcap_{\alpha \in \lambda} z_\alpha \in U$. It follows that, for each $\alpha \in \kappa$, $\alpha \in z$ if and only if
	\[f\mleft(\alpha\mright) = \left\{\beta \in \kappa : x_\beta \in U\right\}.\]
	Since $f$ is injective, $z$ contains at most one element, but then a non-principal ultrafilter $U$ cannot contain $z$. Thus, $\kappa$ is a strong limit by contradiction.
\end{proof}

\begin{lemma}
	\label{lem:elementary-embedding-critical-point-in-club}
	Let $j : V \rightarrow M$ be an elementary embedding and $C \subseteq \crit\mleft(j\mright)$ be a club in $\crit\mleft(j\mright)$, then $\crit\mleft(j\mright) \in j\mleft(C\mright)$.
\end{lemma}

\begin{proof}
	For each $\alpha \in C$, $j\mleft(\alpha\mright) = \alpha$. Thus $C \subseteq j\mleft(C\mright)$. Since $C$ is unbounded in $\crit\mleft(j\mright)$ and $j\mleft(C\mright)$ is a closed subset of $j\mleft(\crit\mleft(j\mright)\mright) > \crit\mleft(j\mright)$, it follows that $\crit\mleft(j\mright) \in j\mleft(C\mright)$.
\end{proof}

\begin{proposition}
	Let $\kappa \in \Card$ be a measurable cardinal, then $\kappa$ is Mahlo.
\end{proposition}

\begin{proof}
	Let $j : V \prec M$ be the corresponding elementary embedding such that $\kappa = \crit\mleft(j\mright)$. Let $R = \left\{\lambda \in \kappa : \text{$\lambda$ is an inaccessible cardinal}\right\}$, then
	\[j\mleft(R\mright) = \left\{\lambda \in j\mleft(\kappa\mright) : \left(\text{$\lambda$ is an inaccessible cardinal}\right)^M\right\}.\]
	Since $M$ is an inner model, by \autoref{prop:measurable-is-inaccessible} and \autoref{prop:small-cardinals-absolute-for-inner-model}, $\kappa \in j\mleft(R\mright)$. Now, for any club $C \subseteq \kappa$, it follows from \autoref{lem:elementary-embedding-critical-point-in-club} that
	\[\kappa \in j\mleft(C\mright) \cap j\mleft(R\mright) = j\mleft(C \cap R\mright).\]
	Thus $C \cap R$ is non-empty and $R$ is indeed stationary.
\end{proof}

Finally, I will define the Woodin cardinals, a type of large cardinals heavily studied in the past several decades, which will serve as a key example to many arguments in subsequent sections. They were first introduced in \cite{woodin-cardinals} in the study of regularity properties of real numbers, which will be explained later in \autoref{subsec:determinacy}:

\begin{definition}[Woodin cardinals]
	A cardinal $\kappa \in \Card$ is \emph{Woodin} if, for any function $f : \kappa \rightarrow \kappa$, there exists a cardinal $\alpha \in \kappa$ and an elementary embedding $j : V \prec M$ such that
	\begin{itemize}
		\item $f\mleft[\alpha\mright] \subseteq \alpha$,
		\item $\crit\mleft(j\mright) = \alpha$,
		\item $V_{j\mleft(f\mright)\mleft(\alpha\mright)} \subseteq M$.
	\end{itemize}
\end{definition}

For now, I will point out that Woodin cardinals are outliers in a certain ways in the hierarchy of large cardinals examined so far: as commented in the remark after Theorem 26.14 in \cite{higher-infinite}, the least Woodin cardinal is not measurable.

However, it follows immediately from the definition that a Woodin cardinal is preceded by measurable cardinals. Thus, the existence of a Woodin cardinal implies the existence of a measurable cardinal. So a Woodin cardinal is also a ``large'' large cardinal and the large cardinals I surveyed in this section all lie on the following linear hierarchy:
\[\text{weakly inaccessible}, \text{inaccessible}, \text{Mahlo}, \text{measurable}, \text{Woodin}, \ldots\]
satisfying the property that, if $\varphi, \psi$ are two conditions on the hierarchy with $\varphi$ placed higher than $\psi$, then the existence of a $\varphi$-cardinal implies the existence of a $\psi$-cardinal. This is a fragment of a longer, roughly linear hierarchy of large cardinals, displayed on page 472 of \cite{higher-infinite}. In the next section, I will proceed to explain that the hierarchy is also that of derivability and interpretability and, as a consequence, a philosophical commitment to the existence of large cardinals requires careful justification.

\subsection{Independence and interpretability}
\label{subsec:large-cardinal-independence}

The large cardinals are distinct from other (often smaller) cardinals studied in set theory in such a way that they are intuitively plausible and have many interesting implications, as I will demonstrate in later sections, but their existence is not derivable from $\ZFC$. Specifically, the large cardinals examined in the previous section are all weakly inaccessible and it suffices to observe the following results:

\begin{lemma}
	\label{lem:weakly-inaccessible-cardinal-model-of-zfc}
	Let $\kappa \in \Card$ be a weakly inaccessible cardinal, then $L_\kappa \vDash \ZFC$, where $L_\kappa$ is the corresponding stage in the hierarchy of G\"odel's constructible universe.
\end{lemma}

\begin{proof}
	First notice that when $\kappa$ is a cardinal,
	\[V_\kappa \vDash \ZFC - \text{PowerSet} - \text{Replacement}.\]
	When $\kappa$ is additionally inaccessible, it is easy to show that, for any subset $x \subseteq V_\kappa$, $x \in V_\kappa$ if and only if $\left|x\right| < \kappa$. It follows that $V_\kappa \vDash \text{PowerSet} + \text{Replacement}$ additionally.

	Now, if $\kappa$ is weakly inaccessible, by \autoref{prop:small-cardinals-absolute-for-inner-model}, it follows that $L \vDash \text{$\kappa$ is weakly inaccessible}$. I use the standard result that $L \vDash \GCH$, shown as Theorem 34 in section 13 of Jech's book \cite{jech-set-theory}, thus it follows that
	\[L \vDash \text{$\kappa$ is strongly inaccessible}.\]
	By the argument above, $L \vDash \left(V_\kappa \vDash \ZFC\right)$, i.e.\ $L_\kappa \vDash \ZFC$ as desired.
\end{proof}

\begin{theorem}
	\label{thm:weakly-inaccessible-proves-zfc-consistent}
	$\ZFC + \exists \kappa \in \Card \ \text{$\kappa$ is weakly inaccessible} \vdash \Con\mleft(\ZFC\mright)$.
\end{theorem}

\begin{proof}
	Assume the existence of a weakly inaccessible cardinal $\kappa$. I shall prove the result by contradiction and suppose otherwise that $\ZFC$ proves $\bot$. Using \autoref{lem:weakly-inaccessible-cardinal-model-of-zfc}, it follows\footnote{To be precise, I shall comment that the entire discussion here must happen internally to set theory, where formulae are treated through G\"odel arithmetisation. It is a standard fact that there are sufficient resources in first-order set theory to define the claim ``$x \vDash \Gamma$'', where $x$ is a set and $\Gamma$ is a set of (arithmetised) formulae.} that $L_\kappa \vDash \bot$. This is a contradiction and thus $\ZFC$ must be consistent.
\end{proof}

\begin{corollary}
	$\ZFC \nvdash \exists \kappa \in \Card \ \text{$\kappa$ is weakly inaccessible}$.
\end{corollary}

\begin{proof}
	Suppose that $\ZFC \vdash \exists \kappa \in \Card \ \text{$\kappa$ is weakly inaccessible}$, then by \autoref{thm:weakly-inaccessible-proves-zfc-consistent}, we immediately have
	\[\ZFC \vdash \Con\mleft(\ZFC\mright).\]
	This contradicts G\"odel's second incompleteness theorem.
\end{proof}

The corollary says that $\ZFC$ cannot guarantee the existence of any weakly inaccessible cardinal. Thus, a claim of the form ``there exists a $\varphi$-cardinal'', where $\varphi$ denotes ``weakly inaccessible'', ``inaccessible'', ``Mahlo'' or other conditions introduced in the previous section, can only be an axiom instead of a theorem of the standard set theory. They are known as the \emph{large cardinal axioms} and generate the possibility of a pluralist perspective of set theory.

Here, I am primarily contrasting between the Platonist anti-pluralist position that commits to a definite collection of abstract entities that one recognises in mathematics, and a Platonist pluralist position that believes in an ontology, where, for every consistent formal theory in mathematics, there is a distinct collection of entities that fit the description of such a theory. However, such a distinction can be found among anti-realists alike between those who believe in a unique objective characterisation of the mathematical structures and those who allow an artificial choice between multiple such accounts and these disputants will be equally vulnerable to many arguments I shall present below.

Essentially, an anti-pluralist now faces the question of whether the large cardinal axioms hold in his ``collection of sets'', which can no longer be answered by widely-accepted descriptive frameworks like $\ZFC$ and mathematical reasoning only. The observation thus persuades one to commit to a pluralist's more appealing counter-proposal, that is, an ontology of multiple different universes, each of which, as a whole, satisfies the axioms of $\ZFC$, while these collections disagree on whether certain types of large cardinals exist. Indeed, given a large cardinal axiom $\varphi$, a model theorist would, given the soundness of the first-order formal system and results above, claim to work under the following hypothesis when needed:

\begin{hypothesis*}
	There exist structures $M, N$ in the language of set theory such that
	\[M \vDash \ZFC + \varphi, \qquad N \vDash \ZFC + \neg \varphi.\]
\end{hypothesis*}

The pluralist's perspective in the philosophy of set theory is simply a literal reading of this entirely mathematical claim.

\subsubsection*{The problem of interpretability}

In addition to this pluralism debate surrounding any alternative axiomatisations of set theory --- or even the axiom of choice within $\ZFC$ --- the problem of whether to accept a large cardinal axiom involves a more significant commitment. Namely, $\ZFC + \exists \kappa \in \Card \ \text{$\kappa$ is weakly inaccessible}$ is not \emph{interpretable} in $\ZFC$, where

\begin{quote}
	[for arbitrary theories $S, S'$,] $S'$ is interpretable in $S$ if, roughly speaking, the primitive concepts and the range of the variables of $S'$ are definable in $S$ in such a way as to turn every theorem of $S'$ into a theorem of $S$,
\end{quote}

using Chapter 6 of Lindstr\"om's book \emph{Aspects of Incompleteness} (\cite{lindstrom-incompleteness}) as a standard reference. A complicated, more precise definition of interpretation can also be found at the beginning of that chapter. I shall follow Lindstr\"om's notation of $S' \leq S$ when $S'$ is interpretable in $S$, and $S' \equiv S$ when both $S' \leq S$ and $S \leq S'$.

The following immediate consequence of his definitions, demonstrated as Theorem 2 in chapter 6 of the book, is relevant here:

\begin{fact*}
	For a theory $T$ that incorporates sufficient amounts of arithmetic, $T + \Con\mleft(T\mright) \not\leq T$.
\end{fact*}

Specifically, in \autoref{thm:weakly-inaccessible-proves-zfc-consistent} above, it is proven that
\[\ZFC + \exists \kappa \in \Card \ \text{$\kappa$ is weakly inaccessible} \vdash \ZFC + \Con\mleft(\ZFC\mright),\]
thus it follows that $\ZFC < \ZFC + \exists \kappa \in \Card \ \text{$\kappa$ is weakly inaccessible}$ in terms of interpretability.

One can compare this to a different famously disputed claim in set theory, the Continuum Hypothesis ($\CH$):
\[2^\omega = \omega_1.\]
G\"odel showed in \cite{godel-ch} that the constructible universe $L \vDash \ZFC + \CH$ while Cohen established a model of $\ZFC + \neg \CH$ in \cite{cohen-ch-1}, \cite{cohen-ch-2} through the method of forcing. The proofs, known nowadays as the dual methods of inner and outer models, effectively imply that
\[\ZFC \equiv \ZFC + \CH \equiv \ZFC + \neg \CH.\]

In other words, given any universe of $\ZFC$, one can construct a class of sets as a model that satisfies $\CH$ and another that satisfies $\neg \CH$, no matter whether $\CH$ holds in the original universe. While it can still be put to philosophical debate whether the axiom should be true or false in an anti-pluralist's conception of sets, or that two collections of sets exist independently as in a pluralist perspective, parties in the dispute can often agree that there are groups of mathematical entities that satisfy either $\ZFC + \CH$ or $\ZFC + \neg \CH$ --- or fictional depictions of the said structures, for an anti-realist. Therefore, it is a reasonable mathematical endeavour to study either of the two incompatible structures regardless of one's choice of a philosophy of mathematics.

The situation is different for a large cardinal axiom because, as commented above, large cardinal axioms are not interpretable in $\ZFC$. In fact, additionally, large cardinals higher up in the hierarchy, as displayed at the end of the last section, are often\footnote{This can fail at some of the more nuanced steps in the hierarchy, though. As an example, the existence of an inaccessible cardinal can be interpreted in the theory $\ZFC + \exists \kappa \in \Card \ \text{$\kappa$ is weakly inaccessible}$ because, as seen in the proof of \autoref{lem:weakly-inaccessible-cardinal-model-of-zfc}, if $\kappa$ is a weakly inaccessible cardinal, then $L \vDash \text{$\kappa$ is an inaccessible cardinal}$.} not interpretable in theories asserting the existence of smaller large cardinals:

\begin{proposition}
	% CHECK: Weird interchange between text- and math-modes to allow line wraps.
	$\ZFC + \exists \kappa \in \Card \ \text{$\kappa$ is Mahlo} \vdash \Con\mleft(\ZFC + \exists \kappa \in \Card\mright.$ $\kappa$ is $\mleft.\text{inaccessible}\mright)$.
\end{proposition}

\begin{proof}
	Let $\kappa$ be the smallest Mahlo cardinal. A same argument as in the proof of \autoref{lem:weakly-inaccessible-cardinal-model-of-zfc} shows that $V_\kappa \vDash \ZFC$. $V_\kappa$ must contain an inaccessible cardinal because the set
	\[\left\{\lambda \in \kappa : \text{$\lambda$ is an inaccessible cardinal}\right\} \subseteq V_\kappa\]
	is stationary in $\kappa$, hence non-empty. The remainder of the proof proceeds exactly as \autoref{thm:weakly-inaccessible-proves-zfc-consistent}.
\end{proof}

It follows that $\ZFC + \exists \kappa \in \Card \ \text{$\kappa$ is Mahlo}$ is not interpretable in the theory $\ZFC + \exists \kappa \in \Card \ \text{$\kappa$ is inaccessible}$. Similar results can be established, albeit with more difficult proofs, for cardinals higher up in the hierarchy. For example, in section 30 of Jech's book \cite{jech-set-theory}, he established as Corollary 3 that

\begin{proposition}
	If there exists a measurable cardinal, then every uncountable cardinal is a Mahlo cardinal in $L$.
\end{proposition}

Then, given some measurable cardinal $\kappa$, $L_\kappa$ will be a model of $\ZFC$ containing Mahlo cardinals, thus justifying the consistency of $\ZFC + \exists \kappa \in \Card \ \text{$\kappa$ is Mahlo}$. Therefore, likewise,
\[\ZFC + \exists \kappa \in \Card \ \text{$\kappa$ is measurable} \vdash \Con\mleft(\ZFC + \exists \kappa \in \Card \ \text{$\kappa$ is Mahlo}\mright)\]
and a theory of sets with a Mahlo cardinal cannot interpret $\ZFC + \exists \kappa \in \Card \ \text{$\kappa$ is measurable}$.

A Carnapian philosophy of set theory need not worry about this problem. As I have just pointed out in the Carnap versus G\"odel discussion, if the distinct theories combining $\ZFC$ and large cardinal axioms are each seen as a linguistic framework to talk about sets, then Carnap did not insist on a formal proof of consistency. He would view the discovery of inconsistency in an axiom system and the subsequent rejection of such a framework as an entirely pragmatic matter.

To most other pluralists, who do not forgo a consistency requirement in their theory --- that is to say, they insist that there must be a reasonable argument for a theory's consistency before it can be considered in their plurality of set theories, as Hilbert and G\"odel maintained --- this is mildly problematic, as a pluralist is now also responsible for demonstrating the consistency of a large cardinal axiom he wishes to introduce. There does exist natural large cardinal axioms that are inconsistent with $\ZFC$, such as the following condition introduced by William Reinhardt in his PhD thesis \cite{reinhardt-thesis}, now known as the Reinhardt cardinals:

\begin{definition}
	A cardinal $\kappa \in \Card$ is \emph{Reinhardt} if it is the critical point of some elementary embedding $j : V \prec V$.
\end{definition}

As Kunen proved in his 1971 paper \cite{kunen-elementary-embeddings}, the existence of a Reinhardt cardinal is inconsistent with the axiom of choice:

\begin{proposition}
	If $j : V \prec V$ is an elementary embedding, then $j$ is identity. Effectively, assuming $\ZFC$, no Reinhardt cardinals can exist.
\end{proposition}

\begin{proof}
	To begin, an elementary embedding that is not identity must move an ordinal. A proof of this standard fact can be found, for example, as proposition 5.1 in section 5 of Kanamori's book \cite{higher-infinite}. Now, suppose that an elementary embedding $j : V \prec V$ with a critical point $\kappa = \crit\mleft(j\mright)$ exists. Let $\lambda = \sup \left\{j^n\mleft(\kappa\mright) : n \in \omega\right\}$. It follows from elementarity of $j$ that
	\[j\mleft(\left\{j^n\mleft(\kappa\mright) : n \in \omega\right\}\mright) = \left\{j^{n + 1}\mleft(\kappa\mright) : n \in \omega\right\},\]
	so $j\mleft(\lambda\mright) = \lambda$.

	Kunen then cited a theorem of Erd\H{o}s--Hajnal in \cite{erdos-hajnal} that there exists a function $f : \tensor[^\omega]{\lambda}{} \rightarrow \lambda$, where $\tensor[^\omega]{\lambda}{}$ refers to the set of functions $\omega \rightarrow \lambda$, such that for any $A \subseteq \lambda$, if $\left|A\right| = \lambda$, then the image $f\mleft[\tensor[^\omega]{A}{}\mright] = \lambda$.

	Consider $A = \left\{j\mleft(\delta\mright) : \delta \in \lambda\right\}$. Then, by elementarity of $j$, precisely
	\[\tensor[^\omega]{A}{} = \left\{j\mleft(t\mright) : \text{$t$ is a function $\omega \rightarrow \lambda$}\right\}.\]
	Since $f$ is surjective, it follows that $j\mleft(f\mright)\mleft[\tensor[^\omega]{A}{}\mright] = j\mleft(\lambda\mright) = \lambda$. Thus, for some $t : \omega \rightarrow \lambda$,
	\[j\mleft(f\mleft(t\mright)\mright) = j\mleft(f\mright)\mleft(j\mleft(t\mright)\mright) = \kappa.\]
	However, the critical point $\kappa$ cannot lie in the range of $j$. A contradiction.
\end{proof}

The risk of inconsistency means that the majority of pluralist philosophers of mathematics seek justification for the large cardinal axioms as well. Many pluralists are ultimately realists and committing to the existence of a collection of sets satisfying a contradictory theory as exemplified above demands a philosophy like that of the ``impossible'' worlds, which is a hard-to-defend position with an overly bloated ontology.

However, I only called this a mild problem for the pluralists because, after all, a formal proof of consistency in Hilbert's sense is simply impossible for any known axiomatisation that interprets enough of $\PA$, as G\"odel's second incompleteness theorem claims. The general belief in the consistency of $\ZFC$ among the mathematicians nowadays is instead due to the thorough studies on the topic --- and the fact that $\ZFC$ has been used to faithfully interpret many of the older areas of mathematics, like number theory and real analysis, that are even more believably consistent --- and thus, inductively, a discovery of inconsistency in $\ZFC$ is seen as an extremely unlikely event. A pluralist supporting the large cardinal axioms can similarly argue that, despite a lack of equiconsistency with $\ZFC$, many large cardinals have been extensively researched on in the past few decades. The many intuitive consequences and strong links to other areas like the descriptive set theory of the projective sets of reals, which I will discuss in more detail in \autoref{subsec:determinacy}, provide support for their consistency almost as powerful as the case for $\ZFC$. In fact, many set theorists are now as confident about the consistency of the majority of natural large cardinal axioms as they are about that of $\ZFC$. For example, as Koellner cited in his essay \cite{koellner-pluralism},

\begin{quote}
	at the G\"odel centenary in Vienna in 2006, Woodin announced that should anyone prove one of these theories\footnote{Here, Koellner is referring to the theory $\ZFC + \AD^{L\mleft(\mathbb{R}\mright)}$, which is roughly equivalent to asserting the existence of many Woodin cardinals.} inconsistent he would resign his post and demand that his position be given to the person who established the inconsistency.
\end{quote}

Therefore, the interpretability results are still mainly an argument against an anti-pluralist. If an anti-pluralist only manages to justify $\ZFC$ or a short initial segment of the hierarchy of large cardinals, then he has no other resort, such as utilising a sub-structure of his collection of sets, when he wishes to adopt a new large cardinal axiom that represents a rise in interpretation strength. In order to maintain that his ontology of sets is a sufficiently powerful tool for many constructions in set theory today, he is tasked with providing metaphysical justifications for each and every large cardinal he intends to commit to. This is indeed a formidable quest for many perspectives of the philosophy of mathematics.

The next sections shall be dedicated to this problem and the classical arguments brought up to justify the large cardinal axioms. I will analyse and argue that these approaches provide essentially pragmatic reasons: they are nowhere near satisfactory for an anti-pluralist, but fit nicely into Carnap's perspective of supporting the adoption of a corresponding linguistic framework.

\section{Arguments for large cardinal axioms}

\subsection{Reflection principles}

One of the heavily investigated ways to justify the large cardinal axioms is through the more intuitive \emph{reflection principles}, that is,

\begin{definition}
	Let $\Gamma$ be a collection of formulae, then \emph{$\Gamma$-reflection} refers to the axiom schema of all statements of the following form:
	\[\forall \mathbf{x} \left(\varphi\mleft(\mathbf{x}\mright) \rightarrow \exists \alpha \in \On \left(\mathbf{x} \in V_\alpha^n \land \varphi^{V_\alpha}\mleft(\mathbf{x}\mright)\right)\right),\]
	where $\mathbf{x}$ is a finite list of $n$ variables, $\varphi \in \Gamma$ and $\varphi^{V_\alpha}$ is the relativisation\footnote{The relativisation	invoked here is the standard notion in set theory such that $V \vDash \varphi^{V_\alpha}$ if and only if $V_\alpha \vDash \varphi$. I shall account for the relativisation of a sentence of higher-order in more details later.} of $\varphi$ to the stage $V_\alpha$ in the cumulative hierarchy of sets.
\end{definition}

Such a schema is essentially asserting that the totality of all sets $V$ is indefinable in such a way that any formulae would not be able to distinguish the totality from a proper initial segment in the hierarchy. G\"odel, for example, had advocated for using such principles as a central motivation for axiomatic set theory, as suggested in \cite{wang-large-sets}:

\begin{quote}
	Generally G\"odel believes that in the last analysis every axiom of infinity should be derived from the (extremely plausible) principle that $V$ is undefinable, where definability is to be taken in more and more generalized and idealized sense.
\end{quote}

where, by the term ``axioms of infinity'', he also included the large cardinal axioms. It is indeed true to maintain that many of the axioms of $\ZFC$ are derivable from the reflection principles. For example,

\begin{proposition}
	$\ZFC - \textup{Infinity} + \textup{$\Delta^0_0$-reflection} \vdash \textup{Infinity}$.
\end{proposition}

\begin{proof}
	Simply take the instance of reflection principle
	\[\Ind\mleft(V\mright) \rightarrow \exists \alpha \in \On \, \Ind\mleft(V_\alpha\mright)\]
	where $\Ind\mleft(A\mright) \coloneqq \varnothing \in A \land \forall x \in A \ x \cup \left\{x\right\} \in A$ asserts that the class $A$ is inductive. Since the axioms of pairing and union imply $\Ind\mleft(V\mright)$, it follows that $\exists \alpha \in \On \, \Ind\mleft(V_\alpha\mright)$, i.e.\ there exists an infinite set.
\end{proof}

However, these first-order instances are very weak and cannot be invoked to support axioms non-derivable from $\ZFC$. In Jech's book \cite{jech-set-theory}, the following result is proven as Theorem 29 in section 11:

\begin{theorem}
	Let $\varphi$ be a formula. For each (set) $M_0$ there is a limit ordinal $\alpha$ such that $M_0 \subseteq V_\alpha$ and $V_\alpha$ reflects $\varphi$, that is, for any tuple of elements $\mathbf{a} \in V_\alpha^n$
	\[V_\alpha \vDash \varphi\mleft[\mathbf{a}\mright] \quad \text{if and only if} \quad \varphi\mleft[\mathbf{a}\mright].\]
\end{theorem}

In other words, for any collection of first-order formulae $\Gamma$, $\ZFC + \text{$\Gamma$-reflection}$ is precisely the same theory as $\ZFC$. Therefore, to realise G\"odel's hope of deriving large cardinal axioms from reflection principles, one must utilise instances where the reflected formula $\varphi$ is higher-order. I shall use Koellner's paper \cite{reflection-principles} as a main reference to survey the contemporary developments in this area before commenting on the philosophical plausibility of the approach.

To begin, suppose that a higher-order language is given where, for each $n \geq 2$, variables $X^{(n)}, Y^{(n)}, Z^{(n)}, \ldots$ of the $n^\text{th}$ order are used. The higher-order reflection principles are evaluated using the following formal characterisation of relativisation of higher-order formulae:

\begin{definition}[Relativisation]
	Let $\alpha \in \On$ and $\varphi$ be a higher-order formula. The \emph{relativisation} $\varphi^{V_\alpha}$ of $\varphi$ to the transitive class $V_\alpha$ is produced by replacing each first-order quantification
	\[\text{$\forall x$ by $\forall x \in V_\alpha$}; \quad \text{$\exists x$ by $\exists x \in V_\alpha$}\]
	and each higher-order quantification
	\[\text{$\forall X^{(n)}$ by $\forall X^{(n)} \in V_{\alpha + n - 1}$}; \quad \text{$\exists X^{(n)}$ by $\exists X^{(n)} \in V_{\alpha + n - 1}$}\]
	for each $n \geq 2$. The relativised formula is explicitly a first-order formula.

	Just like proper classes, an $n^\text{th}$ order \emph{parameter} $A^{(n)}$ is represented by a (possibly parametrised) formula $\varphi\mleft(X^{(n - 1)}\mright)$ with one free variable $X^{(n - 1)}$ such that $B^{(n - 1)} \in A^{(n)}$ if any only if $\varphi\mleft[B^{(n - 1)}\mright]$ holds. The \emph{relativisation} of a parameter is defined recursively:
	\begin{itemize}
		\item let $A^{(2)}$ be a second-order class, its relativisation\footnote{Here, I follow Koellner \cite{reflection-principles} and define the relativisation of a second-order parameter through an explicit intersection instead of the relativisation of its defining formula. The consideration is that a second-order variable $X^{(2)}$ may range over undefinable collections of sets and one cannot define the relativisation of $X^{(2)}$ uniformly if a defining formula is required.} to the transitive class $V_\alpha$ is simply $A^{(2), V_\alpha} = A^{(2)} \cap V_\alpha \in V_{\alpha + 1}$;
		\item for $n \geq 2$, the relativisation of some $\left(n + 1\right)^\text{th}$ parameter $A^{(n + 1)}$ is
		      \[A^{(n + 1), V_\alpha} = \left\{B^{(n), V_\alpha} : B^{(n)} \in A^{(n + 1)}\right\} \in V_{\alpha + n}.\]
	\end{itemize}
\end{definition}

\begin{definition}
	Let $\Gamma$ be a collection of higher-order formulae, then $\Gamma$-reflection refers to the axiom schema of all statements of the following form:
	\begin{align*}
		\forall \mathbf{x} \  & \forall X_1^{(n_1)} \cdots \forall X_k^{(n_k)} \left(\varphi\mleft(\mathbf{x}, X_1^{(n_1)} \ldots, X_k^{(n_k)}\mright)\right.                                                                     \\
		                      & \left.{} \rightarrow \exists \alpha \in \On \left(\mathbf{x} \in V_\alpha^n \land \varphi^{V_\alpha}\mleft(\mathbf{x}, X_1^{(n_1), V_\alpha} \ldots, X_k^{(n_k), V_\alpha}\mright)\right)\right),
	\end{align*}
	where $\mathbf{x}$ is a finite list of $n$ variables, $X_1^{(n_1)}, \ldots, X_k^{(n_k)}$ are higher-order variables and $\varphi \in \Gamma$.
\end{definition}

Immediately, one should observe that reflection principles given in this manner are not always consistent with $\ZFC$. I shall cite Koellner's counterexample:

\begin{proposition}
	\label{prop:non-positive-reflection-principle-inconsistent}
	Let the formula
	\[\varphi\mleft(X^{(3)}\mright) = \forall Y^{(2)} \in X^{(3)} \ \exists \alpha \in On \ Y^{(2)} \subseteq \alpha.\]
	denote the concept ``every element in $X^{(3)}$ is a bounded class of ordinals''. Then the reflection principle
	\[\forall X^{(3)} \left(\varphi\mleft(X^{(3)}\mright) \rightarrow \exists \alpha \in \On \ \varphi^{V_\alpha}\mleft(X^{(3), V_\alpha}\mright)\right)\]
	is false, assuming $\ZFC$.
\end{proposition}

\begin{proof}
	Let $A^{(3)} = \left\{\left\{\delta : \delta < \alpha\right\} : \alpha \in \On\right\}$ be a third-order parameter. Then $\varphi\mleft[A^{(3)}\mright]$ holds by definition. However, for any $\alpha \in \On$,
	\[A^{(3), V_\alpha} = \left\{\alpha \cap \beta : \beta \in \On\right\} = \alpha + 1 \in V_{\alpha + 2}.\]
	Here, $\alpha \in A^{(3), V_\alpha}$, yet $V_\alpha \nvDash \text{$\alpha$ is a bounded class}$. Thus the reflection principle above is false.
\end{proof}

As identified in both Tait's work \cite{tait-cardinals-from-below} and Koellner's follow-up \cite{reflection-principles}, the well-behaved collection of formulae to reflect on are the $\Gamma^{(2)}_n$-formulae, which are positive in the following sense:

\begin{definition}
	Let $\varphi$ be a formula involving only first-order quantification. $\varphi$ can have higher-order free variables. $\varphi$ is said to be \emph{positive} if it is built up by connectives $\land$, $\lor$ and quantifiers from atomic formulae of the form $x = y$, $x \neq y$, $x \in y$, $x \not\in y$, $x \in Y^{(2)}$, $x \not\in Y^{(2)}$ and $X^{(n)} = Y^{(n)}$, $X^{(n)} \in Y^{(n + 1)}$ for $n \geq 2$.

	For $0 < n < \omega$, the $\Gamma^{(2)}_n$-formulae are of the form
	\[\forall X^{(2)}_1 \ \forall Y^{(\ell_1)}_1 \cdots \forall X^{(2)}_n \ \forall Y^{(\ell_n)}_n \ \varphi\]
	where $\varphi$ is a positive formula involving only first-order quantifications, but can have other free variables of any order. $\ell_1, \ldots, \ell_n$ are arbitrary natural numbers.

	In general, for $m > 2$ and $0 < n < \omega$, the $\Gamma^{(m)}_n$-formulae are of the form
	\[\forall X^{(m)}_1 \ \forall Y^{(\ell_1)}_1 \cdots \forall X^{(m)}_n \ \forall Y^{(\ell_n)}_n \ \varphi\]
	where $\varphi$ is a $\Gamma^{(m - 1)}_k$-formula for some $k > 0$ and $\ell_1, \ldots, \ell_n$ are arbitrary natural numbers.
\end{definition}

Notice that in a positive formula, one is essentially avoiding the components $X^{(n)} \neq Y^{(n)}$ and $X^{(n)} \not\in Y^{(n + 1)}$ in its canonical disjunctive normal form. Since a second-order logic is usually taken to be extensional, that is,
\[X^{(2)} = Y^{(2)} \equiv \forall x \left(x \in X^{(2)} \leftrightarrow x \in Y^{(2)}\right),\]
thus any second-order formulae with only first-order quantifications involves no atomic relation expression between two higher-order variables and is immediately positive. I will first show that adding such $\Gamma^{(2)}_n$-reflection axioms does make our set theory ``stronger'':

\begin{proposition}
	$ZFC + \text{$\Gamma^{(2)}_1$-reflection} \vdash \exists \kappa \in \Card \ \text{$\kappa$ is inaccessible}$.
\end{proposition}

\begin{proof}
	To begin, notice that $\alpha \in \On$ is a limit ordinal if $V_\alpha \vDash \text{Pairing} + \text{Union}$. Also, $\alpha > \omega$ if $V_\alpha \vDash \text{Infinity}$. Let
	\[\text{$\On$ is regular} \coloneqq \forall X^{(2)} \ \neg \exists \alpha \in \On \left(\text{$X^{(2)} : \alpha \rightarrow \On$ is unbounded}\right),\]
	then, for a limit ordinal $\alpha$, $\alpha$ is a regular cardinal if $V_\alpha \vDash \text{$\On$ is regular}$.

	Similarly, let
	\[\text{$\On$ is strong limit} \coloneqq \forall X^{(2)} \neg \exists \alpha \in \On \left(\text{$X^{(2)} : \mathcal{P}\mleft(\alpha\mright) \rightarrow \On$ is bijection}\right),\]
	then, for a cardinal $\alpha$, $\alpha$ is a strong limit if $V_\alpha \vDash \text{$\On$ is strong limit}$.

	Therefore, consider the $\Gamma^{(2)}_1$-formula
	\[\varphi = \text{Pairing} \land \text{Union} \land \text{Infinity} \land \text{$\On$ is regular} \land \text{$\On$ is strong limit}\]
	that is obviously true in $V$. It follows that some $V_\alpha \vDash \varphi$ and this $\alpha \in \On$ is correspondingly an inaccessible cardinal.
\end{proof}

\begin{proposition}
	$\ZFC + \text{$\Gamma^{(2)}_1$-reflection} \vdash \exists \kappa \in \Card \ \text{$\kappa$ is Mahlo}$.
\end{proposition}

\begin{proof}
	Let $A^{(2)} = \left\{\kappa \in \Card : \text{$\kappa$ is inaccessible}\right\}$. Consider the formula
	\begin{align*}
		\text{$X^{(2)}$ is stationary} \coloneqq \forall Y^{(2)} & \ \exists Z^{(2)} \Big(\exists x \ x \in X^{(2)} \cap Y^{(2)} \lor \text{$Y^{(2)}$ is bounded}                         \\
		                                                         & \left.{} \lor \left(\text{$Z^{(2)} \subseteq Y^{(2)}$ is bounded} \land \bigcup Z^{(2)} \not\in Y^{(2)}\right)\right).
	\end{align*}
	that literally asserts that $X^{(2)}$ intersects every sub-class of $\On$ that is closed and unbounded. If one can prove that $A^{(2)}$ is stationary, it follows that some
	\[V_\alpha \vDash \text{Pairing} + \text{Union} + \text{$\On$ is regular} + \text{$A^{(2), V_\alpha}$ is stationary},\]
	i.e.\ the corresponding $\alpha \in \On$ is a Mahlo cardinal.

	Now, consider an arbitrary $C^{(2)} \subseteq \On$ that is closed and unbounded. This is the $\Gamma^{(2)}_1$-assertion
	\begin{align*}
		\text{$C^{(2)}$ is club} \coloneqq \forall X^{(2)} & \Big(\text{$C^{(2)}$ is unbounded}                                                                                        \\
		                                                   & \left.{} \land \left(\text{$X^{(2)} \subseteq C^{(2)}$ is bounded} \rightarrow \bigcup X^{(2)} \in C^{(2)}\right)\right).
	\end{align*}
	Using the same $\varphi$ as in the proof of the previous proposition, then $\varphi \land \text{$C^{(2)}$ is a club}$ is true in $V$. It follows from the reflection principle that some
	\[V_\alpha \vDash \varphi \land \text{$C^{(2), V_\alpha}$ is a club}.\]
	Here, $\alpha$ is an inaccessible cardinal, $C^{(2)}$ is closed and $C^{(2), V_\alpha} \subseteq C^{(2)}$ is unbounded in $\alpha$. Thus $\alpha \in C^{(2)} \cap A^{(2)}$ and $A^{(2)}$ is stationary as desired.
\end{proof}

Therefore, if an anti-pluralist can maintain that the collection of sets satisfy the axioms of $\Gamma^{(2)}_n$-reflection, then he is justified in claiming that the inaccessible and Mahlo cardinals exist. This approach is explicitly taken by G\"odel in his essay \cite{godel-continuum-axioms} as he based his intuition of set theory on the repeated applications of the operation ``set of'', that is, the power set construction
\[V_{\alpha + 1} = \mathcal{P}\mleft(V_\alpha\mright)\]
in the iterative conception of set theory. G\"odel argued that

\begin{quote}
	the totality of sets obtainable by the use of the procedures of formation of sets expressed in the other axioms forms again a set,
\end{quote}

implying that any (recursively enumerable) collection of axioms shall characterise a proper initial segment of the $V_\alpha$ hierarchy. Therefore, G\"odel viewed the reflection principles as intrinsically justified through theoretical reasons and that they support the large cardinal axioms ``without arbitrariness''.

However, as Koellner proved in \cite{reflection-principles}, the implications of reflection principles in this form is very limited: through section 4-5 in the paper, he established the following dichotomy:

\begin{theorem}
	Suppose that $\kappa\mleft(\omega\mright)$ is the $\omega$-Erd\H{o}s cardinal. There exists some $\delta \in \kappa\mleft(\omega\mright)$ such that $V_\delta \vDash \text{$\Gamma^{(2)}_n$-reflection}$ for every $0 < n < \omega$.

	On the other hand, the next stronger theory $\Gamma^{(3)}_1$-reflection is inconsistent with $\ZFC$.
\end{theorem}

Here, the large cardinal upper bound involved arises from the study of partitions in \cite{partition-calculus}:

\begin{definition}
	For $\gamma \in \On$ and some set $x \subseteq \On$, let
	\[\left[x\right]^\gamma = \left\{y \subseteq x : \text{$y$ has ordertype $\gamma$}\right\}.\]
	Then, for ordinals $\alpha, \beta, \gamma$ and a cardinal $\delta$, the \emph{ordinary partition relation} $\beta \rightarrow \left(\alpha\right)^\gamma_\delta$ denotes the assertion that, for any function $f : \left[\beta\right]^\gamma \rightarrow \delta$, there exists some $H \in \left[\beta\right]^\alpha$ that is homogeneous for $f$, i.e.\ $f\mleft[\left[H\right]^\gamma\mright] \leq 1$.
\end{definition}

\begin{definition}[Erd\H{o}s cardinals]
	Let $\alpha \leq \omega$ be an ordinal. The \emph{$\alpha$-Erd\H{o}s cardinal} $\kappa\mleft(\alpha\mright)$, if exists, is the least cardinal $\lambda$ such that $\lambda \rightarrow \left(\alpha\right)^{<\omega}_2$, that is, $\lambda \rightarrow \left(\alpha\right)^n_2$ holds for all $n \in \omega$.
\end{definition}

Now, it is proven as Theorem 78 in section 32 of Jech \cite{jech-set-theory} that

\begin{theorem}
	Let $\kappa$ be a cardinal. If $\kappa \rightarrow \left(\omega\right)^{<\omega}_2$, then $L \vDash \kappa \rightarrow \left(\omega \right)^{<\omega}_2$.
\end{theorem}

As an immediate corollary, the $\omega$-Erd\H{o}s cardinal $\kappa\mleft(\omega\mright)$ is small: even if one is to accept the reflection principles on $\Gamma^{(2)}_n$-formulae as a part of the standard axiomatisation of set theory, he has to agree with Koellner that the new axioms are weak. The existence of many larger cardinals, including the measurable and Woodin cardinals that are present heavily in contemporary research, are not guaranteed by reflection principles and G\"odel's hope of deriving every axiom of infinity from the reflection principles is yet far from being realised.

A more significant problem lies in G\"odel's distinction between intrinsic and extrinsic justifications for an axiom of set theory and the fact that most supporters of the reflection principle view it as having intrinsic evidence, that is, on the stronger side of the distinction. As G\"odel classified in \cite{godel-continuum-axioms}, intrinsic reasons arise naturally from the further analysis of the concept of sets and appeal to the means of intuition and analogy, while extrinsic support is measured through the ``success'' of an axiom, in terms of how the adoption of the said axiom can produce fruitful theorems and simplified proofs. As I have mentioned, reflection principles, as G\"odel viewed them, are intuitive consequences of the understanding that the iterations of the set formation operation are unrestricted from above and hence provide an intrinsic justification for hypotheses like certain large cardinal axioms.

However, such a G\"odelian argument, if accepted, is an \emph{a priori} support for all forms of reflection principles whereas, as shown in the counterexample \autoref{prop:non-positive-reflection-principle-inconsistent}, some of them are simply inconsistent with $\ZFC$. Therefore, there is a risk of the fallacy of bad company and one who wishes to adopt the argument as an intrinsic justification for new axioms of set theory must explain why, intuitively, the inconsistent reflection principles are not justified by his procedure. However, this can be extremely difficult as even Koellner's work in \cite{reflection-principles} are restricted to positive formulae in general. In section 6 of the paper, he proved a more refined version of his dichotomy where $\Gamma^{c\mleft(m\mright)}_n$-reflection, involving formulae with $n$ universal quantifications over closed $m^\text{th}$-order classes on $\On$, is consistent given the existence of the $\omega$-Erd\H{o}s cardinal while $\Gamma^{\Gamma\mleft(m\mright)}_n$-reflection, where the universal quantifications are over some $m^\text{th}$-order pointclass in the generalised Borel hierarchy, is inconsistent with $\ZFC$. It is hardly imaginable that philosophical and intuitive considerations can motivate such a nuanced classification in mathematics, justifying only the consistent portion of the reflection principles, let alone covering the more complicated formulae not analysed in Koellner's characterisations.

Indeed, as Koellner has suspected in \cite{reflection-principles}, it is worrying, given this observation, that reflection principles, especially the ones involving higher-order variables, may not be intuitive in G\"odel's picture of the iterative conception of sets at all. This is because the G\"odelian argument is essentially asserting the absoluteness of the height of the universe of sets. However, a higher-order language requires the identification and quantification of arbitrary sub-collections of this totality and it is highly problematic to attempt to interpret it on the absolute picture. Therefore, Koellner found it hard to conceive how intuition can motivate the acceptance of higher-order principles that are no more than formal syntactic manipulations in such a theory of sets.

I believe that the two aforementioned attacks, concerning the theoretical limitations and the lack of intuitive support for a higher-order reflection principle respectively, combine to suggest that it cannot be an intrinsic argument in G\"odel's sense to appeal to reflection principles to justify the large cardinals. It can still be extrinsically plausible as the (consistent) reflection principles are a reasonable analogy of some first-order attributes $\ZFC$ already has and, as I have surveyed, this is a simple approach of encoding desirable properties of set theory like the axiom of infinity or the existence of some small large cardinals. Therefore, it may be pragmatically sound to argue for the commitment to the existence of the $\omega$-Erd\H{o}s cardinal because, as shown through Koellner's work, it ensures that the set theory satisfies a maximal amount of reflection principles currently known to be possible. However, this is not a satisfactory intrinsic justification of the large cardinals for an anti-pluralist any more, nor a guarantee of their consistency needed by the pluralists, even in the very limited sense Koellner identified.

\subsection{Completeness in \texorpdfstring{$\Omega$}{Omega}-logic}

In \cite{koellner-pluralism}, Koellner had himself proposed ``a new orientation'' against pluralism in set theory, primarily by claiming that there are strong reasons for a commitment to the following axiom

\begin{hypothesis*}
	There exists a proper class of Woodin cardinals. This is denoted the axiom $\PCWC$.
\end{hypothesis*}

Koellner's choice of this axiom is motivated by his belief in the ``intrinsic plausibility'' of $\AD^{L\mleft(\mathbb{R}\mright)}$, an axiom of determinacy on the constructive universe of sets formed on $\mathbb{R}$. Details of the implications of determinacy shall be a central focus in the later \autoref{subsec:determinacy}, and not covered in detail here. It is clear here, though, that Koellner spoke of a different sense of intrinsicality from G\"odel and, with ``a non-skeptical stance towards set theory'' as he expressed in the article \cite{koellner-absolute-undecidability}, considered any desirable consequences as adding to the compelling case for a new axiom. Explicitly, a majority of Koellner's reasons for the axiom of determinacy, as enumberated on pages 30--32 in \cite{koellner-pluralism}, are actually the potential implications of $\AD^{L\mleft(\mathbb{R}\mright)}$ and its strong connections to other hypotheses in different areas --- thus extrinsic evidence, at least in G\"odel's sense. However, the distinct argument that I wish to discuss here, with some flavour of intrinsic intuition, is instead his suggestion that

\begin{quote}
	The axiom $\AD^{L\mleft(\mathbb{R}\mright)}$ appears to be ``effectively complete'' for the theory of $L\mleft(\mathbb{R}\mright)$,
\end{quote}

that is, almost all statements about $L\mleft(\mathbb{R}\mright)$ that are ``of prior mathematical interest'' are settled once $\AD^{L\mleft(\mathbb{R}\mright)}$ is assumed. Since $L\mleft(\mathbb{R}\mright)$ contains all sets that are constructible from real parameters, $\AD^{L\mleft(\mathbb{R}\mright)}$ can be seen as providing a satisfactory description for the entire theory of the reals. Koellner moved to the stronger large cardinal axiom $\PCWC$ as an attempt to formalise this argument:

Let $V$ denote the standard universe of $\ZFC$, $\mathbb{P}$ denote some partial ordering in $V$ and $G \subseteq \mathbb{P}$ is a generic subset. Write $V\mleft[G\mright]$ for the generic extension constructed via forcing.

\begin{definition}[$\Omega$-satisfaction]
	Suppose that $T$ is a countable theory in the language of sets and $\varphi$ is a sentence. Then $T \vDash_\Omega \varphi$ if, for any partial ordering $\mathbb{P}$ in $V$ and ordinal $\alpha \in \On$,
	\[\text{if $V\mleft[G\mright]_\alpha \vDash T$ then $V\mleft[G\mright]_\alpha \vDash \varphi$.}\]
\end{definition}

This concept of $\Omega$-logic was first introduced by Woodin as a strong logic that captures the attributes that are preserved through forcing. For example, as noted in \cite{omega-logic-primer}, since the generic extensions $V\mleft[G\mright]$ are standard models of $\ZFC$,
\[\ZFC \vDash_\Omega \Con\mleft(\ZFC\mright),\]
even though the classical implication $\ZFC \vDash \Con\mleft(\ZFC\mright)$ is false due to G\"odel's second incompleteness theorem. The suprising observation is that one can expect some definable theories to be $\Omega$-complete and $\PCWC$ is one nice assumption that produces such a result:

\begin{definition}[$\Omega$-completeness]
	A theory $T$ is \emph{$\Omega$-complete} for a collection of sentences $\Gamma$ if, for each $\varphi \in \Gamma$, either $T \vDash_\Omega \varphi$ or $T \vDash_\Omega \neg \varphi$.
\end{definition}

The following claim is Theorem 10.157 from Woodin's book \cite{woodin-de-gruyter}:

\begin{theorem}[Woodin]
	Assume $\ZFC + \PCWC$. $\ZFC$ is $\Omega$-complete for the collection of all sentences of the form $L\mleft(\mathbb{R}\mright) \vDash \varphi$.
\end{theorem}

In other words, the axiom $\PCWC$ ensures that the theory of $L\mleft(\mathbb{R}\mright)$ is absolute between different forcing extensions. Since most of the interesting independence proofs in set theory are achieved through forcing today, it settles all statements of the form $L\mleft(\mathbb{R}\mright) \vDash \varphi$ in the sense that an independence proof generating an outer model as a counterexample, similar to that of $\CH$, can never be produced. Along with other evidence such as the success in various branches of set theory, Koellner claimed to have provided strong reasons that one should commit to the hypothesis $\PCWC$ and in turn saw the problem of selection for axioms at a lower level on the interpretability hierarchy as solved --- the hypotheses implied by $\PCWC$, such as $\AD^{L\mleft(\mathbb{R}\mright)}$, the axiom of projective uniformisation or other large cardinal axioms, are the ``correct'' axioms for set theory.

I think that the $\Omega$-completeness condition can be viewed as a stronger yet analogous feature compared to the reflection principles, that is, intuitively desirable for the absolute totality of sets. Effectively, the transitive class $V\mleft[G\mright]$ obtained through forcing is seen as a larger universe $V\mleft[G\mright] \supseteq V$ with some extra generic element $G \not\in V$. Therefore, in the claim above that $\ZFC$ is $\Omega$-complete for some collection of sentences $\Gamma$, the universe has exhausted all ``interesting'' forcing conditions in the sense that adding any generic element does not modify the interpretaton of any sentence in $\Gamma$. One can thus expect the totality of sets as a model of some theory that is $\Omega$-complete for as many sentences as possible.

Then, as Koellner subsequently covered in \cite{koellner-pluralism}, $\Omega$-completeness, just like the reflection principles, cannot apply to all sentences in a desirable manner. Theorem 5.4 and 5.5 in that essay laid a dichotomy of the following form:

\begin{theorem}
	Assume $\ZFC + \PCWC$.  If the \emph{Strong $\Omega$ Conjecture}\footnote{I will not go into the details of the Strong $\Omega$ Conjecture here. Essentially, it is the assertion of both the $\Omega$ conjecture and the $\AD^+$ conjecture, two hypotheses that have provided very useful contexts for strong results in section 10.4--10.5 in Woodin's book \cite{woodin-de-gruyter}.} holds, then there is no recursively enumerable theory $A$ such that $\ZFC + A$ is $\Omega$-complete for $\Sigma^2_3$-sentences.

	On the other hand, suppose that $L$ is a
	large cardinal axiom and $A$ is a recursively enumerable theory such that
	\[\text{$\ZFC + L + A$ is $\Omega$-complete for the theory of $V_\lambda$},\]
	where $V_\lambda$ is some \emph{robust specifiable fragment} of the universe at least as large as $V_{\omega + 2}$, then there exists a recursively enumerable theory $B$ such that
	\[\text{$\ZFC + L + B$ is $\Omega$-complete for the theory of $V_\lambda$},\]
	but which differs from $\ZFC + L + A$ on $\CH$.
\end{theorem}

In other words, there is either no (recursively enumerable) theory that is $\Omega$-complete for a large enough collection of sentences, or there are multiple that are incompatible with each other. Similar to the criticism on reflection principles, it may be argued that the intuition about absoluteness is doubtable: while $\Omega$-completeness is a consequential feature, it is not clear why one's concept of sets must possess it for a certain collection of sentences, specifically given the observation above that the condition can never extend fully to all statements alike.

\subsection{An alternative alsoluteness in width}

I have so far assessed some arguments that an anti-pluralist justification for the large cardinal axioms arises from an intuitive analysis of the notion of absoluteness for the totality of sets. While I argued that these perspectives are flawed, the situation is still quite optimistic for an anti-pluralist like Koellner: while there lacks theoretical evidence that the theory of sets, as currently studied in mathematics, must admit certain large cardinals, one who commits to an abstract ontology of sets still commits to a complete, albeit non-enumerable, theory of these entities, to which the modern axiomatisation $\ZFC$ is an approximation. For someone who wishes to improve this approximation, that is, to work in theories that are ``more complete'', the large cardinal axioms on the interpretability hierarchy, up to as high as $\PCWC$, become a sequence of nice candidates which, considering how they are coherent with our intuition of absoluteness as explained in the previous sections, are potentially the best choices one can currently make.

Here, I shall reject such a view through Neil Barton's counter-suggestion in his recent preprint \cite{barton-large-cardinal-restrictive}, where he demonstrated that an alternative intuition on the absoluteness of the set-theoretic universe can yield an axiomatisation incompatible with the large cardinal axioms.

At its core, Barton's argument is similar to how a philosopher can have two different perspectives when examining $\CH$. One can view a universe satisfying $\neg \CH$ as more inclusive because it asserts the existence of subsets of $\mathbb{R}$ that have cardinality between $\omega$ and $2^\omega$ and are thus non-existent in a inner model satisfying $\CH$. Alternatively, one can claim that $\CH$ states the existence of bijections between all ordinals below $2^\omega$, which will be seen as different cardinals should there lack such functions, and hence is the more inclusive axiom of the two. Similarly, Barton proposed that the large cardinal axioms, which are existential statements, may actually be ``restrictive'' in the sense that a universe satisfying their negations can be framed as a better picture of an absolute totality.

The notion of absoluteness Barton examined is that of the ``width'' of a universe:

\begin{definition}[Width extensions]
	A \emph{width extension} of a universe $V$ is another universe $V'$ such that $V$ is an inner model of $V'$, that is, $V$ is a transitive sub-class of $V'$ containing the ordinals $\On^{V'}$.
\end{definition}

One understands the width extension literally to mean that, while $V'$ has the same ordinals as $V$ and hence the same stages in the standard cumulative hierarchy, some stages $V'_\alpha$ contain more sets than $V_\alpha$. The following axiom that, informally speaking, suggests the absolute totality of sets cannot have interesting width extensions, was first formulated by Sy-David Friedman in the paper \cite{friedman-imh-06}:

\begin{definition}
	The inner model hypothesis, abbreviated as $\IMH$, shall denote the following claim: let $\varphi$ be a first-order sentence, if $\varphi$ is true in an inner model of a width extension of $V$, then $\varphi$ is already true in an inner model of $V$.
\end{definition}

As Barton pointed out, $\IMH$ is not really a first-order axiom but rather a schema of axioms for the different extensions. Therefore, Barton chose to move to the von Neumann--Bernays--G\"odel set theory ($\NBG$). The formulations of $\NBG$ will be omitted here and can be found in chapter 4 of Mendelson's textbook \cite{mendelson}, but it is roughly a finitely axiomatisable fragment of second-order set theory realised in two-sorted logic --- with distinct variables ranging over sets and classes --- and thus solves the problem of quantifying over proper classes that serve as extensions of the universe in $\IMH$. In $\NBG$, Barton considers the following version of the hypothesis:

\begin{definition}
	The class-generic inner model hypothesis, abbreviated as $\CIMH$, refers to either of the following two equivalent claims:
	\begin{itemize}
		\item if a first-order sentence $\varphi$ holds in an inner model of a tame class forcing extension of $V$,  then $\varphi$ holds in an inner model of $V$;
		\item if, for a first-order sentence $\varphi$, there exists a tame class forcing notion $\mathbb{P} \subseteq V$ such that some $p \in \mathbb{P}$ forces ``$\varphi$ is true in an inner model'', then $\varphi$ holds in an inner model of $V$,
	\end{itemize}
	where, by a first-order sentence, Barton meant a sentence containing no variables denoting classes and, by a tame class forcing notion, Barton meant some partial ordering $\mathbb{P} \subseteq V$ such that the forcing extension $V\mleft[G\mright]$ is also a model of $\NBG$.
\end{definition}

The latter form of $\CIMH$ explicitly presents the axiom as a formal sentence in the language of $\NBG$.

As Theorem 8 in the 2008 paper \cite{friedman-imh-08}, it is proven that

\begin{theorem}
	\label{thm:imh-consistent}
	Assume $\ZFC$ and the existence of a Woodin cardinal with an inaccessible above. Then $\IMH$ is consistent.
\end{theorem}

The following is then immediate for Barton:

\begin{corollary}
	Assume $\ZFC$ and the existence of a Woodin cardinal with an inaccessible above. Then $\CIMH$ is consistent.
\end{corollary}

\begin{proof}
	In the proof of \autoref{thm:imh-consistent} in \cite{friedman-imh-08}, it is explicitly constructed that some minimal transitive model $M_d$ of $\ZFC$ containing reals of a specific Turing degree $d$ in a forcing extension $V\mleft[G\mright]$ satisfies $\IMH$. It is easy to verify that $\left\langle M_d, \in, \Def\mleft(M_d\mright) \right\rangle$ then becomes a model of $\NBG + \CIMH$, where the class variables range over the definable subsets of $M_d$.
\end{proof}

Therefore, if one believes that large cardinal axioms such as $\PCWC$ is consistent with $\ZFC$, then $\CIMH$ is also consistent. However, $\CIMH$ is not compatible with large cardinal axioms:

\begin{theorem}
	\label{thm:cimh-no-inaccessibles}
	$\CIMH$ implies that, for some real $r$, $\ZFC$ fails in $L_\alpha\mleft[r\mright]$ for all ordinals $\alpha$. In particular, $\CIMH$ implies that there are no inaccessible cardinals.
\end{theorem}

This is Barton's version of Friedman's Theorem 15 in the paper \cite{friedman-imh-06}, where $L_\alpha\mleft[r\mright]$ denote the $\alpha^\text{th}$ stage in the constructive universe formed from the set $r$ --- Mitchell's textbook chapter \cite{mitchell-inner-model-theory} shall be a standard reference on inner models of such forms. Basically, Friedman worked on countable transitive models and the construction in section 5.2 of the book \cite{beller-coding-universe} is invoked to produce a class forcing extension where $L_\alpha\mleft[r\mright] \nvDash \ZFC$ for all ordinals $\alpha$. $\IMH$ thus naturally asserts that this is also true for the original universe $V$.

Therefore, a theory of sets cannot both satisfy $\CIMH$ and the large cardinal axioms. Barton's point is that, between the choices of $\CIMH$ and a large cardinal axiom, $\CIMH$ is intuitively more inclusive because one can find large cardinals in an inner model if the universe satisfies $\NBG + \CIMH$: Friedman's results in Theorem 2 of \cite{friedman-imh-08} established that

\begin{theorem}
	$\CIMH$ implies that there is a (definable) inner model with measurable cardinals of arbitrarily large Mitchell order.
\end{theorem}

The Mitchell order is covered in Definition 2.2--2.4 in \cite{mitchell-inner-model-theory}. As Barton wrote, here the definitions directly imply that, for any cardinal $\lambda \in \Card$, $\CIMH$ interprets the theory $\ZFC + \text{there are $\lambda$-many measurable cardinals}$. Indeed $\CIMH$ provides a fair interpretation of the theory, in the follow sense in Maddy's 1998 exposition \cite{maddy-maximize}:

\begin{definition}
	Let $\varphi$ be a formula with one free variable. $\varphi$ is a \emph{fair interpretation} of some theory $T_1$ in $T_2$ if and only if
	\begin{itemize}
		\item the class
		      \[\varphi^V = \left\{x \in V : \varphi\mleft(x\mright)\right\}\]
		      is a transitive model of $\ZFC$;
		\item either $\On \subseteq \varphi^V$ or $\kappa \subseteq \varphi^V$ for some inaccessible cardinal $\kappa \in \Card$;
		\item for any sentence $\sigma \in T_1$, $T_2 \vdash \varphi^V \vDash \sigma$.
	\end{itemize}
\end{definition}

Informally, Maddy's condition restricts to only accept the interpretations that are (definable) inner models or truncated initial segments of inner models, that is, the cumulative stage $V_\alpha$ evaluated at some inaccessible $\alpha$ in an inner model. Interpretations where one needs to introduce new sets, for example through forcing, are ruled out. As Barton argued, the converse case fails precisely so, where the interpretation $M_d$ obtained in the proof of \autoref{thm:imh-consistent} is the best possible, in the sense that no fair interpretation of $\IMH$ or $\CIMH$ can be obtained in any consistent extension of $\ZFC$ asserting the existence of some inaccessible cardinal. The key to the observation is the following corollary to \autoref{thm:cimh-no-inaccessibles}:

\begin{corollary}
	Assume $\ZFC + \IMH$. Then there are no worldly cardinals, that is, there is no cardinal $\kappa \in \Card$ such that $V_\kappa \vDash \ZFC$.
\end{corollary}

\begin{proof}
	By \autoref{thm:cimh-no-inaccessibles}, for some real $r$, the inner model $L\mleft[r\mright]$ has no worldly cardinals. It suffices to show that worldly cardinals are downwards absolute for $L\mleft[r\mright]$, i.e.\ they are still worldly cardinals in $L\mleft[r\mright]$. Here, suppose that some $V_\kappa \vDash \ZFC$, then $\left|V_\kappa\right| = \kappa$. We know that $r \in V_\kappa$ and $L_\kappa\mleft[r\mright] = L\mleft[r\mright]^{V_\kappa}$, thus it follows from Theorem 37 in Jech's textbook \cite{jech-set-theory} that $L_\kappa\mleft[r\mright] \vDash \ZFC$ is still a model.
\end{proof}

Now, as \autoref{prop:small-cardinals-absolute-for-inner-model} proves, inaccessible cardinals are still inaccessible in inner models. It is a standard fact that the least worldly cardinal has cofinality $\omega$ and hence is strictly less than the least inaccessible cardinal, assuming that some inaccessible cardinals exist. Thus, any inner model, or truncations thereof, must contain a worldly cardinal and cannot be a model of $\IMH$. From this fact, Barton suggested that the large cardinal axioms are more ``restrictive'' than his $\CIMH$.

The intuitive picture is that, while classical arguments like G\"odel's focus on reflection principles depend on the intuition that, in an iterative hierarchy of sets, the power set operation is performed as many times as possible, the width absoluteness represented by $\CIMH$ demonstrated the alternative emphasis that at each ordinal $\alpha \in \On$, the stage $V_\alpha$ should include as many sets as possible. In this context, a universe with large cardinals is too ``thin'' and needs additional elements in order to accomodate an inner model satisfying $\CIMH$, while conversely the axiom $\CIMH$ describes a ``wide enough'' universe where one can already find inner models for large cardinal hypotheses.

\subsubsection*{Philosophical implications}

I believe that Barton has presented a very strong point for an anti-pluralist to commit to $\CIMH$ instead of large cardinal axioms, despite his universe being much less popular and may as well be more impractical to work with due to the constant need to move to a context of some inner models. Therefore, I believe that a comparison between Barton's and the classical arguments of an anti-pluralist reveal an inherent weakness in the mataphysical ontology of a single universe of sets: the intuitive analysis of the ``set of'' operation, which G\"odel has sought hard after to provide intrinsic justifications for both existing and new axioms of set thery, is bound to be ambiguous and even incoherent between different perspectives.

Ultimately, one has to justify why, in establishing the current axiomatic set theory with large cardinals, he must accept some simplistic principles but reject other ones that follow from similar intuitions, but are incompatible or inconsistent. Given the examples I presented in this section, I feel it is hardly convincing to suggest that there is some theoretical reasons at work, as compared to concluding that the considerations are made to produce an interesting, consequential and interconnected theory of sets. In other words, they are pragmatic arguments in a classical perspective of truth.

The difficulties are dissolved in a Carnapian viewpoint where each distinct axiom system is considered a different language of mathematics. Carnap himself may not feel this need to propose multiple linguistic frameworks for mathematical entities like sets, but his constructions are definitely potent for such a purpose, as the new evidence that I surveyed above forces one into such a position. Given this plurality, deciding between the systems with different logical truths, that is, theorems in a mathematical language, is an entirely artificial matter for Carnap, where purely practical arguments are assessed. A set theorist's claim ``assume there is a specific large cardinal'', which I have repeated invoked when stating mathemtical results in this section, will mean precisely that the language speaking of sets in a certain way, such that users of the language would assent to these claims that such large cardinals exist, is the best for describing the work I currently examine.

Admittedly, what I have presented in this section is primarily focused around the anti-pluralist position and, instead of going for a detour of reviving the Carnapian project, one can blatantly maintain a realist pluralist's view and suggest that for any potential axiom system this essay has so far considered --- that has sufficient evidence for its consistency --- there is a collection of mathematical entities in the ontology satisfying the axioms. In addition to the fact I mentioned in \autoref{subsec:large-cardinal-independence} that this is a literal reading of model-theoretic claims about set theory, more and more set theorists are indeed adopting the methodologies of model theory, due to the prevalence of forcing and inner model theory in recent years. This is evident even in the few mathematical results and proofs I looked at in this section.

Therefore, as has been suggested at the beginning of this essay, there have been strong reasons to support pluralism in the philosophy of set theory. However, I believe that it is also important to account for the practices of the other branches of mathematics where the intuition favours anti-pluralism much more. A Carnapian view explains the anti-pluralist tradition while maintaining pluralism implicitly for any potential alternative axiomatisations in set theory. The next section will provide this account from another branch of mathematics.

\section{Mathematicians' perspectives}
\label{sec:mathematicians-perspectives}

\subsection{The projective hierarchy and determinacy}
\label{subsec:determinacy}

In this section, I will retract the focus to only the real numbers. On the contrary to the case for extremely large cardinals in set theory, there is much more intuition for the collection\footnote{For simplicity, many set-theoretic results are instead proven on the Baire space of functions $\tensor[^\omega]{\omega}{}$, known as the ``set theorists' reals''. However, the treatments will be essentially the same for $\tensor[^\omega]{\omega}{}$ and the standard set of reals. Therefore, I will always use the notation $\mathbb{R}$ in this section in order to relate to philosophical intuitions.} $\mathbb{R}$ as the unique family of numbers serving as all possible coordinates on a linear axis or the unique completion of a number system of integer fractions through Dedekind cuts. However, it has been discovered that there are natural questions about $\mathbb{R}$ that cannot be answered unless one clarifies his commitment to the large cardinals. Koellner's expository \cite{koellner-determinacy} in the \emph{Stanford Encyclopedia of Philosophy} serves as an excellent survey on this matter. I will cite the important stipulations and results and argue that such development refutes a pluralist position in the philosophy of mathematics.

To begin, for a set $A \subseteq \mathbb{R}$ of reals, one is interested here in the following three \emph{regularity properties} representing how well-behaved $A$ is:

\begin{definition}[The perfect set property]
	A set $A \subseteq \mathbb{R}$ is \emph{perfect} if it is non-empty, closed and contains no isolated points.

	A set $A \subseteq \mathbb{R}$ \emph{has the perfect set property} if it is countable or contains a perfect subset.
\end{definition}

A set with the perfect set property is thought to be well-behaved due to the following observation:

\begin{proposition}
	If $A \subseteq \mathbb{R}$ is perfect, then $\left|A\right| = 2^\omega$.
\end{proposition}

\begin{proof}
	Suppose $\left|A\right| < 2^\omega$, then $A$ cannot contain any intervals. If $A$ is also perfect, one can find $a, b \not\in A$ such that $\inf A < a < b < \sup A$ and it follows that $A \cap \left(-\infty, a\right]$ and $A \cap \left[b,\infty\right)$ are still perfect.

	Recursively define functions $F_n : \left\{0, 1\right\}^n \rightarrow \mathcal{P}\mleft(A\mright)$ such that $F_1\mleft(0\mright)$ and $F_1\mleft(1\mright)$ are two disjoint perfect subsets of $A$ as constructed above and, similarly, $F_{n + 1}\mleft(\mathbf{v}, 0\mright)$ and $F_{n + 1}\mleft(\mathbf{v}, 1\mright)$ are two disjoint perfect subsets of $F_n\mleft(\mathbf{v}\mright)$ for any $n \in \omega$ and $n$-tuple $\mathbf{v} \in \left\{0, 1\right\}^n$. For any function $f : \omega \rightarrow \left\{0, 1\right\}$, it follows that
	\[F_1\mleft(f\mleft(0\mright)\mright) \supseteq \cdots \supseteq F_n\mleft(f\mleft(0\mright), \ldots, f\mleft(n - 1\mright)\mright) \supseteq \cdots\]
	is a decreasing chain of closed sets and thus has a non-empty intersection. Then, the function $\rho : \tensor[^\omega]{\left\{0, 1\right\}}{} \rightarrow A$ satisfying
	\[\rho\mleft(f\mright) \in \bigcap_{n \in \omega} F_n\mleft(f\mleft(0\mright), \ldots, f\mleft(n - 1\mright)\mright)\]
	is an injection, thus $A$ cannot have cardinality less than $2^\omega$.
\end{proof}

Therefore, the cardinalities of the subsets of $\mathbb{R}$ that have the perfect set property are either finite, $\omega$ or $2^\omega$. This is a construction used in the Cantor--Bendixson theorem that all uncountable closed sets of reals have the size of the continuum and do not serve as counterexamples to $\CH$.

The other two regularity properties arise more naturally from topology and real analysis:

\begin{definition}[The Baire property]
	A set $A \subseteq \mathbb{R}$ is \emph{nowhere dense} if the interior of its closure is empty; it is \emph{meager} if it is a countable union of nowhere dense sets.

	A set $A \subseteq \mathbb{R}$ \emph{has the Baire property} if there exists an open subset $U \subseteq \mathbb{R}$ such that the symmetric difference
	\[A \mathop{\triangle} U = \left(A \setminus U\right) \cup \left(U \setminus A\right)\]
	is meager.
\end{definition}

\begin{definition}[Lebesgue measurability]
	A set $A \subseteq \mathbb{R}$ is \emph{Lebesgue measurable} if there exists a Borel set\footnote{Here, I will omit a detailed treatment of measures, Borel sets and null sets that can be found in any introductory real analysis textbook. One way of defining Borel sets, which is particularly related to the discussion here, will be given later in \autoref{def:borel-hierarchy} though.} $B \subseteq \mathbb{R}$ such that the symmetric difference $A \mathop{\triangle} B$ is null.
\end{definition}

While it is desirable for any set of reals to be well-behaved, that is, to exhibit the aforementioned regularity properties, it is a standard result in real analysis that, assuming $\ZFC$, counterexamples can be found for any of the three properties. Therefore, research has specifically focused on the following increasing hierarchies\footnote{The following constructions are often called the \emph{boldface} hierarchies, in contrast to an alternative family of pointclasses named the lightface hierarchies. However, the lightface hierarchies will not be covered or used in the discussions in this essay.}

\begin{definition}[Pointclasses]
	A \emph{pointclass} $\Gamma$ denotes a definite way to select subsets of any Polish space. Formally, a \emph{pointclass} is a class of tuples $\left\langle A, X \right\rangle$ where $X$ is a Polish space and $A \subseteq X$. When a Polish space $X$ has been assumed, one says that some subset $A \subseteq X$ lies in the pointclass $\Gamma$ if $\left\langle A, X \right\rangle \in \Gamma$.

	For the purpose of the discussion here, one can simply view a pointclass as a set of subsets of $\mathbb{R}$ (or $\mathbb{R}^n$, if appropriate).
\end{definition}

\begin{definition}[The Borel hierarchy]
	\label{def:borel-hierarchy}
	Let $\boldsymbol\Sigma^0_1$ denote the pointclass of open sets and $\boldsymbol\Pi^0_1$ denote the pointclass of closed sets. For each ordinal $1 < \alpha < \omega_1$, let $\boldsymbol\Sigma^0_\alpha$ denote the sets that are countable unions of sets in $\bigcup_{1 \leq \beta < \alpha} \boldsymbol\Pi^0_\beta$ and $\boldsymbol\Pi^0_\alpha$ denote the sets that are countable unions of sets in $\bigcup_{1 \leq \beta < \alpha} \boldsymbol\Sigma^0_\beta$. Let $\boldsymbol\Delta^0_\alpha$ denote the intersection $\boldsymbol\Sigma^0_\alpha \cap \boldsymbol\Pi^0_\alpha$. The pointclasses $\boldsymbol\Sigma^0_\alpha, \boldsymbol\Pi^0_\alpha, \boldsymbol\Delta^0_\alpha$ form the \emph{Borel hierarchy}.

	The unions
	\[\bigcup_{1 \leq \alpha < \omega_1} \boldsymbol\Sigma^0_\alpha = \bigcup_{1 \leq \alpha < \omega_1} \boldsymbol\Pi^0_\alpha\]
	are equal by definition, and form the pointclass of \emph{Borel sets}.
\end{definition}

\begin{definition}[The projective hierarchy]
	Consider some subset $A \subseteq \mathbb{R}^n$ where $n > 1$, the \emph{projection} of $A$ is the set
	\[\left\{\mathbf{v} \in \mathbb{R}^{n - 1} : \text{$\left\langle \mathbf{v}, x \right\rangle \in A$ for some $x \in \mathbb{R}$}\right\}.\]

	Let $\boldsymbol\Sigma^1_0$ denote the pointclass of open sets and $\boldsymbol\Pi^1_0$ denote the pointclass of closed sets. For each number $1 \leq n < \omega$, let $\boldsymbol\Sigma^0_n$ denote the sets that are projections of sets in $\boldsymbol\Pi^1_{n - 1}$ and $\boldsymbol\Pi^1_n$ denote the sets that are complements of sets in $\boldsymbol\Sigma^1_n$. Let $\boldsymbol\Delta^1_n$ denote the intersection $\boldsymbol\Sigma^1_n \cap \boldsymbol\Pi^1_n$. The pointclasses $\boldsymbol\Sigma^1_n, \boldsymbol\Pi^1_n, \boldsymbol\Delta^1_n$ form the \emph{projective hierarchy}.

	The unions
	\[\bigcup_{1 \leq n < \omega} \boldsymbol\Sigma^1_n = \bigcup_{1 \leq n < \omega} \boldsymbol\Pi^1_n\]
	are equal by definition, and form the pointclass of \emph{projective sets}.
\end{definition}

Notice that, for $n \geq 1$, the pointclasses on the projective hierarchy are closed under countable unions and intersections. It follows that $\boldsymbol\Delta^1_1$ contains all the Borel sets. Therefore, as the numbering suggested, the projective hierarchy is an extension of the Borel hierarchy.

The overarching collection of subsets I will consider in this section is $L\mleft(\mathbb{R}\mright)$, defined in a similar manner to G\"odel's constructive universe $L$, following convention in the footnote after definition 1.6 in Mitchell's introductory expoition \cite{mitchell-inner-model-theory}:

\begin{definition}
	Define sets $L_\alpha\mleft(\mathbb{R}\mright)$ recursively as
	\begin{align*}
		L_0\mleft(\mathbb{R}\mright)            & = V_{\omega + 1},                                                \\
		L_{\alpha + 1}\mleft(\mathbb{R}\mright) & = \Def\mleft(L_\alpha\mleft(\mathbb{R}\mright), \in\mright),     \\
		L_\lambda\mleft(\mathbb{R}\mright)      & = \bigcup_{\alpha \in \lambda} L_\alpha\mleft(\mathbb{R}\mright)
	\end{align*}
	where $V_{\omega + 1}$ is the lowest stage in the standard cumulative hierarchy that contains $\mathbb{R}$, if $\mathbb{R}$ is understood as the set $\tensor[^\omega]{\omega}{}$ of functions on $\omega$. Then $L\mleft(\mathbb{R}\mright) = \bigcup_{\alpha \in \On} L_\alpha\mleft(\mathbb{R}\mright)$.

	When one refers to $L\mleft(\mathbb{R}\mright)$ as a pointclass, the $L\mleft(\mathbb{R}\mright)$-subsets of $\mathbb{R}^n$ are exactly the subsets of $\mathbb{R}^n$ that fall in $L\mleft(\mathbb{R}\mright)$.
\end{definition}

Notice that any open subset of $\mathbb{R}^n$ can be written as a countable union of open discs, each of which can be described using two reals. Since a real can easily be designed to code information for countably many reals, any open sub is definable in $\ZFC$ using a real parameter, hence in $L\mleft(\mathbb{R}\mright)$. $L\mleft(\mathbb{R}\mright)$ is subsequently closed under projection and complements, thus contains all projective sets. It follows that $L\mleft(\mathbb{R}\mright)$ is indeed containing every pointclass I have introduced so far.

Around the beginning of the twentieth century, analysts attempted to ``climb the hierarchy'' and determine the range of subsets of $\mathbb{R}$ that exhibit the regularity properties. The pinnacle of the developments was the proofs in Luzin and Suslin's 1917 publications \cite{luzin-analytic-sets} and \cite{suslin-analytic-sets} that

\begin{theorem}
	\label{thm:analytic-set-regularity}
	Assume $\ZFC$. Then $\boldsymbol\Sigma^1_1$-sets have the perfect set property, the Baire property, and are Lebesgue measurable.
\end{theorem}

It is reasonable to expect sets higher up in the projective hierarchy to display these well-behaving properties. In fact, as Solovay worked out later in \cite{solovay-force-regularity} with set-theoretic techniques, assuming the existence of an inaccessible cardinal, one can force a model of $\ZF$ where every subset of $\mathbb{R}$ has all three regularity properties. However, \autoref{thm:analytic-set-regularity} is roughly the furthest one can go in this direction within $\ZFC$ because, as G\"odel first announced\footnote{Kanamori noted that explicit proofs for the results on the Baire property and the perfect set property are not found in G\"odel's published papers, though. Details of these constructions can be found in section 13 of Kanamori's book \cite{higher-infinite}.},

\begin{theorem}
	Assume $\ZFC + V = L$, then there is a $\boldsymbol\Delta^1_2$-set which does not have the Baire property and is not Lebesgue measurable. There is also a $\boldsymbol\Pi^1_1$-set which does not have the perfect set property.
\end{theorem}

\subsubsection*{Axioms of determinacy}

To proceed and explain how later work on Woodin cardinals led to significant results on the regularity properties, I will first provide the definition of a new tool: the determinacy problem.

\begin{definition}
	Let $A \subseteq \tensor*[^\omega]{\omega}{}$ be a subset\footnote{For simplicity, I will provide the formal definition of a game on $\tensor*[^\omega]{\omega}{}$ instead of the standard $\mathbb{R}$. The procedure can easily be moved onto $\mathbb{R}$ though, by looking at, for example, the following injection $\xi : \tensor*[^\omega]{\omega}{} \rightarrow \left[0, 1\right]$:
		\[\xi\mleft(f\mright) = \sum_{i = 0}^\infty \left(1 - 2^{-f\mleft(i\mright)}\right) \prod_{j = 0}^{i - 1} 2^{-f\mleft(j\mright) - 1}.\]}. $G_\omega\mleft(A\mright)$ shall denote an ideal two-player game with perfect information. The game will last $\omega$ rounds where, at round $2n$, the first player will choose a number $x\mleft(2n\mright) \in \omega$ while, at round $2n + 1$, the second player will choose a number $x\mleft(2n + 1\mright) \in \omega$. After the $\omega$ rounds are completed, the game is won by the first player if $x \in A$ and otherwise won by the second player.

	A set $A \subseteq \tensor*[^\omega]{\omega}{}$ is \emph{determined} if, in the game $G_\omega\mleft(A\mright)$, either the first player or the second player has a winning strategy.
\end{definition}

According to Kanamori, Gale and Stewart first analysed games of such a form in \cite{gale-stewart-infinite-games}. I will mention some introductory results that suggest determinacy is similar to the regularity properties, that it should hold for the pointclass at lower levels on the projective hierarchy, but may fail higher up in the hierarchy:

\begin{proposition}
	Any open or closed sets are determined.
\end{proposition}

\begin{proof}
	Observe that, for any initial segment $\left.s\right|_{<2n}$, the first player does not have a winning strategy here if and only if, for any $s\mleft(2n\mright) \in \omega$ the first player selects, the second player has some choice $s\mleft(2n + 1\mright) \in \omega$ such that the first player still has no winning strategy for the initial segment $\left.s\right|_{<2n + 2}$.

	Now, let $A \subseteq \tensor*[^\omega]{\omega}{}$ be open and suppose that the first player does not have a winning strategy. Then the second player has a strategy where he always selects some $s\mleft(2n + 1\mright)$ such that the first player still has no winning strategy for the initial segment $\left.s\right|_{<2n + 2}$. I claim that the outcome $s \not\in A$:

	Otherwise, $s \in A$. Since $A$ is open, there is some $n \in \omega$ that any element in $\tensor*[^\omega]{\omega}{}$ with an initial segment $\left.s\right|_{<n}$ lies in $A$. Then the first player has a winning strategy after the $n^\text{th}$ round. Contradiction.

	Therefore, if the first player does not have a winning strategy, then the second player has one. The case where $A$ is closed is analogous by switching the two players.
\end{proof}

\begin{proposition}
	There exists a subset of $\tensor*[^\omega]{\omega}{}$ that is not determined.
\end{proposition}

\begin{proof}
	A strategy simply consist of one function $f_n : \tensor*[^\omega]{\left\{0, \ldots, n - 1\right\}}{} \rightarrow \omega$ for each $n \in \omega$. Therefore, there are $2^\omega$ strategies. Let $\left\{\tau_\alpha : \alpha < 2^\omega\right\}$ enumerate the strategies and define $a_\alpha, b_\alpha \in \tensor*[^\omega]{\omega}{}$ for each $\alpha < 2^\omega$ recursively, such that each $b_\alpha$ is a possible game outcome when the first player is using strategy $\tau_\alpha$, yet $b_\alpha \not\in \mleft\{a_\beta : \beta < \alpha\mright\}$; each $a_\alpha$ is a possible game outcome when the second player is using strategy $\tau_\alpha$, yet $a_\alpha \not\in \mleft\{b_\beta : \beta < \alpha\mright\}$. This construction is possible because, when one player fixed his strategy, the game still has $2^\omega$ possible outcomes.

	Now, the sets $A = \left\{a_\alpha : \alpha < 2^\omega\right\}$ and $B = \left\{b_\alpha : \alpha < 2^\omega\right\}$ are disjoint. Yet, for any strategy chosen by the first player, it is possible that the outcome lies in $B$; for any strategy chosen by the second player, it is possible that the outcome lies in $A$. Neither has a winning strategy for the game $G_\omega\mleft(A\mright)$.
\end{proof}

As summarised in section 27 of Kanamori's book \cite{higher-infinite}, slightly modified versions of the infinite game can be used to identify meager sets, perfect subsets and Lebesgue measurable subsets. The precise constructions will be omitted here. However, as laid out in section 6A of Moschovakis' book \cite{descriptive-set-theory}, the process effectively leads to the result that

\begin{theorem}
	Let $\Gamma$ be an adequate pointclass closed under Borel substitution. Suppose that every set in $\Gamma$ is determined, then every set in $\Gamma$ has the perfect set property, the Baire property, and is Lebesgue measurable\footnote{In the case where $\Gamma = \boldsymbol\Pi^1_n$, Kechris and Martin proved that the result can be improved to show that $\boldsymbol\Sigma^1_{n + 1}$-sets have the regularity properties. See section 27 in Kanamori's book \cite{higher-infinite} for a discussion of this.}.
\end{theorem}

Here, adequacy refers to the technical conditions of containing all recursive pointsets and being closed under recursive substitution, $\&$, $\lor$, $\exists^\leq$ and $\forall^\leq$, as Moschovakis defined in section 3E of \cite{descriptive-set-theory}. Details aside, $\boldsymbol\Sigma^1_n, \boldsymbol\Pi^1_n, \boldsymbol\Delta^1_n$, the projective sets and $L\mleft(\mathbb{R}\mright)$ are all adequate pointclasses where the theorem above applies.

Therefore, instead of dealing directly with the regularity properties, one can ask whether the following hypotheses hold in some set theory:

\begin{definition}
	The \emph{axioms of determinacy} refers to axioms of the form ``all $\Gamma$-subsets of $\mathbb{R}$ are determined'' for a pointclass $\Gamma$. This is often abbreviated as $\Gamma$-determinacy.

	I will follow convention and write $\PD$ for the axiom schema $\boldsymbol\Pi^1_n$-determinacy for every natural number constant $n$; I write $\AD^{L\mleft(\mathbb{R}\mright)}$ for $L\mleft(\mathbb{R}\mright)$-determinacy.
\end{definition}

Then, following the projective hierarchy, one can write a hierarchy of new axioms for set theory,
\[\text{$\boldsymbol\Pi^1_1$-determinacy}, \text{$\boldsymbol\Pi^1_2$-determinacy}, \ldots, \PD, \AD^{L\mleft(\mathbb{R}\mright)}.\]
It turns out that the hierarchy of axioms of determinacy is tightly connected to the large cardinal hierarchy. As Donald Martin proved\footnote{In the original paper, Martin proved that analytic sets are determined, that is, $\boldsymbol\Sigma^1_1$-determinacy. However, a set is determined if and only if its complement is determined, thus I state the more useful result $\boldsymbol\Pi^1_1$-determinacy here.} in 1970 in \cite{martin-measurable-cardinal-determinacy},

\begin{theorem}
	Assume $\ZFC$. If there exists a measurable cardinal, then $\boldsymbol\Pi^1_1$-determinacy holds.
\end{theorem}

It follows that, assuming the existence of a measurable cardinal, every $\boldsymbol\Sigma^1_2$-subset of $\mathbb{R}$ has the perfect set property, the Baire property, and is Lebesgue measurable. This result is later strengthen to the following in \cite{woodin-cardinal-to-pd}:

\begin{theorem}
	Assume $\ZFC$. For any natural number $n$, if there exist $n$ Woodin cardinals with a measurable above them, then $\boldsymbol\Pi^1_{n + 1}$-determinacy holds.

	If there exist $\omega$ Woodin cardinals with a measurable above them, then $\AD^{L\mleft(\mathbb{R}\mright)}$ holds.
\end{theorem}

It must be mentioned here that the hierarchy of determinacy is not only implied by the existence of sufficiently large cardinals, but further equivalent to the existence of such cardinals, albeit in inner models. As enumerated in section 8 of the textbook chapter \cite{large-cardinals-from-determinacy} by Koellner and Woodin in the \emph{Handbook of Set Theory}, there is the following sequence of results:

\begin{theorem}
	\label{thm:delta-1-2-determinacy-equivalence}
	Assume $\ZFC$. The following are equivalent:
	\begin{itemize}
		\item $\boldsymbol\Delta^1_2$-determinacy;
		\item for every real $x \in \mathbb{R}$, there is an inner model $M$ such that $x \in M$ and $M \vDash \text{there is a Woodin cardinal}$.
	\end{itemize}
\end{theorem}

\begin{theorem}
	Assume $\ZFC$. The following are equivalent:
	\begin{itemize}
		\item $\PD$;
		\item for every $n < \omega$, there is a fine-structural, countably iterable inner
		      model $M$ such that $M \vDash \text{there are $n$ Woodin cardinals}$.
	\end{itemize}
\end{theorem}

\begin{theorem}
	\label{thm:adlr-determinacy-equivalence}
	Assume $\ZFC$. The following are equivalent:
	\begin{itemize}
		\item $\AD^{L\mleft(\mathbb{R}\mright)}$;
		\item in $L\mleft(\mathbb{R}\mright)$, for every set $S$ of ordinals, there is an inner model $M$ and a countable ordinal $\alpha < \omega_1^{L\mleft(\mathbb{R}\mright)}$ such that $S \in M$ and $M \vDash \text{$\alpha$ is a Woodin}$ cardinal.
		      % CHECK: Exit early from math-modes to allow line wraps.
	\end{itemize}
\end{theorem}

Therefore, it has to be admitted, based on the aforementioned mathematical facts, that the regularity behaviours of the subsets of reals, determinacy and the consistency of large cardinal axioms are heavily intertwined matters. By accepting or rejecting different large cardinal axioms, it is possible that the ontology of sets one commits to contains significantly different collections of ``subsets of reals''.

While this may not be a problem for some pluralists, such as a model theorist who is used to working with even different collection of ``reals'' satisfying the first-order theory of real closed fields, like the hyperreal and surreal numbers, it is still a failure in a philosophy of mathematics if it is incapable of explaining the na\"ive notion of reals in daily life and in the work of many more mathematicians who are not bothered by these properties linked to the large cardinals.

In other words, in the previous sections, a pluralist is able dismiss the weak philosophical case for anti-pluralism regarding the large cardinals, primarily due to the fact that these entities seem very distant from our intuitive conception of sets and it is doubtful whether, from such intuition, one can form a definite picture about any large cardinals. However, when the discussion has been brought to the reals, there is suddenly much stronger support for an anti-pluralist viewpoint because it feels much easier to have some intuition about the ``totality of sets of reals'' and every element of this totality must either possess the regularity properties or not possess them, in a pre-determined way, no matter whether they are Borel, projective or neither. It is more difficult to make sense of the pluralist's claim that there exist distinct collections of abstract entities, each of which fits the concept ``the totality of sets'' equally, when the distinction is applied to the reals alike.

\subsection{Friedman's counting propositions}

Admittedly, it can be argued that, similar to how the na\"ive conception of sets has been blatantly paradoxical, one's intuition about the reals may not be reliable as well. Such an attack can be framed, for example, in an intuitionist flavour, based on the limited power of the mind to produce a constructive understanding of mathematical entities. Since the concerns of the regularity properties are entirely about the non-enumerable collections of reals, humans' intuition about them can be incorrect and it is possible that there is, in fact, no unique totality of such sets.

However, in the paper \emph{Finite functions and the necessary use of large cardinals} (\cite{friedman-finite-independence}), Harvey Friedman has proved some more surprising independence results, in his project to search for ``concrete mathematical incompleteness'' in contemporary formal systems. I will explain his results:

For the definitions below, when $y \in \omega^k$ is a $k$-tuple of numbers, the norm
\[\left|y\right| = \max_{1 \leq i \leq k} y_i\]
is read as the supremum norm.

\begin{definition}
	Let $k, r$ be positive numbers and $B \subseteq A \subseteq \omega^k$. For some function $f : A \rightarrow \omega^r$ and $y \in \omega^r$, $y$ is a \emph{regressive value} of $f$ on $B$ if there exists some $x \in B$ such that $y = f\mleft(x\mright)$ and $\left|y\right| < \min_{1 \leq i \leq k} x_i$.
\end{definition}

\begin{definition}
	Let $X$ be a set. A \emph{function assignment} $U$ assigns a unique function $U\mleft(A\mright) : A \rightarrow A$ to each finite subset $A \subseteq X$.

	Let $U$ be a function assignment on $N^k$, where $N \subseteq \omega$. $U$ is \emph{\#-decreasing} if, for any finite $A \subseteq N^k$ and $x \in N^k$, either $U\mleft(A\mright) = \left.U\mleft(A \cup \left\{x\right\}\mright)\right|_A$ or there exists some $y \in A$ such that $\left|y\right| > \left|x\right|$ and
	\[\left|U\mleft(A\mright)\mleft(y\mright)\right| > \left|U\mleft(A \cup \left\{x\right\}\mright)\mleft(y\mright)\right|.\]
\end{definition}

Then, the following statement is labelled by Friedman as \emph{Proposition B}:

\begin{hypothesis*}
	Let $n \gg k$, $p > 0$ and $U$ be a \#-decreasing function assignment for $\left\{0, \ldots, n - 1\right\}^k$. Then some $U\mleft(A\mright)$ has $\leq k^k$ regressive values on some $E^k \subseteq A$, where $\left|E\right| = p$.
\end{hypothesis*}

By $n \gg k$, Friedman meant that, for any positive number $k$, there exists $m > k$ such that, for any $n > m$, the subsequent claim holds for the given $n, k$. Friedman also listed similar hypotheses A, C and D but, as he later showed, these propositions indeed have roughly the same strength, compared to the large cardinal hierarchy. Therefore, I will only use Propsition B as an example here.

Using his complicated stipulations, Friedman was able to prove that his propositions are independent of $\ZFC$. More precisely, he proved that

\begin{theorem}
	Assume that, for any $k \in \omega$, there exists a $k$-subtle cardinal. Then Proposition B holds.
\end{theorem}

\begin{theorem}
	\label{thm:friedman-proposition-equivalent-to-consistency-of-subtle-cardinals}
	Assume Proposition B. Then there is a model of $\ZFC$ that interprets the set of axioms
	\[\left\{\text{there exists a $n$-subtle cardinal} : \text{$n$ is a natural number constant}\right\}.\]
\end{theorem}

Here, the large cardinal conditions involved were first defined by Baumgartner in \cite{ineffable-cardinals}:

\begin{definition}[$k$-subtle cardinals]
	Let $k$ be a positive integer. For some set $X$, let $S_k\mleft(X\mright)$ denote the set of subsets of $X$ with size $k$. For some cardinal $\lambda \in \Card$, a function $f : S_k\mleft(\lambda\mright) \rightarrow \mathcal{P}\mleft(\lambda\mright)$ is \emph{regressive} if for all $A \in S_k\mleft(\lambda\mright)$, either $f\mleft(A\mright) < \min\mleft(A\mright)$ or $\min\mleft(A\mright) = 0$. A subset $E \subseteq \lambda$ is \emph{$f$-homogeneous} if, for
	any $C, D \in S_k\mleft(E\mright)$,
	\[f\mleft(C\mright) \cap \min\mleft(C \cup D\mright) = f\mleft(D\mright) \cap \min\mleft(C \cup D\mright).\]

	Then, a cardinal $\kappa \in \Card$ is \emph{$k$-subtle} if it is infinite and, for any club $C \subseteq \kappa$ and regressive $f : S_k\mleft(\kappa\mright) \rightarrow \mathcal{P}\mleft(\kappa\mright)$, there exists an $f$-homogeneous $A \in S_{k + 1}\mleft(C\mright)$.
\end{definition}

Proofs for many elementary results about $k$-subtle cardinals can be found explicitly in section 1 of Friedman's paper \cite{subtle-cardinals}. Effectively, the $k$-subtle cardinals will insert into the hierarchy of large cardinals I have introduced so far in the following manner:
\[\ldots, \text{Mahlo}, \text{$1$-subtle}, \ldots, \text{$k$-subtle}, \ldots, \kappa\mleft(\omega\mright), \ldots, \text{measurable}, \ldots\]
They are small large cardinals below the $\omega$-Erd\H{o}s cardinal.

While Friedman's hypothesis does not involve large cardinals as large as the ones used to resolve the determinacy problems, his result is notable due to fact that the entire Proposition B only concerns subsets and multivariate functions on $\omega$. Therefore, these statements, as he has hoped\footnote{See section 5 of Friedman's article in \cite{new-axioms-panel}.}, suggest compelling involvement of large cardinals in infinitary discrete mathematics, one of the most concrete branches of mathematics.

It must then be admitted that, if one takes a pluralist approach on the extensions of $\ZFC$ that proves and rejects the consistency of $k$-subtle cardinals respectively --- models of the former should exist due to the overwhelming studies that support the consistency of large cardinals, while models of the latter are guaranteed by G\"odel's second incompleteness theorem --- then, according to Friedman's research, the two theories generate distinct senses for the concept of multivariate integer-valued functions and proves incompatible properties for them. This is simply contradictory to the commonsensical understanding of numbers and functions on numbers.

Therefore, I argue that the observations of independence in various areas of mathematics constitute a strong attack on a pluralist philosophy of mathematics. While accepting a multitude of set-theoretic universes can sweep away disputes surrounding the choice between axiomatisations, losing the ability to pinpoint the \emph{unique} collection of all natural numbers, all functions between sets of integers or all subsets of reals in one's ontology of abstract entities seems too high a price to pay.

I shall suggest that, a Carnapian viewpoint is still implicitly plural and serves as a replacement for a full-blown anti-pluralism to respond to this attack from the mainstream branches of mathematics. The Carnapian philosophy provides an internal environment where, after adopting a fixed formal system as the underlying language, one is free to assert an anti-pluralist claim without worries of overlooking other models that potentially exist. For example, in an analysis textbook where the context is specified to be a language of $\ZFC$ plus a fixed identification of the real numbers as sets, perhaps through Dedekind cuts, then the author is free to refer to ``the reals'' or ``the subsets of $\mathbb{R}$'', fully justified to ignore the fact that there is an indeterminate choice between including only projective sets that have regularity properties and admitting ones that do not.

Meanwhile, the theory of linguistic frameworks is able to respond to the discussions in set theory that I have examined in the previous section as well. While the specific language of $\ZFC$ is used in the majority of mathematical discussions, one can still acknowledge the desirable practical features of large cardinal axioms and assume them whenever necessary by maintaining an implicit pluralist stance and switching between different linguistic frameworks.

Therefore, I believe that a Carnapian philosophy acts as a middle-ground between pluralism and anti-pluralism regarding the metaphysical disputes on abstract entities. In the case of set theory, it is a ready option to dissolve the prominent worries of both parties.

\section{Reflections on the modern philosophy of mathematics}

\subsection{Naturalism}

In the end, I will look at some positions of well-known philosophers of mathematics in recent years and their attempts at responding to some of the problems I have mentioned in the thesis. I will especially focus on Penelope Maddy's naturalist philosophy of mathematics and argue that the Carnapian viewpoint is, in addition, a natural interpretation of contemporary mathematical practices.

Now, Maddy's naturalism is an analogy in mathematics of a Quinean philosophy. Where Quine enforced that philosophical problems shall be tackled and justified through the same standards and practices as the natural sciences, Maddy took it one step further and asserted in her book \emph{Naturalism in Mathematics} (\cite{maddy-naturalism}) that ``a successful enterprise, be it science or mathematics, should be understood and evaluated on its own terms''. Specifically, for mathematics, Maddy maintained that

\begin{quote}
	the idea, then, is that set theoretic practice in particular, and mathematical practice in general, are not in need of justification from philosophical quarters. Justification, on this view, comes from within, couched in simple terms of what means are most effective for meeting the relevant mathematical ends.
\end{quote}

I will use the panel discussion \cite{new-axioms-panel}, where the quote above came from, as a primary background of my comments here. I believe that it was through the exchange of the four participants on the panel that a naturalist position on mathematics was molded into how it should be.

Maddy herself is a proponent of the large cardinal axioms and believes that the inclusion of these axioms helps achieve some set-theoretic goals. Yet Maddy has not made clear in her presentation what these goals are or why they are evident in the mathematical community other than acknowledging that set theory aspires to lay a foundation for all other areas of mathematics. Thus, in response to her claim that commiting to the existence of many Woodin cardinals provided a solution that ``set theorists had sought for decades'' --- through the results I have surveyed in \autoref{subsec:determinacy} --- Feferman asked why ``the `good' properties of Borel and analytic sets should generalize to all projective sets, given that they don't hold for all sets?'' It is indeed unclear why one should view the affirmation of some propositions already shown to be independent of the existing theory as a major goal of a branch of mathematics.

It can be argued that Steel's reference to Maddy's own maxim ``\emph{maximize}'' provides a better account of a set theorist's acceptance of the large cardinals. In her essay \cite{maddy-maximize}, Maddy characterised this goal as that

\begin{quote}
	the set theoretic arena in which mathematics is to be modelled should be as generous as possible; the set theoretic axioms from which mathematical theorems are to be proved should be as powerful and fruitful as possible.
\end{quote}

The large cardinal axioms are undoubtedly successful candidates to fulfill this purpose. As I have demonstrated in \autoref{subsec:large-cardinal-independence}, one can obtain progressively stronger theories in terms of interpretability by incorporating large cardinal axioms higher and higher on the hierarchy. Meanwhile, strong assumptions in other areas of mathematics, such as the axioms of determinacy I examined in \autoref{subsec:determinacy}, are often also provable or interpretable in a theory with sufficiently large cardinals, making such theories a very powerful context for all kinds of mathematical discussions.

In fact, as Maddy has been explicit in her article in \cite{new-axioms-panel}, unlike G\"odel's classical emphasis on intrinsic evidences and justification, a naturalist welcomes extrinsic reasons that connect mathematical axioms to the underlying purposes of a subject. Therefore, the evidence Koellner has provided in recent years for the axioms of Woodin cardinals and determinacy of $L\mleft(\mathbb{R}\mright)$, for example in section 4 of his expository \cite{koellner-pluralism}, counts as solid support for these strong hypotheses in Maddy's framework. Koellner's focus is essentially on the tight connections between the existence of Woodin cardinals and other natural claims in various branches of mathematics and this fits nicely into the picture of a mutually explanatory body of mathematical theorems, completely analogous to Quine's ideal of a coherent web of knowledge across the natural sciences.

However, I believe that there is one crucial mistake in Maddy's version of naturalism. As Friedman pointed out in the panel, the viewpoint of a mathematician may be extremely different from that of a set theorist. Mathematics has developed for a few thousands of years before utilising set theory as a universal foundation and has its own practice and foci. Therefore, it is the recognisation by the general mathematician community that established $\ZFC$ as the ``current golden standard of rigor'', instead of solely through certain set-theoretic properties such as interpretability.

To be more precise, I believe that Maddy has misinterpreted the desirable ``mathematical ends'' as the resolution of open questions, through any means, in a well-behaved direction and the foundational goal of set theory as to simply provide a universe in which any existing mathematical result can be interpreted and proven. However, different branches of mathematics have their own ``code of conduct'' as in what to accept as a proof even before the spread of the set-theoretic techniques and it is key for a new foundation to be \emph{conservative} over this pre-existing system. For example, the proof of $\PD$ from the existence of infinitely many Woodin cardinals, and any other demonstration of mathematical results from large cardinal axioms I have mentioned in \autoref{sec:mathematicians-perspectives}, will, in general, not be received in the same way as Luzin and Suslin's 1917 proof of the weaker results using purely classical assumptions in the mathematical community. It is not a simple matter of pragmatic attraction to modify this standard practice of modern mathematics.

Therefore, while Maddy and other proponents of large cardinals can suggest that such a theory is to be studied in set-theoretic research due to the extrinsic plausibility of the axioms, it is still a long way before they can be accepted as ``axioms of mathematics''. A plausible naturalism in mathematics, as I believe, must acknowledge this fact that to reach the ``relevant mathematical ends'' is firstly to remain loyal to a presumed, rigorous practice, and then to seek the best results that can be established within the constraints.

Hence, it is apparent that one should recognise two distinctive contexts for a mathematical statement: one is to lay out the presumed practice, which includes, in a formal context, the rules of references and the axioms; the other one is for the establishment of the results, that is, the proofs. This corresponds precisely to the two ways one may speak of a Carnapian linguistic framework: either externally of the attributes and pragmaticality of the framework, or internally of the logical consequences of the framework.

I would thus argue that the Carnapian philosophy is a natural candidate for the philosophy of mathematics. It allows for a free choice of the linguistic framework in precisely the way one would see the choice between axiomatisations as entirely pragmatic and artificial and justifies exactly the mathematical statements --- including existential claims --- that most modern mathematicians would view as justified, through the availability of a proof.

\subsection{Consistency versus existence}

A Carnapian perspective of mathematics is an implicit form of pluralism in the way that all axioms are \emph{structural axioms} in Feferman's sense in \cite{feferman-new-axioms}, that is, ``[they] are simply definitions of kinds of structures which have been recognized to recur in various mathematical situations''. While Feferman identified the debate on the axioms of set theory as concerning the \emph{foundational axioms}, which should be laws, axioms and constructions that ``underlie all mathematical concepts'', a Carnapian philosopher recognises none. In other words, there is not one unique language in which all other branches of mathematics are interpreted. Rather, it is a purely coincidental fact --- a pragmatic choice by mathematicians, indeed --- that number theory, real analysis, \emph{et cetera}, are treated as sub-structures in $\ZFC$ and utilise the same axioms from $\ZFC$ for the sets of numbers or the sets of reals.

However, it is still possible to rescue Maddy's naturalism from an anti-pluralist direction, by extending $\ZFC$ using axioms of form
\[\Con\mleft(\ZFC + \text{some large cardinal axiom}\mright)\]
instead of the large cardinal axioms themselves. This is a sensible concession from the overambitious commitment to the large cardinal axioms since, as Feferman and Friedman both noticed in \cite{new-axioms-panel}, many of the connections between various areas of mathematics and the large cardinal axioms are actually established on the consistency claims, instead of existential ones. The axioms will still fulfill Maddy and Steel's goal of maximising the interpretability of the theory, while providing a position one can argue more convincingly for: as I remarked in \autoref{subsec:large-cardinal-independence}, regarding many large cardinal axioms that has been heavily studied, such as the Woodin cardinals, the case for their consistency has been almost as strong as that for the consistency of $\ZFC$. Therefore, if one is devoted to the consistency of some large cardinal axioms, it is reasonable that the \emph{correct} foundational theory reflects this.

This extension of $\ZFC$ is still non-conservative over the theory of some branches of mathematics formalised in $\ZFC$. This is evident through \autoref{thm:friedman-proposition-equivalent-to-consistency-of-subtle-cardinals} in this thesis, a result by Friedman that his Proposition B is equivalent to the consistency of several large cardinal axioms. However, it is much more reasonable to accept Proposition B as a theorem of standard mathematics here because the evidence for the consistency of the large cardinal axioms supports that the consistency statement is an axiom one may use in classical reasoning, in exactly the same nature how one's intuition backs the acceptance of the axioms of $\ZFC$. It can be suggested that we just failed to notice the naturalness of the new hypotheses and hence the possibility of proving Proposition B before they are eventually introduced.

However, the success of such a procedure is limited. Namely, one may notice that \autoref{thm:delta-1-2-determinacy-equivalence} to \autoref{thm:adlr-determinacy-equivalence} of Woodin's results on the axioms of determinacy take a similar form to \autoref{thm:friedman-proposition-equivalent-to-consistency-of-subtle-cardinals} and do not require that Woodin cardinals actually exist. Yet it is doubtful whether one's intuition about the consistency of certain axioms can extend so far to the existence of inner models with specific complex properties. Hence, it is likely that, through the aforementioned means, one can at best justify the axiom $\Con\mleft(\ZFC + \PCWC\mright)$ and invoke the completeness theorem for first-order logic to work in a different model where the Woodin cardinals exist.

Therefore, I wish to conclude that, in most cases where large cardinal axioms are needed, a foundational theory that can be justified naturally through mathematical practices cannot actually free one from the practice of beginning his theorem with ``assume some $\varphi$-cardinals exist'' and thus displays no significant advantage over a pluralist theory based on the Carnapian viewpoint.

I believe that this section thus established a Carnapian philosophy as a natural description of the contemporary mathematical practice by allowing both the set theorists to follow their enthusiasm in the multitude of strong theories for the universe and the other mathematicians to retain their classical, determined notion of proofs. I shall end here with a Wittgensteinian summary Maddy wrote for her naturalist project in \cite{maddy-maximize}:

\begin{quote}
	the best confirmation of success would be for the mathematician to shrug and say, `Of course, everybody knows that.'
\end{quote}

and its striking correlation to how Carnap ended the essay \emph{ESO}, reiterating his principle of tolerance:

\begin{quote}
	let us grant to those who work in any special field of investigation the freedom to use any form of expression which seems useful to them.
\end{quote}

\section{Conclusion}

In conclusion, this thesis has examined the plausibility of a Carnapian position in the philosophy of mathematics, in the context of the various problems surrounding the large cardinal axioms in Zermelo--Fraenkel set theory with the axiom of choice. I have argued that the perspective is a sound one across multiple scenarios where different other attempts reveal their weaknesses.

Section 2 of this thesis was dedicated to the revival of Carnap's project for the specific purpose of analysing mathematical practices, and defended the theory against Quine's and G\"odel's criticisms. In section 3, I rendered the large cardinal axioms on a hierarchy of interpretability and argued that anti-pluralists are faced with a heavy task of justifying their ontology of sets. Their arguments are subsequently dismantled in section 4 to demonstrate the disadvantage of such a position in the philosophy of set theory. I then presented counter-arguments in section 5 involving the implications of large cardinals in more common areas of mathematics, which undermined classical pluralism in favour of the Carnapian position. Finally, Maddy's naturalism in mathematics is assessed in section 6, where I argued that Carnap's philosophy constitutes a more natural perspective on the contemporary mathematical practices.

Therefore, I believe that my arguments establish a Carnapian philosophy of mathematics as a reasonable position to maintain and a balanced resolution between the quarreling pluralists and anti-pluralists. I hope that this analysis in set theory witnesses the fact that mathematics has evolved from an ancient, Platonist science of the numbers and geometries to a modern discipline of the study of any precise patterns: Carnap's internal, formal theory of logical consequences represents the mathematical rigour in the studies, while his loose criteria for the frameworks echo the mathematicians' delving into independence, interpretability and many other more intricated areas of meta-logic in the past century. I believe it is this combination of caution and tolerance that has driven many new discoveries in mathematics and will undoubtedly motivate more.

\begin{thebibliography}{99}
	\addcontentsline{toc}{section}{References}

	\bibitem{omega-logic-primer} Bagaria, J., Castells, N., \& Larson, P. (2006). An $\Omega$-logic Primer. \textit{Set Theory: Centre de Recerca Matem\`atica Barcelona, 2003--2004}, 1--28.

	\bibitem{barton-large-cardinal-restrictive} Barton, N. (2020). Are Large Cardinal Axioms Restrictive? Preprint fetched at \url{https://philpapers.org/rec/BARALC-5}.

	\bibitem{ineffable-cardinals} Baumgartner, J. (1975). Ineffability properties of cardinals I. In: Hajnal, A., Rado, R., \& S\'os, V.\ T.\ (eds). \textit{Infinite and Finite Sets} (Colloquia Mathematica Societatis
	J\'anos Bolyai ; 10), 109--130. Amsterdam: North-Holland.

	\bibitem{beller-coding-universe} Beller, A., Jensen, R., \& Welch, P. (1982). \textit{Coding the Universe} (London Mathematical Society lecture note series ; 47). Cambridge: Cambridge University Press.

	\bibitem{carnap-pls} Carnap, R. (1935). \textit{Philosophy and Logical Syntax} (Psyche miniatures. General series ; no.\ 70). London: Kegan Paul, Trench, Tr\"ubner.

	\bibitem{logical-syntax} Carnap, R., \& Smeaton, A. (1937). \textit{The logical syntax of language} (International library of psychology, philosophy, and scientific method). London: Kegan Paul, Trench, Tr\"ubner.

	\bibitem{carnap-eso} Carnap, R. (1950). Empiricism, Semantics, and Ontology. \textit{Revue Internationale De Philosophie}, 4(11), 20--40.

	\bibitem{cohen-ch-1} Cohen, P.\ J. (1963). The Independence of the Continuum Hypothesis. \textit{Proceedings of the National Academy of Sciences}, 50(6), 1143--1148.

	\bibitem{cohen-ch-2} Cohen, P.\ J. (1964). The Independence of the Continuum Hypothesis, II. \textit{Proceedings of the National Academy of Sciences}, 51(1), 105--110.

	\bibitem{erdos-hajnal} Erd\H{o}s, P., \& Hajnal, A. (1966). On a problem of B.\ J\'onsson. \textit{Bulletin de l'Acad\'emie Polonaise
		des Sciences. S\'erie des Sciences Math\'matiques, Astronomiques et Physiques}, 14, 19--23.

	\bibitem{partition-calculus} Erd\H{o}s, P., \& Rado, R. (1956). A partition calculus in set theory. \emph{Bulletin (New Series) of the American Mathematical Society}, 62(5), 427--489.

	\bibitem{feferman-new-axioms} Feferman, S. (1999). Does Mathematics Need New Axioms? \textit{The American Mathematical Monthly}, 106(2), 99.

	\bibitem{new-axioms-panel} Feferman, S., Friedman, H.\ M., Maddy, P., \& Steel, J.\ R. (2000). Does Mathematics Need New Axioms? \textit{The Bulletin of Symbolic Logic}, 6(4), 401--446.

	\bibitem{friedman-finite-independence} Friedman, H.\ M. (1998). Finite functions and the necessary use of large cardinals. \textit{Annals of Mathematics}, 148(3), 803--893.

	\bibitem{subtle-cardinals} Friedman, H.\ M. (2001). Subtle cardinals and linear orderings. \textit{Annals of Pure and Applied Logic}, 107(1), 1--34.

	\bibitem{friedman-imh-06} Friedman, S.-D. (2006). Internal Consistency and the Inner Model Hypothesis. \textit{Bulletin of Symbolic Logic}, 12(4), 591--600.

	\bibitem{friedman-imh-08} Friedman, S.-D., Welch, P., \& Woodin, W.\ H. (2008). On the Consistency Strength of the Inner Model Hypothesis. \textit{The Journal of Symbolic Logic}, 73(2), 391--400.

	\bibitem{gale-stewart-infinite-games} Gale, D., \& Stewart, F. (2016). Infinite Games with Perfect Information. In: Kuhn, H., \& Tucker, A.\ (eds). \textit{Contributions to the Theory of Games (AM-28), Volume II}, 245--266. Princeton: Princeton University Press.

	\bibitem{godel-ch} Gödel, K. (1940). \textit{The Consistency of the Continuum-Hypothesis}. Princeton University Press.

	\bibitem{godel-continuum-axioms} G\"odel, K. (1964). What is Cantor's Continuum Problem. \textit{Kurt G\"odel Collected Works Volume II: Publications 1938--1974}, 254--270. Oxford University Press.

	\bibitem{godel-positivism} G\"odel, K. (1995). Is mathematics syntax of language? \textit{Kurt G\"odel Collected Works Volume III: Unpublished Essays and Lectures}, 334--356. Oxford University Press.

	\bibitem{goldfarb-positivism} Goldfarb, W. (1996). The Philosophy of Mathematicsin Early Positivism. \textit{Origins of logical empiricism} (Minnesota studies in the philosophy of science ; 16), 213--230. Minneapolis ; London: University of Minnesota Press.

	\bibitem{jech-set-theory} Jech, T. (1997). \textit{Set Theory} (2nd corr.\ ed., Perspectives in mathematical logic). Berlin: Springer-Verlag.

	\bibitem{higher-infinite} Kanamori, A. (2009). \textit{The Higher Infinite: Large cardinals in set theory from their beginnings} (2nd ed., Springer monographs in mathematics). Berlin: Springer-Verlag.

	\bibitem{koellner-absolute-undecidability} Koellner P. (2006). On the Question of Absolute Undecidability. \textit{Philosophia Mathematica}, 14(2), 153--188.

	\bibitem{reflection-principles} Koellner, P. (2009). On reflection principles. \textit{Annals of Pure and Applied Logic}, 157(2), 206--219.

	\bibitem{koellner-pluralism} Koellner, P. (2009). Truth in Mathematics: The Question of Pluralism. \textit{New Waves in Philosophy of Mathematics} (New waves in philosophy), 80--116. Basingstoke ; New York: Palgrave Macmillan.

	\bibitem{koellner-determinacy} Koellner, P. (2014). Large Cardinals and Determinacy. \textit{The Stanford Encyclopedia of Philosophy} (Spring 2014 Edition). Archive edition at \url{https://plato.stanford.edu/archives/spr2014/entries/large-cardinals-determinacy}.

	\bibitem{large-cardinals-from-determinacy} Koellner, P., \& Woodin, W.\ H. (2010). Large Cardinals from Determinacy. In: Foreman, M., \& Kanamori, A.\ (eds). \textit{Handbook of Set Theory}, 1951--2119. Springer, Dordrecht.

	\bibitem{kunen-elementary-embeddings} Kunen, K. (1971). Elementary embeddings and infinitary combinatorics. \textit{The Journal of Symbolic Logic}, 36(3), 407--413.

	\bibitem{luzin-analytic-sets} Luzin, N.\ N. (1917). Sur la classification de M.\ Baire. \textit{Comptes Rendus de l'Acad\'emie des Sciences, S\'erie I}, 164, 91--94.

	\bibitem{maddy-naturalism} Maddy, P. (2000). \textit{Naturalism in mathematics}. Oxford: Oxford University Press.

	\bibitem{maddy-maximize} Maddy, P. (2017). $V = L$ and Maximize. In: Makowsky, J., \& Ravve, E.\ (eds). \textit{Logic Colloquium '95: Proceedings of the Annual European Summer Meeting of the Association of Symbolic Logic, Haifa, Israel, August 9--18, 1995} (Lecture notes in logic ; 11), 134-152. Cambridge: Cambridge University Press.

	\bibitem{martin-measurable-cardinal-determinacy} Martin, D.\ A. (1970). Measurable cardinals and analytic games. \textit{Fundamenta Mathematicae}, 66(3), 287--291.

	\bibitem{woodin-cardinal-to-pd} Martin, D.\ A., \& Steel, J.\ R. (1989). A proof of projective determinacy. \textit{Journal of the American Mathematical Society}, 2(1), 71--125.

	\bibitem{mendelson} Mendelson, E. (1964). \textit{Introduction to mathematical logic} (University series in undergraduate mathematics). New York ; London: Van Nostrand Reinhold.

	\bibitem{mitchell-inner-model-theory} Mitchell, W.\ J. (2010). Beginning Inner Model Theory. In: Foreman, M., \& Kanamori, A.\ (eds). \textit{Handbook of Set Theory}, 1449--1495. Springer, Dordrecht.

	\bibitem{descriptive-set-theory} Moschovakis, Y.\ N. (2009). \textit{Descriptive Set Theory} (Mathematical surveys and monographs ; 155). American Mathematical Society.

	\bibitem{lindstrom-incompleteness} Lindstr\"om, P. (2016). \textit{Aspects of Incompleteness} (Lecture notes in logic ; 10). Cambridge University Press.

	\bibitem{two-dogmas} Quine, W.\ V. (1951). Two Dogmas of Empiricism. \textit{Philosophical Review}, 60(1), 20--43.

	\bibitem{quine-to-carnap} Quine, W.\ V. (1960). Carnap and logical truth. \textit{Synthese}, 12(4), 350--374.

	\bibitem{reinhardt-thesis} Reinhardt, W. (1967). \textit{Topics in the Metamathematics of Set Theory}. PhD thesis, University of California, Berkeley.

	\bibitem{scotts-trick} Scott, D. (1955). Definitions by abstraction in axiomatic set theory. \textit{Bulletin of the American Mathematical Society}, 61(5), 442.

	\bibitem{woodin-cardinals} Shelah, S., \& Woodin, W.\ H. (1990). Large cardinals imply that every reasonably definable set of reals is lebesgue measurable. \textit{Israel Journal of Mathematics}, 70, 381--394.

	\bibitem{solovay-force-regularity} Solovay, R.\ M. (1970). A Model of Set-Theory in Which Every Set of Reals is Lebesgue Measurable. \textit{Annals of Mathematics}, 92(1), 1--56.

	\bibitem{suslin-analytic-sets} Suslin, M. (1917). Sur une d\'efinition des ensembles mesurables B sans nombres transfinis. \textit{Comptes Rendus de l'Acad\'emie des Sciences, S\'erie I}, 164, 88--91.

	\bibitem{tait-cardinals-from-below} Tait, W.\ W. (2005). Constructing Cardinals from Below. In: Tait, W.\ W. \textit{The provenance of pure reason: Essays in the philosophy of mathematics and its history} (Logic and computation in philosophy), 133--154. Oxford ; New York: Oxford University Press.

	\bibitem{wang-large-sets} Wang, H. (1977). Large Sets. In: Butts, R.\ E., \& Hintikka, J.\ (eds). \textit{Logic, Foundations of Mathematics, and Computability Theory: Part one of the proceedings of the Fifth International Congress of Logic, Methodology and Philosophy of Science, London, Ontario, Canada, 1975} (University of Western Ontario series in philosophy of science ; 9), 309--333. Dordrecht: Reidel.

	\bibitem{woodin-de-gruyter} Woodin, W.\ H. (2010). \textit{The Axiom of Determinacy, Forcing Axioms, and the Nonstationary Ideal}. Berlin, New York: De Gruyter.

\end{thebibliography}

\end{document}