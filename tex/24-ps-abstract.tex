%?deps asl.cls

% workaround for asl.cls compile error
\let\negmedspace\undefined
\let\negthickspace\undefined

\documentclass[bsl,meeting]{asl}

\AbstractsOn

\pagestyle{plain}

\def\urladdr#1{\endgraf\noindent{\it URL Address}: {\tt #1}.}

\newcommand{\NP}{}

\begin{document}
\thispagestyle{empty}

\NP
\absauth{Shuwei Wang}
\meettitle{The global well-ordering on Weaver's third-order conceptual mathematics}
\affil{School of Mathematics, University of Leeds, Leeds LS2 9JT, UK}
\meetemail{mmsw@leeds.ac.uk}

Around 2010, Nik Weaver proposed quite a few novel candidates for the formal foundation of mainstream mathematics, motivated by the philosophical stance he calls \emph{mathematical conceptualism}, effectively a revival of the predicativism trend that advocates for founding mathematics upon only a small portion of the set hierarchy that is ``built up from below'', seeing the rest as ill-defined due to the use of impredicative definitions.

This talk will specifically be concerned with Weaver's formal system $\mathrm{CM}$ in \cite{weaver09-cm} of intuitionistic third-order arithmetic with a classical first-order fragment, characterised by a comprehension scheme for all decidable formulae and second-order dependent choice. In the same paper, Weaver also suggested an extension of the theory by adding a primitive global well-ordering on all second-order objects. However, while he gave an abundance of examples on how various traditional topics in mathematics, such as topology, functional analysis and measure theory, can be formalised in $\mathrm{CM}$, the implications of the proposed extension remain unexplored.

In this talk, I will give a proof-theoretic analysis of Weaver's theory, as well as the effect of including the extra axioms for the global well-ordering. This should bound Weaver's conceptualism between the strength of the base theory $\mathrm{CM}$ (with a proof-theoretic ordinal of $\varphi_{\varepsilon_0}0$) and the impredicative extension with a full transfinite induction axiom on the global well-ordering (whose strength is at the Bachmann--Howard ordinal). Interestingly, Weaver did not make clear where exactly the distinction between predicativism and impredicativism lies (though in a different paper \cite{weaver09-predicativity} he did argue that his philosophy can justify certain theories with a proof-theoretic ordinal above $\Gamma_0$), and I will attempt to interpret what Weaver might accept as a predicative fragment of the full-strength theory.

I will additionally continue Weaver's discussion of mainstream mathematics, by demonstrating that sufficient transfinite induction on the global well-ordering can play the role of the axiom of choice for mathematics formalised in $\mathrm{CM}$. In the extended theory, one can prove a restricted analogue of Zorn's lemma, from which many classical results can be derived, including the existence of a basis or the Hahn--Banach theorem for any Banach space over a complete separable $\mathbb{R}$-normed field.

\begin{thebibliography}{10}

  \bibitem{weaver09-cm}
  {\scshape Nik Weaver},
  {\itshape Axiomatizing mathematical conceptualism in third order arithmetic}, 2009.
  Preprint available at \texttt{arXiv:0905.1675 [math.HO]}.

  \bibitem{weaver09-predicativity}
  {\scshape Nik Weaver},
  {\itshape Predicativity beyond {$\Gamma_0$}},
  2009.
  Preprint available at \texttt{arXiv:math/0509244 [math.LO]}.

\end{thebibliography}

\vspace*{-0.5\baselineskip}

\end{document}
